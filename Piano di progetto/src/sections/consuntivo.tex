\section{Consuntivo di periodo}
In questa sezione del documento viene riportata la distribuzione reale delle risorse del gruppo nei vari periodi dello sviluppo del progetto, confrontandole con quelle preventivate.\\
Il bilancio potrà essere:
\begin{itemize}
	\item \textbf{Positivo} se il costo totale del periodo analizzato è minore di quello preventivato;
	\item \textbf{In pari} se il costo totale del periodo analizzato è uguale a quello preventivato;
	\item \textbf{Negativo} se il costo totale del periodo analizzato è superiore di quello preventivato.
\end{itemize}

\subsection{Analisi}
%
% ----------------------------------------------------------------------------------------------------------------
\subsubsection{Consuntivo sprint I}

Questa tabella mostra come le risorse del gruppo sono state utilizzate realmente nel primo sprint del progetto, svolto nel periodo di analisi, e le confronta con quelle preventivate.
% ----------------------------------------------------------------------------------------------------------------

\setlength\extrarowheight{5pt}
\rowcolors{2}{gray!10}{gray!40}
\begin{tabularx}{\textwidth}{|c|XcXX|c|}
	\hline
	\rowcolor{white}
	\textbf{Ruolo} & \textbf{Ore preventivate} & \textbf{Ore reali} & \textbf{Costo preventivato (€)} & \textbf{Costo reale (€)} & \textbf{Errore (€)} \\
	\hline
	Responsabile &6&6&180&180&+0\\
	Amministratore &26&28 (+2)&520&560&+40\\
	Analista &28&26 (-2)&700&650&-50\\
	Verificatore &-&-&-&-&-\\
	Programmatore &-&-&-&-&-\\
	Progettista &-&-&-&-&- \\
	\hline
	Totale &60&60&1400&1390&-10\\
	\hline
	\rowcolor{white}
	\caption{Consuntivo ore e costi per ruolo del primo sprint}
\end{tabularx}
\subsubsection{Analisi retrospettiva sprint I}
Nello sprint I le ore preventivate per ogni ruolo sono state piuttosto accurate rispetto a quelle reali, tenendo conto che il gruppo ha scelto di dedicare delle ore in più al ruolo di amministratore dal momento che all'inizio del progetto l'organizzazione dell'ambiente di lavoro e la definizione del way of working hanno richiesto più tempo. Avendo sottratto delle ore dal ruolo dell'analista, il gruppo è riuscito a non sforare i costi preventivati.


% ----------------------------------------------------------------------------------------------------------------

\newpage
\subsubsection{Consuntivo sprint II}
Questa tabella mostra come le risorse del gruppo sono state utilizzate realmente nel secondo sprint del progetto, svolto nel periodo di analisi, e le confronta con quelle preventivate.
% ----------------------------------------------------------------------------------------------------------------

\setlength\extrarowheight{5pt}
\rowcolors{2}{gray!10}{gray!40}
\begin{tabularx}{\textwidth}{|c|XcXX|c|}
	\hline
	\rowcolor{white}
	\textbf{Ruolo} & \textbf{Ore preventivate} & \textbf{Ore reali} & \textbf{Costo preventivato (€)} & \textbf{Costo reale (€)} & \textbf{Errore (€)} \\
	\hline
	Responsabile &6&8 (+2)&180&240&+60\\
	Amministratore &16&16&320&320&+0\\
	Analista &41&44 (+3)&1025&1100&+75\\
	Verificatore &27&25 (-2)&405&375&-30\\
	Programmatore &-&-&-&-&-\\
	Progettista &-&-&-&-&-\\
	\hline
	Totale &90&93 (+3)&1930&2035&+105\\
	\hline
	\rowcolor{white}
	\caption{Consuntivo ore e costi per ruolo del secondo sprint}
\end{tabularx}
\subsubsection{Analisi retrospettiva sprint II}
Nello sprint II si è reso necessario recuperare le ore dell'analista non effettuate nello sprint precedente. Le ore aggiuntive del responsabile sono state utilizzate per cercare di suddividere i compiti fra i membri del gruppo nella maniera più efficiente ed efficace possibile.

% ----------------------------------------------------------------------------------------------------------------
\newpage
\subsubsection{Consuntivo sprint III}
Questa tabella mostra come le risorse del gruppo sono state utilizzate realmente nel terzo sprint del progetto, svolto nel periodo di analisi, e le confronta con quelle preventivate.
% ----------------------------------------------------------------------------------------------------------------

\setlength\extrarowheight{5pt}
\rowcolors{2}{gray!10}{gray!40}
\begin{tabularx}{\textwidth}{|c|XcXX|c|}
	\hline
	\rowcolor{white}
	\textbf{Ruolo} & \textbf{Ore preventivate} & \textbf{Ore reali} & \textbf{Costo preventivato (€)} & \textbf{Costo reale (€)} & \textbf{Errore (€)} \\
	\hline
	Responsabile &3&3&90&90&+0\\
	Amministratore &6&7 (+1)&120&140&+20\\
	Analista &8&6 (-2)&200&150&-50\\
	Verificatore &13&15 (+2)&195&225&+30\\
	Programmatore &-&-&-&-&-\\
	Progettista &-&-&-&-&- \\
	\hline
	Totale &30&31 (+1)&605&605&+0\\
	\hline
	\rowcolor{white}
	\caption{Consuntivo ore e costi per ruolo del terzo sprint}
\end{tabularx}
\subsubsection{Analisi retrospettiva sprint III}
Nello sprint III si è scelto di dare più importanza al ruolo del verificatore, fondamentale per consolidare quanto fatto fino a quel momento, rinunciando ad alcune ore dell'analista che nello sprint precedente ha avuto tempo sufficiente per svolgere le sue attività. Nel complesso non ci sono stati aumenti dei costi per questo sprint.

% ----------------------------------------------------------------------------------------------------------------
\newpage
\subsubsection{Consuntivo sprint V}
Questa tabella mostra come le risorse del gruppo sono state utilizzate realmente nel quinto sprint del progetto, svolto nel periodo di analisi in parallelo al periodo di produzione del proof of concept, e le confronta con quelle preventivate.
% ----------------------------------------------------------------------------------------------------------------

\setlength\extrarowheight{5pt}
\rowcolors{2}{gray!10}{gray!40}
\begin{tabularx}{\textwidth}{|c|XcXX|c|}
	\hline
	\rowcolor{white}
	\textbf{Ruolo} & \textbf{Ore preventivate} & \textbf{Ore reali} & \textbf{Costo preventivato (€)} & \textbf{Costo reale (€)} & \textbf{Errore (€)} \\
	\hline
	Responsabile &3&4 (+1)&90&120&+30\\
	Amministratore &3& 3&60&60&+0\\
	Analista &5&4 (-1)&125&100&-25\\
	Verificatore &7&8 (+1)&105&120&+15\\
	Programmatore &-&-&-&-&-\\
	Progettista &-&-&-&-&- \\
	\hline
	Totale &18&19 (+1)&355&375&+20\\
	\hline
	\rowcolor{white}
	\caption{Consuntivo ore e costi per ruolo del quinto sprint}
\end{tabularx}
\subsubsection{Analisi retrospettiva sprint V}
Nello sprint V le ore preventivate per ogni ruolo sono state piuttosto accurate rispetto a quelle reali, con delle differenze minime che hanno portato ad un aumento dei costi poco significativo.

% ----------------------------------------------------------------------------------------------------------------
\newpage
\subsubsection{Consuntivo sprint VI}
Questa tabella mostra come le risorse del gruppo sono state utilizzate realmente nel sesto sprint del progetto, svolto nel periodo di analisi e le confronta con quelle preventivate.
% ----------------------------------------------------------------------------------------------------------------

\setlength\extrarowheight{5pt}
\rowcolors{2}{gray!10}{gray!40}
\begin{tabularx}{\textwidth}{|c|XcXX|c|}
	\hline
	\rowcolor{white}
	\textbf{Ruolo} & \textbf{Ore preventivate} & \textbf{Ore reali} & \textbf{Costo preventivato (€)} & \textbf{Costo reale (€)} & \textbf{Errore (€)} \\
	\hline
	Responsabile &2&4 (+2)&60&120&+60\\
	Amministratore &4&3 (-1)&80&60&-20\\
	Analista &-&1 (+1)&0&25&+25\\
	Verificatore &14&14&210&210&+0\\
	Programmatore &-&-&-&-&-\\
	Progettista &-&-&-&-&- \\
	\hline
	Totale &20&21 (+1)&350&415&+65\\
	\hline
	\rowcolor{white}
	\caption{Consuntivo ore e costi per ruolo del sesto sprint}
\end{tabularx}
\subsubsection{Analisi retrospettiva sprint VI}
Nello sprint VI la differenza più significativa tra le ore preventivate e quelle reali è quella evidenziata nel ruolo del responsabile, che ha dovuto pianificare tanti compiti finalizzati a preparare il materiale necessario per la candidatura alla revisione RTB vista l'imminente scadenza. Ciò ha comportato un aumento dei costi rispetto al preventivo.

% ----------------------------------------------------------------------------------------------------------------
\newpage
\subsubsection{Consuntivo periodo di analisi}
Questa tabella mostra come le risorse del gruppo sono state utilizzate realmente nel periodo di analisi e le confronta con quelle preventivate.
% ----------------------------------------------------------------------------------------------------------------

\setlength\extrarowheight{5pt}
\rowcolors{2}{gray!10}{gray!40}
\begin{tabularx}{\textwidth}{|c|XcXX|c|}
	\hline
	\rowcolor{white}
	\textbf{Ruolo} & \textbf{Ore preventivate} & \textbf{Ore reali} & \textbf{Costo preventivato (€)} & \textbf{Costo reale (€)} & \textbf{Errore (€)} \\
	\hline
	Responsabile &18&21 (+3)&540&630&+90\\
	Amministratore &51&54 (+3)&1020&1080&+60\\
	Analista &82&80 (-2)&2050&2000&-50\\
	Verificatore &47&48 (+1)&705&720&+15\\
	Programmatore &-&-&-&-&-\\
	Progettista &-&-&-&-&-\\
	\hline
	Totale &198&203 (+5)&4315&4430&+115\\
	\hline
	\rowcolor{white}
	\caption{Consuntivo ore e costi per ruolo durante il periodo di analisi}
\end{tabularx}

\subsubsection{Conclusioni per il periodo di analisi}
Valutando con occhio critico il consuntivo del periodo di analisi, gli errori più significativi sono stati i seguenti:
\begin{itemize}
	\item Il ruolo di responsabile ha richiesto ore aggiuntive per poter monitorare l'avanzamento delle attività, viste le dimensioni del progetto e la poca esperienza dei membri del gruppo nella gestione di progetto;
    \item Il ruolo di amministratore ha richiesto ore aggiuntive per poter definire un way of working sufficiente per riuscire a far collaborare efficacemente tutti i membri del team.
\end{itemize}
Il gruppo si impegnerà pertanto a ridurre i costi durante i periodi successivi del progetto didattico.
% ----------------------------------------------------------------------------------------------------------------
\newpage
\subsection{Produzione del proof of concept}
%
% ----------------------------------------------------------------------------------------------------------------
\subsubsection{Consuntivo sprint IV}

Questa tabella mostra come le risorse del gruppo sono state utilizzate realmente nel quarto sprint del progetto, svolto nel periodo di produzione del proof of concept, e le confronta con quelle preventivate.
% ----------------------------------------------------------------------------------------------------------------

\setlength\extrarowheight{5pt}
\rowcolors{2}{gray!10}{gray!40}
\begin{tabularx}{\textwidth}{|c|XcXX|c|}
	\hline
	\rowcolor{white}
	\textbf{Ruolo} & \textbf{Ore preventivate} & \textbf{Ore reali} & \textbf{Costo preventivato (€)} & \textbf{Costo reale (€)} & \textbf{Errore (€)} \\
	\hline
	Responsabile &3&2 (-1)&90&60&-30\\
	Amministratore &2&2 &40&40&+0\\
	Analista &2&2&50&50&+0\\
	Verificatore &2&2&30&30&+0\\
	Programmatore &2&4 (+2)&30&60&+30\\
	Progettista &7&6 (-1)&175&150&-25 \\
	\hline
	Totale &18&18&415&390&-25\\
	\hline
	\rowcolor{white}
	\caption{Consuntivo ore e costi per ruolo del quarto sprint}
\end{tabularx}
\subsubsection{Analisi retrospettiva sprint IV}
Nello sprint IV non sono state evidenziate differenze significative tra le ore prevenivate e quelle reali; anzi, cercando di diminuire le ore di responsabile e progettista, si è riusciti a risparmiare sui costi totali.

% ----------------------------------------------------------------------------------------------------------------

\newpage
\subsubsection{Consuntivo sprint V}
Questa tabella mostra come le risorse del gruppo sono state utilizzate realmente nel quinto sprint del progetto, svolto nel periodo di produzione del proof of concept in parallelo al periodo di analisi, e le confronta con quelle preventivate.
% ----------------------------------------------------------------------------------------------------------------

\setlength\extrarowheight{5pt}
\rowcolors{2}{gray!10}{gray!40}
\begin{tabularx}{\textwidth}{|c|XcXX|c|}
	\hline
	\rowcolor{white}
	\textbf{Ruolo} & \textbf{Ore preventivate} & \textbf{Ore reali} & \textbf{Costo preventivato (€)} & \textbf{Costo reale (€)} & \textbf{Errore (€)} \\
	\hline
	Responsabile &3&3&90&90&+0\\
	Amministratore &3&2 (-1)&60&40&-20\\
	Analista &2&2&50&50&+0\\
	Verificatore &7&8 (+1)&105&120&+15\\
	Programmatore &13&16 (+3)&195&240&+45\\
	Progettista &8&6 (-2)&200&150&-50 \\
	\hline
	Totale &36&37 (+1)&700&690&-10\\
	\hline
	\rowcolor{white}
	\caption{Consuntivo ore e costi per ruolo del quinto sprint}
\end{tabularx}
\subsubsection{Analisi retrospettiva sprint V}
Nello sprint V sono state necessarie più ore di programmatore rispetto a quelle preventivate in quanto durante lo sviluppo del PoC alcuni membri del team hanno riscontrato difficoltà nel portare a termine le attività di codifica assegnate, data la poca esperienza con alcune delle tecnologie scelte. Essendo però riusciti a diminuire le ore di amministratore e progettista, a fine sprint non si è verificato un aumento dei costi totali.
% ----------------------------------------------------------------------------------------------------------------
\newpage
\subsubsection{Consuntivo periodo di produzione del proof of concept}
Questa tabella mostra come le risorse del gruppo sono state utilizzate realmente nel periodo di produzione del proof of concept e le confronta con quelle preventivate.
% ----------------------------------------------------------------------------------------------------------------

\setlength\extrarowheight{5pt}
\rowcolors{2}{gray!10}{gray!40}
\begin{tabularx}{\textwidth}{|c|XcXX|c|}
	\hline
	\rowcolor{white}
	\textbf{Ruolo} & \textbf{Ore preventivate} & \textbf{Ore reali} & \textbf{Costo preventivato (€)} & \textbf{Costo reale (€)} & \textbf{Errore (€)} \\
	\hline
	Responsabile &6&5 (-1)&180&150&-30\\
	Amministratore &5&4 (-1)&100&80&-20\\
	Analista &4&4&100&100&+0\\
	Verificatore &9&10 (+1)&135&150&+15\\
	Programmatore &15&20 (+5)&225&300&+75\\
	Progettista &15&12 (-3)&375&300&-75\\
	\hline
	Totale &54&55 (+1)&1115&1080&-35\\
	\hline
	\rowcolor{white}
	\caption{Consuntivo ore e costi per ruolo durante il periodo di produzione del proof of concept}
\end{tabularx}

\subsubsection{Conclusioni per il periodo di produzione del proof of concept}
Valutando con occhio critico il consuntivo del periodo di produzione del proof of concept, gli errori più significativi sono stati i seguenti:
\begin{itemize}
	\item Il ruolo di programmatore ha richiesto ore aggiuntive per poter portare a termine la codifica del PoC, e sono state utilizzate per colmare le lacune tecnologiche dei membri del gruppo;
    \item Il ruolo di progettista ha richiesto meno ore rispetto a quelle preventivate dal momento che è stato sufficiente selezionare le tecnologie da includere nella realizzazione del PoC.
\end{itemize}
Il gruppo è pertanto riuscito a compensare una parte dei costi che hanno superato il preventivo durante la fase di analisi.
% ----------------------------------------------------------------------------------------------------------------
\newpage
