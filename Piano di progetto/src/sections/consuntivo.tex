\section{Consuntivo di periodo}
In questa sezione del documento viene riportata la distribuzione reale delle risorse del gruppo nei vari periodi dello sviluppo del progetto, confrontandole con quelle preventivate.\\
Il bilancio potrà essere:
\begin{itemize}
	\item \textbf{Positivo} se il costo totale del periodo analizzato è minore di quello preventivato;
	\item \textbf{In pari} se il costo totale del periodo analizzato è uguale a quello preventivato;
	\item \textbf{Negativo} se il costo totale del periodo analizzato è superiore di quello preventivato.
\end{itemize}

\subsection{Analisi}
%
% ----------------------------------------------------------------------------------------------------------------
\subsubsection{Consuntivo sprint\textsubscript{G} I}

Questa tabella mostra come le risorse del gruppo sono state utilizzate realmente nel primo sprint\textsubscript{G} del progetto, svolto nel periodo di analisi, e le confronta con quelle preventivate.
% ----------------------------------------------------------------------------------------------------------------

\setlength\extrarowheight{5pt}
\rowcolors{2}{gray!10}{gray!40}
\begin{tabularx}{\textwidth}{|c|XcXX|c|}
	\hline
	\rowcolor{white}
	\textbf{Ruolo} & \textbf{Ore preventivate} & \textbf{Ore reali} & \textbf{Costo preventivato (€)} & \textbf{Costo reale (€)} & \textbf{Errore (€)} \\
	\hline
	Responsabile &6&6&180&180&+0\\
	Amministratore &26&28 (+2)&520&560&+40\\
	Analista &28&26 (-2)&700&650&-50\\
	Verificatore &-&-&-&-&-\\
	Programmatore &-&-&-&-&-\\
	Progettista &-&-&-&-&- \\
	\hline
	Totale &60&60&1400&1390&-10\\
	\hline
	\rowcolor{white}
	\caption{Consuntivo ore e costi per ruolo del primo sprint\textsubscript{G}}
\end{tabularx}
\subsubsection{Analisi retrospettiva sprint\textsubscript{G} I}
Nello sprint\textsubscript{G} I le ore preventivate per ogni ruolo sono state piuttosto accurate rispetto a quelle reali, tenendo conto che il gruppo ha scelto di dedicare delle ore in più al ruolo di amministratore e meno a quello dell'analista per poter definire fin da subito delle basi per il way of working e per avere una buona comprensione dell'ambiente di lavoro scelto. Avendo sottratto delle ore dal ruolo dell'analista, che è stato meno necessario di quanto preventivato in questo primo sprint\textsubscript{G}, il gruppo è riuscito a non sforare i costi preventivati.


% ----------------------------------------------------------------------------------------------------------------

\newpage
\subsubsection{Consuntivo sprint\textsubscript{G} II}
Questa tabella mostra come le risorse del gruppo sono state utilizzate realmente nel secondo sprint\textsubscript{G} del progetto, svolto nel periodo di analisi, e le confronta con quelle preventivate.
% ----------------------------------------------------------------------------------------------------------------

\setlength\extrarowheight{5pt}
\rowcolors{2}{gray!10}{gray!40}
\begin{tabularx}{\textwidth}{|c|XcXX|c|}
	\hline
	\rowcolor{white}
	\textbf{Ruolo} & \textbf{Ore preventivate} & \textbf{Ore reali} & \textbf{Costo preventivato (€)} & \textbf{Costo reale (€)} & \textbf{Errore (€)} \\
	\hline
	Responsabile &6&8 (+2)&180&240&+60\\
	Amministratore &16&16&320&320&+0\\
	Analista &41&44 (+3)&1025&1100&+75\\
	Verificatore &27&25 (-2)&405&375&-30\\
	Programmatore &-&-&-&-&-\\
	Progettista &-&-&-&-&-\\
	\hline
	Totale &90&93 (+3)&1930&2035&+105\\
	\hline
	\rowcolor{white}
	\caption{Consuntivo ore e costi per ruolo del secondo sprint\textsubscript{G}}
\end{tabularx}
\subsubsection{Analisi retrospettiva sprint\textsubscript{G} II}
Nello sprint\textsubscript{G} II si è reso necessario utilizzare ore in più per il ruolo di analista a causa di alcune difficoltà riscontrate nello svolgimento dell'attività di analisi dei requisiti\textsubscript{G}, dovute soprattutto all'inesperienza del gruppo in questo campo. Attraverso alcuni incontri con il proponente e con il professor Cardin, il gruppo pur utilizzando più ore del previsto è riuscita a risolvere i dubbi riscontrati e procedere. Le ore aggiuntive del responsabile sono state utilizzate per l'attività di pianificazione del progetto e delle sue attività, in modo che lo svolgimento di questo potesse essere il più efficiente ed efficace possibile. 

% ----------------------------------------------------------------------------------------------------------------
\newpage
\subsubsection{Consuntivo sprint\textsubscript{G} III}
Questa tabella mostra come le risorse del gruppo sono state utilizzate realmente nel terzo sprint\textsubscript{G} del progetto, svolto nel periodo di analisi, e le confronta con quelle preventivate.
% ----------------------------------------------------------------------------------------------------------------

\setlength\extrarowheight{5pt}
\rowcolors{2}{gray!10}{gray!40}
\begin{tabularx}{\textwidth}{|c|XcXX|c|}
	\hline
	\rowcolor{white}
	\textbf{Ruolo} & \textbf{Ore preventivate} & \textbf{Ore reali} & \textbf{Costo preventivato (€)} & \textbf{Costo reale (€)} & \textbf{Errore (€)} \\
	\hline
	Responsabile &3&3&90&90&+0\\
	Amministratore &6&7 (+1)&120&140&+20\\
	Analista &8&6 (-2)&200&150&-50\\
	Verificatore &13&15 (+2)&195&225&+30\\
	Programmatore &-&-&-&-&-\\
	Progettista &-&-&-&-&- \\
	\hline
	Totale &30&31 (+1)&605&605&+0\\
	\hline
	\rowcolor{white}
	\caption{Consuntivo ore e costi per ruolo del terzo sprint\textsubscript{G}}
\end{tabularx}
\subsubsection{Analisi retrospettiva sprint\textsubscript{G} III}
Nello sprint\textsubscript{G} III si è scelto di dare più importanza al ruolo del verificatore, fondamentale per consolidare quanto fatto fino a quel momento. Considerando che lo sforamento delle ore preventivate all'analista nello sprint\textsubscript{G} precedente ha permesso di avere delle basi solide di analisi dei requisiti\textsubscript{G} già all'inizio di questo sprint\textsubscript{G}, il ruolo di analista ha necessitato di meno ore di quelle preventivate. Nel complesso non ci sono stati aumenti dei costi per questo sprint\textsubscript{G}.

% ----------------------------------------------------------------------------------------------------------------
\newpage
\subsubsection{Consuntivo sprint\textsubscript{G} V}
Questa tabella mostra come le risorse del gruppo sono state utilizzate realmente nel quinto sprint\textsubscript{G} del progetto, svolto nel periodo di analisi in parallelo al periodo di produzione del proof of concept, e le confronta con quelle preventivate.
% ----------------------------------------------------------------------------------------------------------------

\setlength\extrarowheight{5pt}
\rowcolors{2}{gray!10}{gray!40}
\begin{tabularx}{\textwidth}{|c|XcXX|c|}
	\hline
	\rowcolor{white}
	\textbf{Ruolo} & \textbf{Ore preventivate} & \textbf{Ore reali} & \textbf{Costo preventivato (€)} & \textbf{Costo reale (€)} & \textbf{Errore (€)} \\
	\hline
	Responsabile &3&4 (+1)&90&120&+30\\
	Amministratore &3& 3&60&60&+0\\
	Analista &5&4 (-1)&125&100&-25\\
	Verificatore &7&8 (+1)&105&120&+15\\
	Programmatore &-&-&-&-&-\\
	Progettista &-&-&-&-&- \\
	\hline
	Totale &18&19 (+1)&380&400&+20\\
	\hline
	\rowcolor{white}
	\caption{Consuntivo ore e costi per ruolo del quinto sprint\textsubscript{G}}
\end{tabularx}
\subsubsection{Analisi retrospettiva sprint\textsubscript{G} V}
Nello sprint\textsubscript{G} V le ore preventivate per ogni ruolo sono state piuttosto accurate rispetto a quelle reali, con delle differenze minime che hanno portato ad un aumento dei costi poco significativo. Le modifiche ai requisiti\textsubscript{G}, ottenute dal riscontro con i risultati del PoC\textsubscript{G}, sono state meno significative di quanto preventivato. Hanno richiesto invece del tempo in più le attività di verifica\textsubscript{G} dei vari documenti redatti.

% ----------------------------------------------------------------------------------------------------------------
\newpage
\subsubsection{Consuntivo sprint\textsubscript{G} VI}
Questa tabella mostra come le risorse del gruppo sono state utilizzate realmente nel sesto sprint\textsubscript{G} del progetto, svolto nel periodo di analisi e le confronta con quelle preventivate.
% ----------------------------------------------------------------------------------------------------------------

\setlength\extrarowheight{5pt}
\rowcolors{2}{gray!10}{gray!40}
\begin{tabularx}{\textwidth}{|c|XcXX|c|}
	\hline
	\rowcolor{white}
	\textbf{Ruolo} & \textbf{Ore preventivate} & \textbf{Ore reali} & \textbf{Costo preventivato (€)} & \textbf{Costo reale (€)} & \textbf{Errore (€)} \\
	\hline
	Responsabile &2&4 (+2)&60&120&+60\\
	Amministratore &4&3 (-1)&80&60&-20\\
	Analista &-&1 (+1)&0&25&+25\\
	Verificatore &14&14&210&210&+0\\
	Programmatore &-&-&-&-&-\\
	Progettista &-&-&-&-&- \\
	\hline
	Totale &20&21 (+1)&350&415&+65\\
	\hline
	\rowcolor{white}
	\caption{Consuntivo ore e costi per ruolo del sesto sprint\textsubscript{G}}
\end{tabularx}
\subsubsection{Analisi retrospettiva sprint\textsubscript{G} VI}
Nello sprint\textsubscript{G} VI la differenza più significativa tra le ore preventivate e quelle reali è quella evidenziata nel ruolo del responsabile, che ha dovuto compiere alcuni cambiamenti nella pianificazione, dovuta a dei ritardi causati dall'aver sottovalutato gli impegni esterni al progetto e del conseguente rallentamento dello sviluppo di esso. Il responsabile ha dovuto inoltre gestire la divisione dei vari compiti finalizzati a preparare il materiale necessario per la candidatura alla revisione RTB vista l'imminente scadenza. Ciò ha comportato un aumento dei costi rispetto al preventivo.

% ----------------------------------------------------------------------------------------------------------------
\newpage
\subsubsection{Consuntivo periodo di analisi}
Questa tabella mostra come le risorse del gruppo sono state utilizzate realmente nel periodo di analisi e le confronta con quelle preventivate.
% ----------------------------------------------------------------------------------------------------------------

\setlength\extrarowheight{5pt}
\rowcolors{2}{gray!10}{gray!40}
\begin{tabularx}{\textwidth}{|c|XcXX|c|}
	\hline
	\rowcolor{white}
	\textbf{Ruolo} & \textbf{Ore preventivate} & \textbf{Ore reali} & \textbf{Costo preventivato (€)} & \textbf{Costo reale (€)} & \textbf{Errore (€)} \\
	\hline
	Responsabile &18&21 (+3)&540&630&+90\\
	Amministratore &51&54 (+3)&1020&1080&+60\\
	Analista &82&80 (-2)&2050&2000&-50\\
	Verificatore &47&48 (+1)&705&720&+15\\
	Programmatore &-&-&-&-&-\\
	Progettista &-&-&-&-&-\\
	\hline
	Totale &198&203 (+5)&4315&4430&+115\\
	\hline
	\rowcolor{white}
	\caption{Consuntivo ore e costi per ruolo durante il periodo di analisi}
\end{tabularx}

\subsubsection{Conclusioni per il periodo di analisi}
Valutando con occhio critico il consuntivo del periodo di analisi, gli errori più significativi sono stati i seguenti:
\begin{itemize}
	\item Il ruolo di responsabile ha richiesto ore aggiuntive per poter monitorare l'avanzamento delle attività e la pianificazione di esse, viste le dimensioni del progetto e la poca esperienza dei membri del gruppo nella gestione di progetto;
    \item Il ruolo di amministratore ha richiesto ore aggiuntive per poter definire un way of working che potesse essere una solida base allo sviluppo futuro e che riuscisse a far collaborare efficacemente tutti i membri del gruppo.
\end{itemize}
Un punto critico di questo periodo è stato il sottovalutare alcuni rischi riscontrati. Per questo motivo il gruppo si impegnerà a mitigare meglio i rischi analizzati.
% ----------------------------------------------------------------------------------------------------------------
\newpage
\subsection{Produzione del proof of concept}
%
% ----------------------------------------------------------------------------------------------------------------
\subsubsection{Consuntivo sprint\textsubscript{G} IV}

Questa tabella mostra come le risorse del gruppo sono state utilizzate realmente nel quarto sprint\textsubscript{G} del progetto, svolto nel periodo di produzione del proof of concept, e le confronta con quelle preventivate.
% ----------------------------------------------------------------------------------------------------------------

\setlength\extrarowheight{5pt}
\rowcolors{2}{gray!10}{gray!40}
\begin{tabularx}{\textwidth}{|c|XcXX|c|}
	\hline
	\rowcolor{white}
	\textbf{Ruolo} & \textbf{Ore preventivate} & \textbf{Ore reali} & \textbf{Costo preventivato (€)} & \textbf{Costo reale (€)} & \textbf{Errore (€)} \\
	\hline
	Responsabile &3&2 (-1)&90&60&-30\\
	Amministratore &2&2 &40&40&+0\\
	Analista &2&2&50&50&+0\\
	Verificatore &2&2&30&30&+0\\
	Programmatore &2&4 (+2)&30&60&+30\\
	Progettista &7&6 (-1)&175&150&-25 \\
	\hline
	Totale &18&18&415&390&-25\\
	\hline
	\rowcolor{white}
	\caption{Consuntivo ore e costi per ruolo del quarto sprint\textsubscript{G}}
\end{tabularx}
\subsubsection{Analisi retrospettiva sprint\textsubscript{G} IV}
Nello sprint\textsubscript{G} IV non sono state evidenziate differenze significative tra le ore prevenivate e quelle reali. Al contrario si il gruppo è riuscito a definire più chiaramente del previsto in che direzione si dovesse sviluppare il PoC\textsubscript{G} necessitando di conseguenza di meno ore per i ruoli di responsabile e progettista, riuscendo così a risparmiare sui costi totali.

% ----------------------------------------------------------------------------------------------------------------

\newpage
\subsubsection{Consuntivo sprint\textsubscript{G} V}
Questa tabella mostra come le risorse del gruppo sono state utilizzate realmente nel quinto sprint\textsubscript{G} del progetto, svolto nel periodo di produzione del proof of concept in parallelo al periodo di analisi, e le confronta con quelle preventivate.
% ----------------------------------------------------------------------------------------------------------------

\setlength\extrarowheight{5pt}
\rowcolors{2}{gray!10}{gray!40}
\begin{tabularx}{\textwidth}{|c|XcXX|c|}
	\hline
	\rowcolor{white}
	\textbf{Ruolo} & \textbf{Ore preventivate} & \textbf{Ore reali} & \textbf{Costo preventivato (€)} & \textbf{Costo reale (€)} & \textbf{Errore (€)} \\
	\hline
	Responsabile &3&3&90&90&+0\\
	Amministratore &3&2 (-1)&60&40&-20\\
	Analista &2&2&50&50&+0\\
	Verificatore &7&8 (+1)&105&120&+15\\
	Programmatore &13&16 (+3)&195&240&+45\\
	Progettista &8&6 (-2)&200&150&-50 \\
	\hline
	Totale &36&37 (+1)&700&690&-10\\
	\hline
	\rowcolor{white}
	\caption{Consuntivo ore e costi per ruolo del quinto sprint\textsubscript{G}}
\end{tabularx}
\subsubsection{Analisi retrospettiva sprint\textsubscript{G} V}
Nello sprint\textsubscript{G} V sono state necessarie più ore di programmatore rispetto a quelle preventivate in quanto durante lo sviluppo del PoC\textsubscript{G} alcuni membri del team hanno riscontrato difficoltà nel portare a termine le attività di codifica assegnate, data la poca esperienza con alcune delle tecnologie scelte. Essendo però riusciti a diminuire le ore di amministratore e progettista, le quali sono state meno necessarie di quanto preventivato, a fine sprint\textsubscript{G} non si è verificato un aumento dei costi totali.
% ----------------------------------------------------------------------------------------------------------------
\newpage
\subsubsection{Consuntivo periodo di produzione del proof of concept}
Questa tabella mostra come le risorse del gruppo sono state utilizzate realmente nel periodo di produzione del proof of concept e le confronta con quelle preventivate.
% ----------------------------------------------------------------------------------------------------------------

\setlength\extrarowheight{5pt}
\rowcolors{2}{gray!10}{gray!40}
\begin{tabularx}{\textwidth}{|c|XcXX|c|}
	\hline
	\rowcolor{white}
	\textbf{Ruolo} & \textbf{Ore preventivate} & \textbf{Ore reali} & \textbf{Costo preventivato (€)} & \textbf{Costo reale (€)} & \textbf{Errore (€)} \\
	\hline
	Responsabile &6&5 (-1)&180&150&-30\\
	Amministratore &5&4 (-1)&100&80&-20\\
	Analista &4&4&100&100&+0\\
	Verificatore &9&10 (+1)&135&150&+15\\
	Programmatore &15&20 (+5)&225&300&+75\\
	Progettista &15&12 (-3)&375&300&-75\\
	\hline
	Totale &54&55 (+1)&1115&1080&-35\\
	\hline
	\rowcolor{white}
	\caption{Consuntivo ore e costi per ruolo durante il periodo di produzione del proof of concept}
\end{tabularx}

\subsubsection{Conclusioni per il periodo di produzione del proof of concept}


Valutando con occhio critico il consuntivo del periodo di produzione del proof of concept, gli errori più significativi sono stati i seguenti:
\begin{itemize}
	\item Il ruolo di programmatore ha richiesto ore aggiuntive per poter portare a termine la codifica del PoC\textsubscript{G}, e sono state utilizzate per colmare le lacune tecnologiche dei membri del gruppo;
    \item Il ruolo di progettista ha richiesto meno ore rispetto a quelle preventivate dal momento che la scelta delle tecnologie  da includere nella realizzazione del PoC\textsubscript{G} ha richiesto meno tempo del previsto.
\end{itemize}
Il gruppo è pertanto riuscito a compensare una parte dei costi che hanno superato il preventivo durante il periodo di analisi.
% ----------------------------------------------------------------------------------------------------------------
\newpage

\subsection{Preventivo a finire}
Viene di seguito illustrata una comparazione tra il preventivo iniziale del costo del progetto e quello calcolato a seguito dei periodi di analisi e produzione del proof of concept.
\setlength\extrarowheight{5pt}
\rowcolors{2}{gray!10}{gray!40}
\begin{tabularx}{\textwidth}{|c|XcXX|c|}
	\hline
	\rowcolor{white}
	\textbf{Ruolo} & \textbf{Ore preventivate} & \textbf{Ore reali} & \textbf{Costo preventivato (€)} & \textbf{Costo reale (€)} & \textbf{Errore (€)} \\
	\hline
	Responsabile &49&51 (+2)&1470&1530&+60\\
	Amministratore &85&87 (+2)&1700&1740&+40\\
	Analista &89&87 (-2)&2225&2175&-50\\
	Verificatore &138&140 (+2)&2070&2100&+30\\
	Programmatore &102&107 (+5)&1530&1605&+75\\
	Progettista &77&74 (-3)&1925&1850&-75 \\
	\hline
	Totale &540&546&10920&11000&+80\\
	\hline
	\rowcolor{white}
	\caption{Preventivo a finire di ore e costi per ruolo}
\end{tabularx}
