\appendix
\section{Attualizzazione dei rischi}

\subsection{Rischi durante il periodo di Analisi}

\begin{tabularx}{\textwidth}{|X|X|}
\hline
\rowcolor{white}
\multicolumn{2}{|c|}{\textbf{RO1 - Calcolo delle tempistiche e dei costi}} \\
\hline
\textbf{Descrizione:}& A causa dell'inesperienza con un progetto di queste dimensioni, il gruppo ha inizialmente sottostimato il tempo necessario per il periodo di Analisi e Produzione del PoC\textsubscript{G} che precedono la revisione RTB.
Questo anche a causa dell'aver sottovalutato l'impatto che impegni esterni al progetto potessero avere sulla disponibilità di ore dei membri del gruppo. Tutto ciò a portato a dei ritardi rispetto alla pianificazione iniziale. \\
\hline
\textbf{Mitigazione:}& Il gruppo ha aggiornato la pianificazione delle attività e riassegnato i compiti in modo da ridurre il più possibile il ritardo rispetto alla data prevista in origine per la revisione RTB, tenendo costantemente informati il committente ed il proponente. \\
\hline
\rowcolor{white}
\caption{Mitigazione RO1}
\end{tabularx}

\subsection{Rischi durante il periodo di Produzione del PoC\textsubscript{G}}

\begin{tabularx}{\textwidth}{|X|X|}
\hline
\rowcolor{white}
\multicolumn{2}{|c|}{\textbf{RT1 - Inesperienza in ambito tecnologico}} \\
\hline
\textbf{Descrizione:}& Alcuni membri del gruppo non avevano esperienza con i linguaggi di programmazione scelti per la realizzazione del PoC\textsubscript{G}, per esempio Python\textsubscript{G}.  \\
\hline
\textbf{Mitigazione:}& I compiti per la realizzazione del PoC\textsubscript{G} sono stati distribuiti in modo da assicurare un supporto adeguato ai membri del gruppo con meno esperienza di sviluppo. In questo modo tutti hanno potuto contribuire, aumentando le proprie competenze tecnologiche. \\
\hline
\rowcolor{white}
\caption{Mitigazione RT1}
\end{tabularx}

\newpage

\begin{tabularx}{\textwidth}{|X|X|}
\hline
\rowcolor{white}
\multicolumn{2}{|c|}{\textbf{RP3 - Conflitti interni per lo sviluppo del progetto}} \\
\hline
\textbf{Descrizione:}& Durante le scelte delle tecnologie da utilizzare per la realizzazione del PoC\textsubscript{G} i membri del gruppo hanno avuto opinioni divergenti, per esempio se salvare i dati in un database\textsubscript{G} relazionale oppure in un file .json in modo da accorciare i tempi di sviluppo del PoC\textsubscript{G}. \\
\hline
\textbf{Mitigazione:}& I membri del gruppo si sono riuniti per discutere i pro e i contro delle opzioni proposte, e alla fine dell'incontro la decisione comune è stata di utilizzare un database\textsubscript{G} SQL\textsubscript{G} anche per il PoC\textsubscript{G}, visto che era una tecnologia da utilizzare anche per lo sviluppo del prodotto finale. \\
\hline
\rowcolor{white}
\caption{Mitigazione RP3}
\end{tabularx}

\subsection{Rischi durante il periodo di Progettazione architetturale}

\begin{tabularx}{\textwidth}{|X|X|}
\hline
\rowcolor{white}
\multicolumn{2}{|c|}{\textbf{RO1 -  Calcolo delle tempistiche e dei costi}} \\
\hline
\textbf{Descrizione:}& E' stato riscontrato un prolungamento dei tempi rispetto a quanto inizialmente previsto. Questo rallentamento è stato causato principalmente da fattori come l'inesperienza del team, idee poco chiare e la presenza di attività personali che hanno influenzato il tempo dedicato al progetto.\\
\hline
\textbf{Mitigazione:}& Il team ha organizzato incontri straordinari nel fine settimana con una durata prolungata, durante i quali tutti i membri del team sono tenuti a partecipare. Questi incontri aggiuntivi consentono di dedicare più tempo alla revisione e al raffinamento delle idee inizialmente proposte. Al fine di risolvere i dubbi e ottenere ulteriori indicazioni sulle aspettative del progetto, il team ha organizzato un incontro specifico con il proponente. \\
\hline
\rowcolor{white}
\caption{Mitigazione RO1}
\end{tabularx}

\begin{tabularx}{\textwidth}{|X|X|}
\hline
\rowcolor{white}
\multicolumn{2}{|c|}{\textbf{RO2 - Modifiche in corso d'opera}} \\
\hline
\textbf{Descrizione:}& Durante la progettazione si è notato la necessità di utilizzare un framework PHP, il quale può offrire notevoli vantaggi in termini di efficienza, scalabilità e manutenibilità del progetto. Tuttavia, va notato che, durante la fase del proof of concept, non è stato ancora effettuato un esame dettagliato o l'integrazione di un framework specifico.\\
\hline
\textbf{Mitigazione:}& I membri del gruppo si sono incontrati con il Professor Cardin per discutere della possibilità di adottare un framework nonostante la mancanza di valutazione nella fase di proof of concept. Durante l'incontro, è stata esposta la situazione e è stato richiesto il suo parere su quale framework sarebbe più indicato per le esigenze specifiche del progetto.\\
\hline
\rowcolor{white}
\caption{Mitigazione RO2}
\end{tabularx}

\subsection{Rischi durante il periodo di Progettazione di dettaglio e codifica}

\begin{tabularx}{\textwidth}{|X|X|}
\hline
\rowcolor{white}
\multicolumn{2}{|c|}{\textbf{RT1 - Inesperienza in ambito tecnologico}} \\
\hline
\textbf{Descrizione:}& I membri del gruppo non avevano esperienza con il framework\textsubscript{G} di programmazione Laravel. \\
\hline
\textbf{Mitigazione:}& Il team ha dedicato ore extra per seguire i video tutorial sul framework\textsubscript{G} Laravel. \\
\hline
\rowcolor{white}
\caption{Mitigazione RT1}
\end{tabularx}

