\section{Introduzione}

\subsection{Scopo del documento}
Nel seguente documento viene esposta in modo dettagliato la pianificazione delle attività da svolgere nel corso del progetto, trattando i seguenti punti:
\begin{itemize}
	\item Analisi dei rischi;
	\item Modello dello sviluppo adottato;
	\item Pianificazione delle fasi;
	\item Stima dei costi e delle risorse necessarie.
\end{itemize}

\subsection{Scopo del prodotto}
Gli attuali sistemi di rilevazione dei bot rispetto agli esseri umani prevedono l'utilizzo di un test CAPTCHA, progettato per cercare di bloccare azioni con fini malevoli nel web da parte di sistemi automatizzati. Nel capitolato “CAPTCHA: Umano o Sovrumano?” viene evidenziata una criticità presente in tali sistemi: grazie ai notevoli progressi nel campo dell’intelligenza artificiale si è nel tempo giunti al punto che i task i quali si ritenevano impossibili (o quantomeno, molto difficili) da svolgere per una macchina ora vengono effettuate dai bot talvolta persino meglio delle persone.
Dal proponente “Zucchetti S.p.A” viene richiesto lo sviluppo di un'applicazione web contenente una pagina di login con un sistema in grado di rilevare i bot rispetto agli esseri umani in maniera più efficace.

\subsection{Glossario}
Per risolvere ambiguità relative al linguaggio utilizzato nei documenti prodotti, è stato creato un documento denominato “Glossario”. Questo documento fornisce le definizioni relative a tutti i termini tecnici utilizzati nei vari documenti, segnalando questi termini con pedice G accanto alla parola.

\subsection{Riferimenti}

\subsubsection{Riferimenti normativi}

\begin{itemize}
	\item Norme di Progetto v1.0.0;
	\item Capitolato d'appalto C1 "CAPTCHA: Umano o Sovrumano?"
		\url{https://www.math.unipd.it/~tullio/IS-1/2022/Progetto/C1.pdf}
	\item Slide PD2 del corso di Ingegneria del Software - Regolamento del Progetto Didattico:
		\url{https://www.math.unipd.it/~tullio/IS-1/2022/Dispense/PD02.pdf}
\end{itemize}

\subsection{Riferimenti informativi}
\begin{itemize}
	\item Analisi dei Requisiti v2.0.0;
	\item Slide T04 del corso di Ingegneria del Software - Gestione di progetto:
		\url{https://www.math.unipd.it/~tullio/IS-1/2022/Dispense/T04.pdf}
	\item Slide T02 del corso di Ingegneria del Software - Processi di ciclo di vita del software:
		\url{https://www.math.unipd.it/~tullio/IS-1/2022/Dispense/T02.pdf}
\end{itemize}



