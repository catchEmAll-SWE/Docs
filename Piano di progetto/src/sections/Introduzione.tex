\section{Introduzione}

\subsection{Scopo del documento}
Nel seguente documento viene esposto in modo dettagliato la pianificazione delle attività da svolgere  per lo sviluppo del progetto, trattando i seguenti punti:
\begin{itemize}
	\item Analisi dei rischi;
	\item Modello dello sviluppo adottato;
	\item Suddivisione dei compiti tra i membri del gruppo;
	\item Stima dei costi e delle risorse necessarie;
\end{itemize}

\subsection{Scopo del prodotto}
Gli attuali sistema di sicurezza di rilevazione dei bot rispetto agli esseri umani, al fine di bloccare azioni con fini malevoli nel web, oggi giorno viene effettuato tramite test CAPTCHA. Viene evidenziato, nel capitolato “CAPTCHA: Umano o Sovrumano?” una criticità in questi sistemi, in quanto grazie ai notevoli progressi nel campo dell’intelligenza artificiale, task che si ritenevano impossibili da svolgere per una macchina, ora vengono effettuate spesso meglio delle persone, superando le barriere poste attualmente nei motori di ricerca.
Dal proponente “Zucchetti S.p.A” viene richiesto lo sviluppo di un applicazione web contenente una pagina di login con un sistema in grado di rilevare in modo ottimale ed efficiente i bot rispetto agli esseri umani.

\subsection{Glossario}
Per risolvere ambiguità relative al linguaggio utilizzato nei documenti prodotti, è stato creato un documento denominato “Glossario”. Questo documento fornisce le definizioni relative a tutti i termini tecnici utilizzati nei vari documenti, segnalando questi termini con l’apice G accanto alla parola.

\subsection{Riferimenti}

\subsubsection{Riferimenti normativi}

\begin{itemize}
	\item Norme di Progetto v1.0.0; (a caso)
	\item Capitolato d'appalto C1 "CAPTCHA: Umano o Sovrumano?"
		\url{https://www.math.unipd.it/~tullio/IS-1/2022/Progetto/C1.pdf}
	\item Slide PD2 del corso di Ingegneria del Software - Regolamento del Progetto Didattico:
		\url{https://www.math.unipd.it/~tullio/IS-1/2022/Dispense/PD02.pdf}
\end{itemize}

\subsection{Riferimenti informativi}
\begin{itemize}
	\item Analisi dei Requisiti v2.0.0; (a caso)
	\item Slide T04 del corso di Ingegneria del Software - Gestione di progetto:
		\url{https://www.math.unipd.it/~tullio/IS-1/2022/Dispense/T04.pdf}
	\item Slide T02 del corso di Ingegneira del Software - Processi di ciclo di vita del software:
		\url{https://www.math.unipd.it/~tullio/IS-1/2022/Dispense/T02.pdf}
\end{itemize}



