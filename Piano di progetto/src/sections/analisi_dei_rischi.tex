\section{Analisi dei rischi}

Grazie ad un attenta analisi dei rischi il gruppo si pone l'obiettivo di prevedere e mitigare rischi e problematiche che possono nascere nel corso delle varie attività del progetto, cercando le possibili strategie per minimizzarli. 
La gestione dei rischi avviene tramite le 4 attività seguenti:
\begin{itemize}
	\item \textbf{Identificazione} dei possibili eventi che possono causare problemi durante l'avanzamento delle attività;
	\item \textbf{Analisi} di tali eventi tramite una stima delle probabilità di occorrenza e delle possibili conseguenze;
	\item \textbf{Pianificazione} della metodologia per impedire il verificarsi dei rischi individuati e dei comportamenti da adottare nel caso in cui si dovessero presentare;
	\item \textbf{Monitoraggio} costante durante le attività del progetto, in modo da procedere con l'attuazione delle procedure di mitigazione quando necessario e raffinare le strategie adottate in base ai risultati sperimentati.
\end{itemize}

I rischi sono stati suddivisi in tre categorie:
\begin{itemize}
	\item Rischi personali;
	\item Rischi tecnologici;
	\item Rischi organizzativi.
\end{itemize}

\newpage
\subsection{Rischi personali}

\renewcommand\tabularxcolumn[1]{>{\Centering}m{#1}}
\begin{tabularx}{\textwidth}{ |X|X|}
\hline
\multicolumn{2}{|c|}{\textbf{Difficoltà nella comunicazione interna}} \\
\hline
\textbf{Descrizione:}& La comunicazione scritta tra i membri del gruppo non è sempre efficace e può essere causa di incomprensioni e difficoltà nella collaborazione. \\
\hline
\textbf{Identificazione:}& Ogni membro del gruppo ha impegni fissi e 
 che possono ostacolarne la partecipazione alle riunioni stabilite, dove 
tali incomprensioni vengono chiarite. \\
\hline
\textbf{Precauzioni:}& Ogni membro del gruppo che deve avviare una discussione con una o più persone proporrà diverse date per concordare un meeting, tenendo conto delle disponibilità dei partecipanti necessari. \\
\hline
\textbf{Pericolosità:}& Alta.\\
\hline
\textbf{Stima di manifestazione:}& Media.\\
\hline
\textbf{Conseguenze:}& Possibili ritardi nell'avanzamento del progetto.\\
\hline
\textbf{Piano di contingenza:}& In caso di impossibilità di organizzare agevolmente un meeting, la discussione dovrà avvenire necessariamente in maniera asincrona tramite messaggi scritti, e in tal caso ognuno si impegnerà di esprimere i concetti in maniera semplice e priva di ambiguità. Vengono messi a disposizione diversi strumenti per la comunicazione, tra cui l'app di messaggistica WhatsApp, la piattaforma Discord e la comunicazione tramite email. E' richiesto a ciascun membro del gruppo di controllare periodicamente questi strumenti. \\
\hline
\caption{Difficoltà nella comunicazione interna}
\end{tabularx}

\begin{tabularx}{\textwidth}{|X|X|}
\hline
\multicolumn{2}{|c|}{\textbf{Difficoltà nella comunicazione esterna}} \\
\hline
\textbf{Descrizione:}& La comunicazione scritta tra il gruppo e il proponente può essere causa di incomprensioni. \\
\hline
\textbf{Identificazione:}& Può essere impossibile organizzare un meeting in breve tempo tra gruppo e proponente. \\
\hline
\textbf{Precauzioni:}& Quando il gruppo dovrà avviare una discussione con il proponente proporrà con anticipo diverse date per concordare un meeting, tenendo conto sia delle disponibilità interne che delle disponibilità del proponente.\\
\hline
\textbf{Pericolosità:}& Media.\\
\hline
\textbf{Stima di manifestazione:}& Media.\\
\hline
\textbf{Conseguenze:}& Possibili ritardi nell'avanzamento del progetto.\\
\hline
\textbf{Piano di contingenza:}& In caso di impossibilità di organizzare agevolmente un meeting tra gruppo e proponente, la discussione dovrà avvenire necessariamente in maniera asincrona tramite email. In tal caso il gruppo si impegnerà ad esprimere i concetti in maniera semplice e priva di ambiguità, avendo anche cura di aggiornare il proponente sullo stato di avanzamento del progetto. \\
\hline
\caption{Difficoltà nella comunicazione esterna}
\end{tabularx}

\begin{tabularx}{\textwidth}{|X|X|}
\hline
\multicolumn{2}{|c|}{\textbf{Conflitti interni per lo sviluppo del progetto}} \\
\hline
\textbf{Descrizione:}& Data la libertà di scelta per gli strumenti e le tecnologie da utilizzare durante il progetto è possibile che i diversi punti di vista di alcuni membri del team si scontrino.\\
\hline
\textbf{Identificazione:}& Il gruppo si trova in difficoltà nel prendere una decisione riguardante il progetto.\\
\hline
\textbf{Precauzioni:}& Tutte le decisioni che regolano lo svolgimento delle attività, e quindi impattano tutti i membri del gruppo, non possono essere prese senza l'approvazione comune.\\
\hline
\textbf{Pericolosità:}& Alta.\\
\hline
\textbf{Stima di manifestazione:}& Alta.\\
\hline
\textbf{Conseguenze:}& Il capitolato viene svolto in un clima avverso.\\
\hline
\textbf{Piano di contingenza:}& Chi dovesse non essere d'accordo con una certa decisione presa dal gruppo può richiederne la rivalutazione, fornendo anche una documentazione di supporto alle sue idee. Il gruppo, tutto riunito, ascolterà le proposte alternative e deciderà come procedere.\\
\hline
\caption{Conflitti interni per lo sviluppo del progetto}
\end{tabularx}

\subsection{Rischi tecnologici}

\renewcommand\tabularxcolumn[1]{>{\Centering}m{#1}}
\begin{tabularx}{\textwidth}{|X|X|}
\hline
\multicolumn{2}{|c|}{\textbf{Inesperienza in ambito tecnologico}} \\
\hline
\textbf{Descrizione:}& Nessun membro del team ha un'elevata esperienza con le tecnologie scelte per lo sviluppo del progetto.\\
\hline
\textbf{Identificazione:}& Chi è in difficoltà comunica al resto del team i problemi riscontrati.\\
\hline
\textbf{Precauzioni:}& Studio approfondito delle tecnologie da utilizzare tramite manuali e tutorial online.\\
\hline
\textbf{Pericolosità:}& Alta.\\
\hline
\textbf{Stima di manifestazione:}& Media.\\
\hline
\textbf{Conseguenze:}& Ritardi o inadempienze nello svolgere i lavori assegnati.\\
\hline
\textbf{Piano di contingenza:}& Chi ha riscontrato un problema durante lo svolgimento di un'attività dovrà consultare la documentazione ufficiale e/o i tutorial online. In caso di necessità potrà richiedere ai membri del gruppo con più esperienza di ragionare insieme ai problemi riscontrati per trovare una soluzione.\\
\hline
\caption{Inesperienza in ambito tecnologico}
\end{tabularx}

\vspace{20pt}

\begin{tabularx}{\textwidth}{|X|X|}
\hline
\multicolumn{2}{|c|}{\textbf{Implementazione in diversi browser}} \\
\hline
\textbf{Descrizione:}& Per visualizzare una pagina web è possibile utilizzare diversi browser, ognuno con le proprie caratteristiche.\\
\hline
\textbf{Identificazione:}& Il prodotto finale presenta delle anomalie in specifiche versioni di un browser. \\
\hline
\textbf{Precauzioni:}& Scelta di un sottoinsieme di browser e relative versioni per i quali garantire la compatibilità del prodotto. \\
\hline
\textbf{Pericolosità:}& Media.\\
\hline
\textbf{Stima di manifestazione:}& Media.\\
\hline
\textbf{Conseguenze:}& Presenza di bug nel prodotto finale .\\
\hline
\textbf{Piano di contingenza:}& Nel caso in cui le precauzioni non dovessero essere sufficienti sarà necessario organizzare delle attività di correzione dei bug individuati. \\
\hline
\caption{Implementazione in diversi browser}
\end{tabularx}

\vspace{20pt}

\begin{tabularx}{\textwidth}{|X|X|}
\hline
\multicolumn{2}{|c|}{\textbf{Problemi hardware}} \\
\hline
\textbf{Descrizione:}& Ciascun membro del gruppo lavora su un computer in remoto il quale può essere soggetto a guasti e mancanza di connessione internet.\\
\hline
\textbf{Identificazione:}& Chi si trova in difficoltà comunica al resto del team il problema riscontrato.\\
\hline
\textbf{Precauzioni:}& Tutti i file riguardanti il progetto devono dovranno essere caricati su GitHub\textsubscript{G} in modo da evitare la perdita di dati.\\
\hline
\textbf{Pericolosità:}& Media.\\
\hline
\textbf{Stima di manifestazione:}& Bassa.\\
\hline
\textbf{Conseguenze:}& Ritardi nell'avanzamento del singolo individuo nel progetto.\\
\hline
\textbf{Piano di contingenza:}& Utilizzare un altro dispositivo disponibile oppure rivolgersi all'ateneo per richiedere l'utilizzo di un computer in un laboratorio.\\
\hline
\caption{Problemi hardware}
\end{tabularx}

\begin{tabularx}{\textwidth}{|X|X|}
\hline
\multicolumn{2}{|c|}{\textbf{Problemi software}} \\
\hline
\textbf{Descrizione:}& Per svolgere qualsiasi attività inerente al progetto il team utilizza software di terze parti, che possono contenere bug ed essere soggetti a momenti di inutilizzabilità.\\
\hline
\textbf{Identificazione:}& Chi identifica problemi negli strumenti utilizzati comunica quanto riscontrato al resto del gruppo.\\
\hline
\textbf{Precauzioni:}& I software di terze parti da utilizzare nel progetto vengono scelti in base alla loro affidabilità. Tutti i file riguardanti il progetto dovranno essere caricati su GitHub\textsubscript{G} in modo da evitare la perdita di dati.\\
\hline
\textbf{Pericolosità:}& Media.\\
\hline
\textbf{Stima di manifestazione:}& Bassa.\\
\hline
\textbf{Conseguenze:}& Perdite di dati e indisponibilità nello svolgere le attività previste.\\
\hline
\textbf{Piano di contingenza:}& In caso di problematiche gravi e durature, il responsabile del gruppo durante lo sprint\textsubscript{G} in questione dovrà ricercare un software alternativo a quello non più utilizzabile.\\
\hline
\caption{Problemi software}
\end{tabularx}

\subsection{Rischi organizzativi}

\renewcommand\tabularxcolumn[1]{>{\Centering}m{#1}}
\begin{tabularx}{\textwidth}{|X|X|}
\hline
\multicolumn{2}{|c|}{\textbf{Calcolo delle tempistiche e dei costi}} \\
\hline
\textbf{Descrizione:}& A causa dell'inesperienza di ciascun membro del gruppo nello svolgere progetti a livello professionale, è difficile stabilire le milestone\textsubscript{G} concrete e raggiungibili nei tempi prefissati. \\
\hline
\textbf{Identificazione:}& Le attività non vengono portate a termine nel tempo previsto. \\
\hline
\textbf{Precauzioni:}& I compiti da portare a termine per ciascuno sprint\textsubscript{G} vengono pensati per essere svolti in un tempo breve, in modo da poter stabilire le tempistiche con una buona precisione. \\
\hline
\textbf{Pericolosità:}& Alta.\\
\hline
\textbf{Stima di manifestazione:}& Media.\\
\hline
\textbf{Conseguenze:}& Nel caso di sottostima del tempo necessarie da impiegare per un'attività non verrebbe rispettata la scadenza imposta, portando ritardi alla conclusione del progetto e necessità di ulteriori ore a quelle preventivate; una sovrastima invece può portare a notevoli discrepanze tra preventivo e consuntivo. \\
\hline
\textbf{Piano di contingenza:}& In caso di sottostima del tempo necessario il responsabile avrà il compito di riassegnare le risorse nella maniera più efficace possibile in modo da ridurre al minimo i ritardi. In caso di sovrastima il gruppo potrà dedicarsi allo sviluppo dei vari requisiti\textsubscript{G} opzionali proposti nel capitolato.\\
\hline
\caption{Calcolo delle tempistiche e dei costi}
\end{tabularx}

\vspace{20pt}

\begin{tabularx}{\textwidth}{|X|X|}
\hline
\multicolumn{2}{|c|}{\textbf{Modifiche in corso d'opera}} \\
\hline
\textbf{Descrizione:}& Durante lo sviluppo del progetto potrebbero nascere delle necessità da parte del gruppo o del proponente di cambiare dei requisiti\textsubscript{G}.\\
\hline
\textbf{Identificazione:}& I requisiti\textsubscript{G} stabiliti diventano obsoleti oppure insufficienti. \\
\hline
\textbf{Precauzioni:}& Il gruppo, durante i primi meeting con il proponente, si pone l'obiettivo di definire in maniera più dettagliata possibile i bisogni che deve soddisfare il prodotto finale.\\
\hline
\textbf{Pericolosità:}& Alta.\\
\hline
\textbf{Stima di manifestazione:}& Bassa.\\
\hline
\textbf{Conseguenze:}& Non è garantito che sia possibile rispettare le milestone\textsubscript{G} prefissate.\\
\hline
\textbf{Piano di contingenza:}& Il gruppo dovrà ripianificare i compiti nella maniera più efficace possibile in modo da ridurre al minimo i ritardi. \\
\hline
\caption{Modifiche in corso d'opera}
\end{tabularx}
