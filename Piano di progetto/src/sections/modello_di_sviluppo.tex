\section{Modello di sviluppo}
Il gruppo decide di utilizzare il modello AGILE con framework Scrum.

\subsection{Modello AGILE}
Il modello AGILE con framework Scrum prevede di dividere il progetto in blocchi rapidi di lavoro (Sprint) 
alla fine di ciascuno dei quali creare un incremento per lo sviluppo del prodotto. 
Esso indica come definire i dettagli del lavoro da fare nell'immediato futuro e prevede 
vari meeting con caratteristiche precise per creare occasioni di ispezione e controllo del lavoro svolto.

I cicli di Scrum avranno durata settimanale. All'inizio di ogni ciclo vi sarà una breve riunione nella quale si discuterà:
\begin{itemize}
	\item Resoconto e status dei lavori della settimana precedente;
	\item Problemi riscontrati durante il lavoro;
	\item Pianificazione e assegnazione delle attività (task) da svolgere nella nuova settimana ed eventuali riunioni intermedi usando Product Backlog Refinement.
\end{itemize}
Ogni ciclo di Scrum, detto anche Sprint, inizierà il lunedi della settimana.\\
Si è deciso di dividere il progetto in varie fasi, divise appunto in Sprint per gestire meglio i vari processi che comporranno gli vari stadi del progetto. Ogni fase avrà uno scopo preciso e sarà considerata un punto importante da raggiungere per il corretto andamento del progetto.