\section{Modello di sviluppo}
Il gruppo ha scelto di utilizzare il modello \textbf{AGILE}\textsubscript{G} con framework scrum\textsubscript{G}.

\subsection{Modello AGILE\textsubscript{G}}
Il modello AGILE\textsubscript{G} con framework scrum\textsubscript{G} prevede di dividere il progetto in blocchi rapidi di lavoro (Sprint\textsubscript{G}),
alla fine di ciascuno dei quali viene realizzato un incremento nello sviluppo del prodotto. 
Esso indica come definire i dettagli del lavoro da fare nell'immediato futuro e prevede 
vari meeting con caratteristiche precise per creare occasioni di ispezione e controllo del lavoro svolto.

I cicli di scrum\textsubscript{G}, detti anche sprint\textsubscript{G}, avranno durata che variano da una a quattro settimane. All'inizio di ogni ciclo vi sarà una riunione nella quale si discuterà:
\begin{itemize}
	\item Resoconto e status dei lavori del ciclo precedente;
	\item Riepilogare i problemi riscontrati durante il lavoro;
	\item Definire chiaramente la milestone\textsubscript{G} dello sprint\textsubscript{G};
	\item Pianificazione e assegnazione delle attività (task\textsubscript{G}) da svolgere nel nuovo ciclo e le date delle riunioni intermedie usando Product Backlog Refinement\textsubscript{G}.
\end{itemize}
Le riunioni intermedie si terranno ogni settimana e hanno funzione di:
\begin{itemize}
	\item Comunicare ai membri del gruppo lo stato di avanzamento delle attività dello sprint\textsubscript{G};
	\item Assegnazione delle attività di revisione per le task\textsubscript{G} completate.
\end{itemize}
Verranno inoltre mantenuti costantemente monitorati gli avanzamenti delle attività attraverso JIRA\textsubscript{G} e discussioni giornaliere svolte utilizzando gli strumenti di comunicazione scelti dal gruppo.\\
Ogni Sprint\textsubscript{G} inizierà il lunedì della settimana.\\
Si è deciso di dividere il progetto in vari periodi, divisi in sprint\textsubscript{G} per gestire meglio i vari processi che comporranno gli vari stadi del progetto. Ogni sprint\textsubscript{G} avrà uno scopo preciso e sarà considerato un punto importante da raggiungere per il corretto andamento del progetto.\\
Il gruppo inizialmente aveva optato per utilizzare sprint\textsubscript{G} di durata settimanale per mantenere un controllo stretto sull'andamento del progetto. Ci si è accorti però che spesso uno sprint\textsubscript{G} non era fine a se stesso e alcune attività si andavano a prolungare anche nello sprint\textsubscript{G} successivo. Il gruppo ha quindi deciso di allungare la durata degli sprint\textsubscript{G}  definendone chiaramente per ognuno l'obiettivo, che sarà anche una milestone\textsubscript{G} per il progetto. In questo modo gli sprint\textsubscript{G} indicano molto più chiaramente l'andamento del progetto. È stato comunque mantenuto l'incontro settimanale necessario per monitorare l'andamento delle task\textsubscript{G} da svolgere.
\newpage
\subsection{Sprint\textsubscript{G} individuati}
Di seguito viene riportata una tabella contenenti gli sprint\textsubscript{G} svolti e da svolgere, definiti secondo gli obiettivi che il gruppo ha individuato per uno sviluppo ottimale del progetto.
\begin{center}
	\renewcommand\tabularxcolumn[1]{>{\Centering}m{#1}}
	\setlength\extrarowheight{5pt}
	\rowcolors{2}{gray!10}{gray!40}
	\begin{tabularx}{\textwidth}{| c | X |} 
		\hline
		\rowcolor{white}
		\textbf{Numero} & \textbf{Obiettivo}\\
		I &  Analisi preliminare e creazione di una base del way of working\textsubscript{G} \\
		II & Definizione dei requisiti\textsubscript{G} e dei casi d'uso\textsubscript{G} necessari e impostazione della pianificazione dei vari periodi che costituiranno il progetto \\
		III & Verifica\textsubscript{G} dei documenti e miglioramento del way of working\textsubscript{G} \\
		IV & Scelta degli strumenti e tecnologie da utilizzare per lo sviluppo del PoC\textsubscript{G} \\
		V & Sviluppo del PoC\textsubscript{G} e miglioramento dei documenti \\
		VI & Verifica\textsubscript{G} e approvazione dei documenti necessari alla revisione \textit{RTB} e collaudo del Poc\textsubscript{G}\\
		VII & Conclusione della progettazione architetturale ad alto livello\\
		VIII & Conclusione della progettazione di dettaglio e definiti i test di unità \\
		IX & Codifica e verifica\textsubscript{G} dei requisiti\textsubscript{G} obbligatori \\
		X & Codifica e verifica\textsubscript{G} dei requisiti\textsubscript{G} opzionali \\
		XI & Validazione\textsubscript{G} e collaudo del MVP\textsubscript{G} e dei documenti necessari alla revisione \textit{PB}\\
		\hline
	\end{tabularx}
\end{center}