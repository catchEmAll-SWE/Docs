\section{Modello di sviluppo}
Il gruppo ha scelto di utilizzare il modello \textbf{AGILE}\textsubscript{G} con framework scrum\textsubscript{G}.

\subsection{Modello AGILE\textsubscript{G}}
Il modello AGILE\textsubscript{G} con framework scrum\textsubscript{G} prevede di dividere il progetto in blocchi rapidi di lavoro (Sprint\textsubscript{G}),
alla fine di ciascuno dei quali viene realizzato un incremento nello sviluppo del prodotto. 
Esso indica come definire i dettagli del lavoro da fare nell'immediato futuro e prevede 
vari meeting con caratteristiche precise per creare occasioni di ispezione e controllo del lavoro svolto.

I cicli di scrum\textsubscript{G} avranno durata settimanale. All'inizio di ogni ciclo vi sarà una breve riunione nella quale si discuterà:
\begin{itemize}
	\item Resoconto e status dei lavori della settimana precedente;
	\item Problemi riscontrati durante il lavoro;
	\item Pianificazione e assegnazione delle attività (task) da svolgere nella nuova settimana ed eventuali riunioni intermedi usando Product Backlog Refinement\textsubscript{G}.
\end{itemize}
Ogni ciclo di scrum\textsubscript{G}, detto anche Sprint\textsubscript{G}, inizierà il lunedì della settimana.\\
Si è deciso di dividere il progetto in varie fasi, divise appunto in Sprint\textsubscript{G} per gestire meglio i vari processi che comporranno gli vari stadi del progetto. Ogni fase avrà uno scopo preciso e sarà considerata un punto importante da raggiungere per il corretto andamento del progetto.
