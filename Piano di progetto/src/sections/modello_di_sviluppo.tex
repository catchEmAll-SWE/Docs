\section{Modello di sviluppo}
Il gruppo ha scelto di utilizzare il modello \textbf{AGILE}\textsubscript{G} con framework scrum\textsubscript{G}.

\subsection{Modello AGILE\textsubscript{G}}
Il modello AGILE\textsubscript{G} con framework scrum\textsubscript{G} prevede di dividere il progetto in blocchi rapidi di lavoro (Sprint\textsubscript{G}),
alla fine di ciascuno dei quali viene realizzato un incremento nello sviluppo del prodotto. 
Esso indica come definire i dettagli del lavoro da fare nell'immediato futuro e prevede 
vari meeting con caratteristiche precise per creare occasioni di ispezione e controllo del lavoro svolto.

I cicli di scrum\textsubscript{G}, detti anche Sprint\textsubscript{G}, avranno durata che variano da due a tre settimane. All'inizio di ogni ciclo vi sarà una riunione nella quale si discuterà:
\begin{itemize}
	\item Resoconto e status dei lavori del ciclo precedente;
	\item Riepologare i problemi riscontrati durante il lavoro;
	\item Definire il milestone\textsubscript{G} dello sprint\textsubscript{G};
	\item Pianificazione e assegnazione delle attività (task\textsubscript{G}) da svolgere nel nuovo ciclo e le date dei riunioni intermedi usando Product Backlog Refinement\textsubscript{G}.
\end{itemize}
I riunioni intermedi si terranno ogni settimana e hanno funzione di:
\begin{itemize}
	\item Comunicare ai membri del gruppo lo stato di avanzamento delle attività;
	\item Assegnazione delle attività di revisione dei documenti.
\end{itemize}
Ogni Sprint\textsubscript{G} inizierà il lunedì della settimana.\\
Si è deciso di dividere il progetto in vari periodi, divise appunto in Sprint\textsubscript{G} per gestire meglio i vari processi che comporranno gli vari stadi del progetto. Ogni periodo avrà uno scopo preciso e sarà considerata un punto importante da raggiungere per il corretto andamento del progetto.

\subsection{Sprint\textsubscript{G} individuati}
Di seguito viene riportata una tabella contenenti gli Sprint\textsubscript{G} individuati e i loro obiettivi.

\setlength\extrarowheight{5pt}
	\rowcolors{2}{gray!10}{gray!40}
	\begin{tabularx}{\textwidth}{|ccccccc|c|}
		\hline
		\rowcolor{white}
		\textbf{Numero} & \textbf{Obiettivi}\\
		\hline
		I &  Analisi preliminare e creazione di una base del way of working\textsubscript{G}. \\
		II & Definizione dei requisiti e dei casi d'uso necessari e impostazione della pianificazione del piano di progetto. \\
		III & Verifica\textsubscript{G} dei documenti e miglioramento del way of working\textsubscript{G}. \\
		IV & Scelta degli strumenti e tecnologie da utilizzare per lo sviluppo del PoC\textsubscript{G}. \\
		V & Sviluppo del PoC\textsubscript{G} e miglioramento dei documenti. \\
		VI & Validazione e collaudo dei documenti necessari alla revisione \textit{RTB} e del Poc\textsubscript{G}.\\
		VII & Conclusione della progettazione architetturale ad alto livello. \\
		VIII & Conclusione della progettazione di dettaglio e definiti i test di unità. \\
		IX & Codifica e verifica dei requisiti obbligatori. \\
		X & Codifica e verifica dei requisiti opzionali. \\
		XI & Validazione e collaudo del MVP\textsubscript{G} e dei documenti necessari alla revisione \textit{PB}.\\
		\hline
		\rowcolor{white}
		\caption{ Sprint individuati}
	\end{tabularx}