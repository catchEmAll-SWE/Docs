\section{Introduzione}

\subsection{Scopo del documento}
Nel seguente documento viene esposta in modo dettagliato la pianificazione delle attività da svolgere nel corso del progetto, trattando i seguenti punti:
\begin{itemize}
	\item Analisi dei rischi;
	\item Modello dello sviluppo adottato;
	\item Pianificazione dei periodi;
	\item Preventivo dei costi e delle ore necessarie;
	\item Consuntivo;
	\item Attualizzazione dei rischi.
\end{itemize}

\subsection{Scopo del prodotto}
Dal proponente Zucchetti S.p.A. viene evidenziato, nel capitolato da loro proposto, una criticità negli attuali sistemi di sicurezza sulla rilevazione dei bot\textsubscript{G} rispetto agli esseri umani. Oggi giorno il meccanismo più utilizzato per risolvere questo problema è il test CAPTCHA\textsubscript{G}.\\
Un bot\textsubscript{G} non è altro che una procedura automatizzata che, in questo caso, ha fini malevoli, come per esempio:
\begin{itemize}
	\item Registrazione presso siti web;
	\item Creazione di spam\textsubscript{G};
	\item Violare sistemi di sicurezza.
\end{itemize}
I bot\textsubscript{G}, grazie alle nuove tecnologie sviluppate con sistemi che utilizzano principalmente l'intelligenza artificiale, riescono a svolgere compiti che fino a poco tempo fa venivano considerati impossibili da svolgere per una macchina.\\
Ciò evidenzia che i CAPTCHA\textsubscript{G} attuali risultano sempre più obsoleti, non andando a individuare correttamente tutti i bot\textsubscript{G}, se non quasi nessuno.\\
Un'altra criticità individuata dal proponente è il sistema di classificazione delle immagini che sta effettuando Google grazie al proprio reCAPTCHA\textsubscript{G}, che attualmente è il sistema più diffuso.\\
Questa criticità nasce dal beneficio che questa big tech\textsubscript{G} ottiene dall'interazione degli utenti nel risolvere le task\textsubscript{G} proposte, che portano alla creazione di enormi dataset\textsubscript{G} di immagini classificate che possono essere utilizzate per l'apprendimento dei propri sistemi di machine learning o vendibili a terzi.\\
Il capitolato C1 richiede di sviluppare una applicazione web costituita da una pagina di login provvista di questo sistema di rilevazione in grado di distinguere un utente umano da un bot\textsubscript{G}.\\
L'utente quindi, dopo aver compilato il form in cui inserirà il nome utente e la password, dovrà svolgere una task\textsubscript{G} che sarà il cosiddetto test CAPTCHA\textsubscript{G}.



\subsection{Glossario}
Per evitare ambiguità relative al linguaggio utilizzato nei documenti prodotti, viene fornito il \textbf{Glossario v 2.0.0}. In questo documento sono contenuti tutti i termini tecnici, i quali avranno una definizione specifica per comprenderne al meglio il loro significato.\\
Tutti i termini inclusi nel Glossario, vengono segnalati all'interno del documento Piano di progetto con una G a pedice.

\subsection{Riferimenti}

\subsubsection{Riferimenti normativi}\:
\begin{itemize}
	\item \textit{Norme di Progetto v1.0.0};
	\item Capitolato d'appalto C1 \textit{CAPTCHA: Umano o Sovrumano?}\\
		\url{https://www.math.unipd.it/~tullio/IS-1/2022/Progetto/C1.pdf};
	\item Slide PD2 del corso di Ingegneria del Software - Regolamento del Progetto Didattico:\\
		\url{https://www.math.unipd.it/~tullio/IS-1/2022/Dispense/PD02.pdf}.
\end{itemize}

\subsubsection{Riferimenti informativi}\:
\begin{itemize}
	\item \textit{Analisi dei Requisiti v1.0.0};
	\item Slide T04 del corso di Ingegneria del Software - Gestione di progetto:\\
		\url{https://www.math.unipd.it/~tullio/IS-1/2022/Dispense/T04.pdf};
	\item Slide T02 del corso di Ingegneria del Software - Processi di ciclo di vita del software:\\
		\url{https://www.math.unipd.it/~tullio/IS-1/2022/Dispense/T02.pdf}.
\end{itemize}



