\section{Obiettivi e metriche di qualità}
\subsection{Obiettivi e metriche di qualità di processo\textsubscript{G}}
In questa sezione viene illustrato come il gruppo vuole verificare e misurare i progressi dei processi primari e di supporto nel corso del progetto. 
\subsubsection{Obiettivi di qualità di processo\textsubscript{G}}
\subsubsubsection{Gestione processi}
\begin{center}
	\setlength\extrarowheight{5pt}
	\rowcolors{2}{gray!10}{gray!40}
	\begin{tabularx}{\textwidth}{|c|X|X|X|}
	\hline
	\rowcolor{white}
	\textbf{ID} & \textbf{Nome} & \textbf{Descrizione} & \textbf{Metriche associate}\\
	\hline
	OQPC01 & Miglioramento continuo & Il processo si deve poter valutare e migliorare continuamente & MQPC01 - SPICE\\
	\hline
	\rowcolor{white}
	\caption{Obiettivi di qualità di gestione di processo}
	\end{tabularx}
\end{center}
\subsubsubsection{Pianificazione}
\begin{center}
	\setlength\extrarowheight{5pt}
	\rowcolors{2}{gray!10}{gray!40}
	\begin{tabularx}{\textwidth}{|c|X|X|X|}
		\hline
		\rowcolor{white}
		\textbf{ID} & \textbf{Nome} & \textbf{Descrizione} & \textbf{Metriche associate}\\
		\hline
		OQPC02 & Efficienza nell'utilizzo delle risorse & Le risorse disponibili durante la durata del progetto devono essere distribuite ed utilizzate al meglio & MQPC02 - Costo pianificato di progetto; MQPC03 - Costo reale di progetto svolto;\\
		\hline
		OQPC03 & Variazioni dalla pianificazione & Assicurare che le scadenze e i limiti di costi illustrati nel documento \textit{Piano di progetto} siano rispettati &  MPC04: Variazioni nella programmazione;\hspace{65pt} MPC05: Variazioni nei costi. \\
		\hline
		\rowcolor{white}
		\caption{Obiettivi di qualità di processo\textsubscript{G} di pianificazione}
	\end{tabularx}
\end{center}
\newpage
\subsubsubsection{Documentazione}
\begin{center}
	\setlength\extrarowheight{5pt}
	\rowcolors{2}{gray!10}{gray!40}
	\begin{tabularx}{\textwidth}{|c|X|X|X|}
		\hline
		\rowcolor{white}
		\textbf{ID} & \textbf{Nome} & \textbf{Descrizione} & \textbf{Metriche associate}\\
		\hline
		OQPC04 & Leggibilità dei documenti & I documenti devono essere comprensibile all'utente medio & MQPC06 - Indice di Gulpease\\
		\hline
		OQPC05 & Correttezza ortografica & I documenti devono essere scritti senza errori ortografici & MQPC07 - Correttezza documento \\
		\hline
		\rowcolor{white}
		\caption{Obiettivi di qualità del processo di documentazione}
	\end{tabularx}
\end{center}


\subsubsection{Metriche di qualità di processo\textsubscript{G}}
\begin{center}
	\setlength\extrarowheight{5pt}
	\rowcolors{2}{gray!10}{gray!40}
	\begin{tabularx}{\textwidth}{|c|X|X|X|X|}
		\hline
		\rowcolor{white}
		\textbf{ID} & \textbf{Nome} & \textbf{Obiettivo} & \textbf{Valore accettabile} & \textbf{Valore ottimo}\\
		\hline
		MQPC01 & SPICE & OQPC01 - Miglioramento continuo & Level of Capability\textsubscript{G} $\geq$ 2 (Managed process) & Level of Capability\textsubscript{G} $\geq$ 4 (Predictable process) \\
		\hline
		MQPC02 & Costo pianificato di progetto & OQPC02 - Efficienza nell'utilizzo delle risorse & $\geq$ 0 \& $\le$ 11.100 & $\geq$ 0 \& $\le$ 11.100\\
		\hline
		MQPC03 & Costo reale di progetto svolto & OQPC02 - Efficienza nell'utilizzo delle risorse & BCWS $ \pm $ 15\% & BCWS \\
		\hline
		MQPC04 & Variazioni nella pianificazione & OQPC03 - Rispetto della pianificazione & $ \pm $ 15\% & 0\% \\
		\hline
		MQPC05 & Variazioni nei costi & OQPC03 - Rispetto della pianificazione & $ \pm $ 15\% & 0\% \\
		\hline
		MQPC06 & Indice di Gulpease & OQPC04 - Leggibilità dei documenti & $\geq$ 40 & $\geq$ 80 \\
		\hline
		MQPC07 & Numero errori ortografici & OQPC05 - Correttezza ortografica & 0 & 0 \\
		\hline
		\rowcolor{white}
		\caption{Metriche di qualità di processo\textsubscript{G}.}
	\end{tabularx}
\end{center}
\newpage
\subsection{Obiettivi e metriche di qualità di prodotto\textsubscript{G}}
Riferendoci alla serie di standard ISO/IEC\textsubscript{G} 25000 SQuaRE possiamo osservare un insieme di caratteristiche che il prodotto deve avere per essere considerato di qualità. Queste caratteristiche saranno misurabili tramite metriche apposite, le quali forniranno i valori accettabili per il raggiungimento dell'obiettivo.
\subsubsection{Obiettivi di qualità di prodotto\textsubscript{G}}

\subsubsubsection{Software}
\begin{center}
	\setlength\extrarowheight{5pt}
	\rowcolors{2}{gray!10}{gray!40}
	\begin{xltabular}{\textwidth}{|c|c|X|X|}
		\hline
		\rowcolor{white}
		\textbf{ID} & \textbf{Nome} & \textbf{Descrizione} & \textbf{Metriche associate}\\
		\hline
		OQPD03 & Appropriatezza funzionale & Si vogliono soddisfare in modo completo i requisiti\textsubscript{G} presenti nel documento \textit{Analisi dei requisiti} & MQPD03 - Copertura funzionale \\
		\hline
		OQPD04 & Efficienza & Si vuole realizzare un prodotto che soddisfi gli obiettivi prefissati dando all'utente un'esperienza che utilizzi al meglio le capacità del sistema. & MQPD04 - Tempo di risposta dei servizi all'utente\\
		\hline
		OQPD05 & Affidabilità & Si vuole che il prodotto fornito sia sempre disponibile e con meno errori possibili. Nel caso se ne verifichino il prodotto deve poter rispondere adeguatamente. & MQPD05 - Copertura dei test, MQPD06 - Robustezza agli errori\\
		\hline
		OQPD06 & Usabilità & Si vuole realizzare un prodotto facilmente usabile dagli utenti e che non richieda sforzi nel capire il suo funzionamento. & MQPD07 - Completezza di descrizione, MQPD08 - Completezza della guida utente\\
		\hline
		QQPD07 & Sicurezza & Si vuole realizzare un prodotto che garantisca la sicurezza dei sistemi e degli utenti che interagiscono con quest'ultimo. & MQPD10 - Procedure di autenticazione \\
		\hline
		OQPD08 & Manutenibilità & Si vuole ottenere un prodotto riutilizzabile e facilmente migliorabile in futuro. & MQPD11 - Accoppiamento\textsubscript{G} di componenti, MQPD12 - Adeguatezza della complessità ciclomatica\textsubscript{G}, MQPD13 - Completezza della funzione di test\\
		\hline
		OQPD09 & Compatibilità & Il prodotto dovrà essere accessibile al numero più elevato di utenti possibile, garantendo quindi la compatibilità con tutti i browser più diffusi. & MQPD14 - Browser supportati \\
		\hline
		\rowcolor{white}
		\caption{Obiettivi di qualità di prodotto\textsubscript{G}.}
	\end{xltabular}
\end{center}

\newpage
\subsubsection{Metriche di qualità di prodotto\textsubscript{G}}
Alcuni valori accettabili e ottimi per le metriche di qualità di prodotto\textsubscript{G} verranno fissati in futuro.
\begin{center}
	\setlength\extrarowheight{2pt}
	\rowcolors{2}{gray!10}{gray!40}
	\begin{xltabular}{\textwidth}{|c|X|X|X|X|}
		\hline
		\rowcolor{white}
		\textbf{ID} & \textbf{Descrizione} & \textbf{Obiettivo} & \textbf{Valore accettabile} & \textbf{Valore ottimo}\\
		\hline
		MQPD03 & Copertura funzionale & OQPD03 - Appropriatezza funzionale & 100\% dei requisiti\textsubscript{G} obbligatori & 100\% di tutti i requisiti\textsubscript{G}\\
		\hline
		MQPD04 & Tempo di risposta medio dei servizi all'utente & OQPD04 - Efficienza & 1s & 2s \\
		\hline
		MQPD05 & Copertura dei test & OQPD05 - Affidabilità & 80\% & 100\% \\
		\hline
		MQPD06 & Robustezza agli errori & OQPD05 - Affidabilità & 80\% & 100\% \\
		\hline
		MQPD07 & Completezza di descrizione & OQPD06 - Usabilità & 100\% &  100\% \\
		\hline
		MQPD08 & Completezza della guida utente & OQPD06 - Usabilità & 90\% &  100\% \\
		\hline
		MQPD09 & Interfaccia utente auto-esplicativa & OQPD06 - Usabilità & 85\% &  100\% \\
		\hline
		MQPD10 & Procedure di autenticazione & OQPD07 - Sicurezza & 25\% &  0\% \\
		\hline
		MQPD11 & Accoppiamento\textsubscript{G} di componenti & OQPD08 - Manutenibilità & - & - \\
		\hline
		MQPD12 & Adeguatezza della complessità ciclomatica\textsubscript{G} & OQPD08 - Manutenibilità & 1 & 5 \\
		\hline
		MQPD13 & Completezza della funzione di test & OQPD08 - Manutenibilità & 90\% & 100\% \\
		\hline
		MQPD14 & Browser supportati & OQPD09 - Compatibilità & 75\% & 100\% \\
		\hline
		\rowcolor{white}
		\caption{Metriche di qualità di prodotto\textsubscript{G}}
	\end{xltabular}
\end{center}
