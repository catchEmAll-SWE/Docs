\section{Specifiche dei test}

\subsection{Scopo della verifica\textsubscript{G} software}
La verifica\textsubscript{G} software serve per accertare che l'esecuzione delle attività attuate nel periodo in esame non abbia introdotto errori. La forma di verifica\textsubscript{G} software utilizzata dal gruppo \textit{Catch Em All} sarà l'Analisi Dinamica, che viene effettuata tramite test che richiedono l'esecuzione dell'oggetto di verifica\textsubscript{G}. In particolare, i test dovranno essere:
\begin{itemize}
	\item Ripetibili;
	\item Automatizzabili.
\end{itemize}
Gli oggetti della verifica\textsubscript{G} saranno le unità\textsubscript{G} software, le integrazioni tra unità\textsubscript{G}, e anche l'intero sistema.\\
La verifica\textsubscript{G} software così descritta prepara il successo della validazione\textsubscript{G} software, la quale invece servirà per accertare che il prodotto finale sia conforme alle aspettative.\\
Le specifiche dei test di integrazione ed unità\textsubscript{G} verranno definite nelle prossime versioni del presente documento.
\subsection{Test di unità}
Solitamente un'unità\textsubscript{G} software può essere realizzata da un singolo programmatore, e pertanto il test di unità, che ha il compito di verificare il comportamento di ogni unità\textsubscript{G} isolandola dalle altre, potrà essere a carico dello stesso autore. Il test di unità potrà considerarsi completo una volta che tutte le unità\textsubscript{G} software saranno state verificate.
\begin{center}
	\setlength\extrarowheight{5pt}
	\rowcolors{2}{gray!10}{gray!40}
	\begin{xltabular}{\textwidth}{|c|X|c|}
		\hline
		\rowcolor{white}
		\textbf{ID} & \textbf{Obiettivo test} & \textbf{Stato di implementazione} \\
		\hline
		TVU01 & Si verifichi che il numero di classi presenti nel CAPTCHA\textsubscript{G} sia compreso tra 2 e 4 & Superato\\
		\hline
		TVU02 & Si verifichi che il numero di immagini per classe presenti nel CAPTCHA\textsubscript{G} sia compreso tra 2 e 7 & Superato\\
		\hline
		TVU03 & Si verifichi che il numero di immagini visibili all'utente che compongono il CAPTCHA\textsubscript{G} sia sempre 9 & Superato\\
		\hline
		TVU04 & Si verifichi che le classi vengano ritornate correttamente a seguito di un'interrogazione al DB & Superato\\
		\hline
		TVU05 & Si verifichi che il numero di classi ritornato sia uguale a quello richiesto in seguito ad un'interrogazione al DB & Superato\\
		\hline
		TVU06 & Si verifichi che a seguito di una richiesta con un numero negativo di classi sia ritornato un errore OutOfBoundsException & Superato\\
		\hline
		TVU07 & Si verifichi che a seguito di una richiesta di immagini appartenenti alla stessa classe e con una specifica affidabilità, queste vengano ritornate nel formato corretto & Superato\\
		\hline
		TVU08 &  Si verifichi che a seguito di una richiesta di immagini appartenenti alla stessa classe e con una specifica affidabilità, il numero queste ultime sia corretto & Superato\\
		\hline
		TVU09 &  Si verifichi che a seguito di una richiesta di immagini appartenenti alla stessa classe e con una specifica affidabilità, la classe di ognuna sia corretta & Superato\\
		\hline
		TVU10 &  Si verifichi che a seguito di una richiesta di immagini appartenenti alla stessa classe e con una specifica affidabilità, l'affidabilità di ognuna sia corretta & Superato\\
		\hline
		TVU11 &  Si verifichi che a seguito di una richiesta di immagini appartenenti ad una classe inesistente sia ritornato un errore InvalidArgumentException & Superato\\
		\hline
		TVU12 &  Si verifichi che a seguito di una richiesta di un numero di immagini superiore a quello presente nel DB sia ritornato un errore InvalidArgumentException & Superato\\
		\hline
		TVU13 &  Si verifichi che l'operazione di modifica dell'affidabilità di un'immagine sia svolta correttamente & Superato\\
		\hline
		TVU14 &  Si verifichi che l'operazione di modifica dell'affidabilità di un'immagine inesistente non produca alcune cambiamento & Superato\\
		\hline
		TVU15 &  Si verifichi che la costruzione della soluzione di un CAPTCHA\textsubscript{G} ritorni il risultato desiderato & Superato\\
		\hline
		TVU16 &  Si verifichi che la soluzione del CAPTCHA\textsubscript{G} sia costruita nel formato corretto & Superato\\
		\hline
		TVU17 &  Si verifichi che alla costruzione del CAPTCHA\textsubscript{G} la soglia di affidabilità minima sia raggiunta & Superato\\
		\hline
		TVU18 &  Si verifichi che alla costruzione del CAPTCHA\textsubscript{G} la soglia di affidabilità minima per le immagini della classe target sia raggiunta & Superato\\
		\hline
		TVU19 &  Si verifichi che alla costruzione del CAPTCHA\textsubscript{G} la soglia di affidabilità minima per le immagini della classe non target sia raggiunta & Superato\\
		\hline
		TVU20 &  Si verifichi che le fixedString, utilizzate per il calcolo del proof of work\textsubscript{G} siano costruite nella maniera corretta & Superato\\
		\hline
		TVU21 & Si verifichi che data una soluzione corretta il risultato della verifica sia positivo  & Superato\\
		\hline
		TVU22 & Si verifichi che data una soluzione nella quale è stato selezionato l'honeypot\textsubscript{G} il risultato della verifica sia negativo  & Superato\\
		\hline
		TVU23 & Si verifichi che data una soluzione nella quale non sono state selezionate le immagini target affidabili il risultato della verifica sia negativo  & Superato\\
		\hline
		TVU24 & Si verifichi che data una soluzione nella quale sono state selezionate immagini non target affidabili il risultato della verifica sia negativo & Superato\\
		\hline
		TVU25 & Si verifichi che data una soluzione nella quale il numero di immagini non affidabili appartenenti alla classe target selezionate sia inferiore alla soglia minima il risultato della verifica sia negativo & Superato\\
		\hline
		TVU26 & Si verifichi che data una soluzione nella quale il numero di immagini non affidabili appartenenti alla classe target selezionate sia superiore alla soglia minima il risultato della verifica sia positivo & Superato\\
		\hline
		TVU27 & Si verifichi che data una soluzione errata per il completamento del proof of work\textsubscript{G} il risultato della verifica sia negativo & Superato\\
		\hline
		TVU28 & Si verifichi che data una soluzione corretta per il completamento del proof of work\textsubscript{G} il risultato della verifica sia positivo & Superato\\
		\hline
		TVU29 & Si verifichi che data una stringa criptata non valida, venga ritornato un errore nel momento dell'operazione di decrittazione & Superato\\
		\hline
		TVU30 & Si verifichi che data una stringa criptata valida, venga ritornata la stringa originale a seguito dell'operazione di decrittazione & Superato\\
		\hline
		TVU31 & Si verifichi che l'operazione di decrittazione ritorni sempre l'originale & Superato\\
		\hline
		\rowcolor{white}
		\caption{Test di unità}
	\end{xltabular}
\end{center}


\subsection{Test di integrazione}
I test di integrazione si applicano per testare la corretta interazione tra le componenti del sistema. Essi vengono definiti durante la progettazione architetturale e si basano sui componenti in essa specificati.
Per definire i test di integrazione è necessario selezionare quali funzionalità integrare individuandone le componenti coinvolte e ordinandole per dipendenze crescenti.
I problemi rilevati dai test di integrazione rappresentano difetti di progettazione o una scarsa qualità dei test di unità. Il numero dei test di integrazione è il necessario per accertare che i dati scambiati tra interfacce siano conformi e che i flussi di controllo siano tutti testati e funzionanti.
\begin{center}
	\setlength\extrarowheight{5pt}
	\rowcolors{2}{gray!10}{gray!40}
	\begin{xltabular}{\textwidth}{|c|X|c|}
		\hline
		\rowcolor{white}
		\textbf{ID} & \textbf{Obiettivo test} & \textbf{Stato di implementazione} \\
		\hline
		TVI01 & Si verifichi che il tentativo di richiesta per la generazione di un CAPTCHA\textsubscript{G} senza il bearer token corretto reindirizzi alla pagina contenente la documentazione & Superato\\
		\hline
		TVI02 & Si verifichi che il tentativo di richiesta per la generazione di un CAPTCHA\textsubscript{G} con tutti i parametri richiesti venga ritornato il json con i valori corretti & Superato\\
		\hline
		TVI03 & Si verifichi che il tentativo di richiesta per la verifica di un CAPTCHA\textsubscript{G} senza il bearer token corretto reindirizzi alla pagina contenente la documentazione & Superato\\
		\hline
		TVI04 & Si verifichi che il tentativo di richiesta per la verifica di un CAPTCHA\textsubscript{G} con una risposta in un formato non valido reindirizzi alla pagina contenente la documentazione & Superato\\
		\hline
		TVI05 & Si verifichi che il tentativo di richiesta per la verifica di un CAPTCHA\textsubscript{G} con una risposta valida e corretta ritorni uno status 200 e un json indicante il risultato positivo della verifica & Superato\\
		\hline
		TVI06 & Si verifichi che il tentativo di richiesta per la verifica di un CAPTCHA\textsubscript{G} con una risposta valida ma con il proof of work\textsubscript{G} calcolato in maniera errata ritorni uno status 200 e un json indicante il risultato negativo della verifica & Superato\\
		\hline
		TVI07 & Si verifichi che il tentativo di richiesta per la verifica di un CAPTCHA\textsubscript{G} con una risposta valida ma con le immagini selezionate in maniera errata ritorni uno status 200 e un json indicante il risultato negativo della verifica & Superato\\
		\hline
		TVI08 & Si verifichi che il tentativo di richiesta per la verifica di un CAPTCHA\textsubscript{G} con una risposta valida ma con l'immagine honeypot\textsubscript{G} selezionata ritorni uno status 200 e un json indicante il risultato negativo della verifica & Superato\\
		\hline
		TVI09 & Si verifichi che il tentativo di richiesta per la verifica di un CAPTCHA\textsubscript{G} inesistente ritorni uno status 404 & Superato\\
		\hline
		TVI10 & Si verifichi che il tentativo di richiesta per la verifica di un CAPTCHA\textsubscript{G} con una risposta valida ritorni il json con il risultato nel formato corretto & Superato\\
		\hline
		\rowcolor{white}
		\caption{Test di integrazione}
	\end{xltabular}
\end{center}

\subsection{Test di sistema}
I test di sistema sono finalizzati all'accertamento della copertura dei requisiti\textsubscript{G} individuati nella fase di analisi, e sono quindi test propedeutici al collaudo.
\begin{center}
	\setlength\extrarowheight{5pt}
	\rowcolors{2}{gray!10}{gray!40}
	\begin{xltabular}{\textwidth}{|c|X|c|c|}
		\hline
		\rowcolor{white}
		\textbf{ID} & \textbf{Obiettivo test} & \textbf{Stato di implementazione} & \textbf{Requisito correlato}\\
		\hline
		TVS01 & Si verifica che l’utente riesca ad effettuare
		il login in seguito alla corretta compilazione dei campi per le credenziali e del CAPTCHA\textsubscript{G} e dopo aver svolto il proof of work\textsubscript{G} & Non implementato & RF-1\\
		\hline
		TVS02 & Si verifica che l'utente possa inserire l'username nel campo corrispondente & Non implementato & RF-2\\
		\hline
		TVS03 & Si verifica che l'utente possa inserire la password nel campo corrispondente & Non implementato & RF-3\\
		\hline
		TVS04 & Si verifica che l'utente abbia superato con successo il CAPTCHA\textsubscript{G} in caso di autenticazione riuscita & Non implementato & RF-4\\
		\hline
		TVS05 & Si verifica che il margine di errore dato all'utente per la soluzione fornita sia calcolato correttamente & Non implementato & RF-5\\
		\hline
		TVS06 & Si verifica che l'utente abbia evitato l'honeypot\textsubscript{G} in caso di  autenticazione riuscita & Non implementato & RF-6\\
		\hline
		TVS07 & Si verifica che l'utente abbia completato il lavoro di proof of work\textsubscript{G} in caso di autenticazione riuscita & Non implementato & RF-7\\
		\hline
		TVS08 & Si verifica che all’utente venga mostrato un errore in caso di autenticazione fallita & Non implementato & RF-8\\
		\hline
		TVS09 & Si verifica che all’utente venga mostrato un errore in caso di inserimento di username non valido & Non implementato & RF-9\\
		\hline
		TVS10 & Si verifica che all’utente venga mostrato un errore in caso di inserimento di password non valida & Non implementato & RF-10\\
		\hline
		TVS11 & Si verifica che all’utente venga mostrato un errore in caso di non superamento del test CAPTCHA\textsubscript{G} immagini & Non implementato & RF-11\\
		\hline
		TVS12 & Si verifica che all’utente venga mostrato un errore in caso di non superamento del test honeypot\textsubscript{G} & Non implementato & RF-12\\
		\hline
		TVS13 & Si verifica che all’utente venga mostrato un errore in caso di non completamento del lavoro di proof of work\textsubscript{G} & Non implementato & RF-13\\
		\hline
		TVS14 & Si verifica che all’utente venga mostrato un errore in caso di superamento dei tentativi consentiti & Non implementato & RF-14\\
		\hline
		TVS15 & Si verifica che alla richiesta di un nuovo CAPTCHA\textsubscript{G} da parte dell'utente, questo venga generato correttamente & Non implementato & RF-15\\
		\hline
		TVS16 & Si verifica che all’utente venga mostrato un errore in caso di superamento delle richieste di generazione di nuovi CAPTCHA\textsubscript{G} & Non implementato & RF-16\\
		\hline
		TVS17 & Si verifica che il sistema fornisca correttamente i CAPTCHA\textsubscript{G} immagini & Non implementato & RF-17\\
		\hline
		TVS18 & Si verifica che il sistema generi correttamente la trappola honeypot\textsubscript{G} & Non implementato & RF-18\\
		\hline
		TVS19 & Si verifica che il sistema fornisca correttamente il test per il calcolo del proof of work\textsubscript{G} & Non implementato & RF-19\\
		\hline
		TVS20 & Si verifica che il sistema mitighi attacchi brute force\textsubscript{G} secondo le aspettative & Non implementato & RF-20\\
		\hline
		TVS21 & Si verifica che il sistema fornisca correttamente la funzionalità di verifica del CAPTCHA\textsubscript{G} & Non implementato & RF-21\\
		\hline
		TVS22 & Si verifica che il sistema fornisca correttamente la funzionalità di verifica della trappola honeypot\textsubscript{G} & Non implementato & RF-22\\
		\hline
		TVS23 & Si verifica che il sistema fornisca correttamente la funzionalità di verifica del proof of work\textsubscript{G} & Non implementato & RF-23\\
		\hline
		\rowcolor{white}
		\caption{Test di sistema}
	\end{xltabular}
\end{center}


\subsection{Test di regressione}
I test di regressione vengono utilizzati per accertare che le modifiche effettuate per aggiunta, correzione o rimozione, non pregiudichino le funzionalità già verificate in un periodo precedente, causando regressione. Consistono nella ripetizione dei test già definiti ed eseguiti con esito positivo in precedenza.

\subsection{Test di collaudo}
Il test di collaudo saranno supervisionati dal committente, per dimostrazione di conformità del prodotto rispetto alle aspettative.
