\appendix
\section{Resoconto delle attività di verifica}
\subsection{Periodo di analisi e produzione del proof of concept}
In questa sezione sono raccolti i vari resoconti delle attività di verifica svolti nei periodi precedenti alla revisione RTB, ovvero il periodo di analisi e quello di produzione del proof of concept.
Dato che non sono ancora state svolte attività di progettazione e codifica del prodotto finale, verranno misurare solo le metriche riguardanti i processi attivi.
\subsubsection{Gestione processi}
\begin{figure}[H]
	\centering
	\includegraphics[scale=1.1]{img/SPICE.png}
	\caption{Livello di capacità dei processi attivi nel progetto}
\end{figure}
\paragraph{Analisi retrospettiva sui risultati}\mbox{}\\
I processi primari di Fornitura e Sviluppo non essendo ancora ben monitorati e controllati sono ancora da considerarsi al primo livello; il gruppo sta lavorando per gestirli al meglio.\\
Anche i processi di Assicurazione della qualità e di Validazione non sono ancora al secondo livello. Il gruppo ha dovuto comprendere al meglio come assicurare qualità dei processi e prodotti del progetto, e come far si che gli obiettivi fissati siano stati raggiunti attraverso misurazioni utilizzando le metriche scelte. Il prossimo passo sarà monitorare al meglio e rendere ripetibili questi processi.\\
I processi di Documentazione e Verifica raggiungono invece il secondo livello, dato che sono da considerarsi ben monitorati e gestiti attraverso l'utilizzo di una checklist (vedi sezione §3.4.2.2 delle norme di progetto). Il processo di supporto di Documentazione è da considerarsi il più vicino al terzo livello di capability\textsubscript{G}.\\
Anche i processi di Gestione della configurazione e Gestione di progetto arrivano al livello due essendo ben gestiti e controllati dal gruppo attraverso gli strumenti scelti nel periodo iniziale di progetto.
\subsubsection{Pianificazione}
\subsubsubsection{Efficienza nell'utilizzo delle risorse}
\begin{figure}[H]
	\centering
	\includegraphics[scale=0.5]{img/BCWS-ACWS.png}
	\caption{Grafico che mostra l'andamento dei costi pianificati correlato a quelli reali}
\end{figure}
\paragraph{Analisi retrospettiva sui risultati}\mbox{}\\
Il costo reale rispetto a quello preventivato rientra nel range di errore previsto dal gruppo. Sono state utilizzate delle ore in più durante il secondo sprint per alcuni problemi avuti dal gruppo nell'analisi dei requisiti e casi d'uso del capitolato, dovendo confrontarsi sia con il proponente, che con il professor Cardin per chiarire i vari dubbi (i confronti hanno permesso la creazione di una solida base per lo sviluppo del PoC\textsubscript{G}). La validazione finale dei documenti per la revisione RTB nel sesto sprint ha avuto bisogno di alcune ore aggiuntive a causa di alcune verifiche approssimative nel periodo iniziale del progetto, dato che non tutte le norme erano state ancora ben definite.
\subsubsubsection{Variazioni dalla pianificazione}
\begin{figure}[H]
	\centering
	\includegraphics[scale=0.5]{img/SV.png}
	\caption{Grafico che mostra la differenza in percentuale tra le ore pianificate (ottime) e le ore effettivamente impiegate}
\end{figure}
\begin{figure}[H]
	\centering
	\includegraphics[scale=0.5]{img/CV.png}
	\caption{Grafico che mostra la differenza in percentuale tra i costi pianificati (ottimi) e i costi effettivi}
\end{figure}
\paragraph{Analisi retrospettiva sui risultati}\mbox{}\\
Sia le variazioni sulla pianificazione che quelle sui costi rientrano nel range d'errore che il gruppo si aspettava. Infatti le ore preventivate per le varie attività che si erano pianificate di svolgere sono state rispettate per la maggior parte. Sono state richieste alcune ore in più per l'analisi dei requisiti e per la programmazione del PoC\textsubscript{G}, il quale, al contrario, non ha avuto bisogno di tutte le ore preventivate per la sua progettazione. 
È stato riscontrato un problema nella previsione delle ore che i vari membri del gruppo avrebbero reso disponibile settimanalmente, più nello specifico si è sottovalutato l'impatto che altri impegni universitari ed esterni avrebbero avuto nello svolgimento del progetto, i quali hanno costituito un rallentamento e hanno costretto il gruppo ha cambiare le date pianificate per la revisione RTB. Il gruppo si impegnerà per mitigare e prevenire meglio questo tipo di rischi per le attività future.
\subsubsection{Documentazione}
\subsubsubsection{Indice di Gulpease}
\begin{table}[H]
	\centering
	\setlength\extrarowheight{5pt}
	\rowcolors{2}{gray!10}{gray!40}
	\renewcommand\tabularxcolumn[1]{>{\Centering}m{#1}}
	\begin{tabularx}{\textwidth}{| c | X |} 
		\hline
		\rowcolor{white}
		\textbf{Documento} & \textbf{Valore}\\
		\hline
		\textit{Analisi\_dei\_Requisiti v 1.0.0} & 90 \\
		\hline
		\textit{Norme\_di\_Progetto v 1.0.0} & 75\\
		\hline
		\textit{Piano\_di\_Progetto v 1.0.0} & 68\\
		\hline
		\textit{Piano\_di\_Qualifica v 1.0.0} & 84\\
		\hline
		\textit{Glossario v 1.0.0} & 69\\
		\hline
		\rowcolor{white}
		\caption{Indice di Gulpease}
	\end{tabularx}
\end{table}
\begin{figure}[H]
	\centering
	\includegraphics[scale=1.1]{img/Gulpease.png}
	\caption{Grafico che mostra l'indice di Gulpease per i vari documenti redatti}
\end{figure}
\paragraph{Analisi retrospettiva sui risultati}\mbox{}\\
I risultati ottenuti dai documenti sono soddisfacenti e superano la soglia che il gruppo ha definito accettabile. Tutti i documenti rilasciati hanno quindi un indice di leggibilità più che accettabile, alcuni superando anche l'ottimo definito. Non è stato calcolato l'indice sui vari verbali redatti, dato è stato utilizzato il template fornito dal servizio confluence di JIRA\textsubscript{G} per scriverli.
