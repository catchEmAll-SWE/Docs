\section{Obiettivi e metriche di qualità}
\subsection{Obiettivi e metriche di qualità di processo}
In questa sezione viene illustrato come il gruppo vuole verificare e misurare i progressi dei processi primari e di supporto nel corso del progetto. 
\subsubsection{Obiettivi di qualità di processo}
\begin{table}[H]
	\centering
	\begin{tabularx}{\textwidth}{|c|X|X|X|}
	\hline
	\textbf{ID} & \textbf{Nome} & \textbf{Descrizione} & \textbf{Metriche associate}\\
	\hline
	OQPC01 & Miglioramento continuo & Il processo si deve poter valutare e migliorare continuamente & MQPC01 - SPICE\\
	\hline
	OQPC02 & Efficienza nell'utilizzo delle risorse & Le risorse disponibili durante la durata del progetto devono essere distribuite ed utilizzate al meglio & MQPC02 - Costo pianificato di progetto; MQPC03 - Costo pianificato di progetto svolto; \hspace{25pt} MQPC04 - Costo reale di progetto svolto \\
	\hline
	OQPC03 & Variazioni dalla pianificazione & Assicurare che le scadenze e i limiti di costi illustrati nel documento \textit{Piano di progetto} siano rispettati &  MPC05: Variazioni nella programmazione;\hspace{65pt} MPC06: Variazioni nei costi. \\
	\hline
	\caption{Obiettivi di qualità di processo.}
	\end{tabularx}
\end{table}

\subsubsection{Metriche di qualità di processo}
\begin{table}[H]
	\centering
	\begin{tabularx}{\textwidth}{|c|X|X|X|X|}
		\hline
		\textbf{ID} & \textbf{Nome} & \textbf{Obiettivo} & \textbf{Valore accettabile} & \textbf{Valore ottimo}\\
		\hline
		MQPC01 & SPICE & OQPC01 - Miglioramento continuo & Level of Capability\textsubscript{G} $\geq$ 2 (Managed process) & Level of Capability\textsubscript{G} $\geq$ 4 (Predictable process) \\
		\hline
		MQPC02 & Costo pianificato di progetto & OQPC02 - Efficienza nell'utilizzo delle risorse & $\geq$ 0 \& $\le$ 11.100 & $\geq$ 0 \& $\le$ 11.100\\
		\hline
		MQPC03 & Costo pianificato di progetto svolto & OQPC02 - Efficienza nell'utilizzo delle risorse & BCWS $ \pm $ 10\% & BCWS \\
		\hline
		MQPC04 & Costo reale di progetto svolto & OQPC02 - Efficienza nell'utilizzo delle risorse & BCWS $ \pm $ 15\% & $\geq$ BCWS \\
		\hline
		MQPC05 & Variazioni nella pianificazione & OQPC03 - Rispetto della pianificazione & -15\% & 0\% \\
		\hline
		MQPC06 & Variazioni nei costi & OQPC03 - Rispetto della pianificazione & -15\% & 0\% \\
		\hline
		\caption{Metriche di qualità di processo.}
	\end{tabularx}
\end{table}
\subsubsubsection {Dettagli delle metriche di qualità di processo utilizzate}
Le metriche di qualità a cui ogni processo deve essere conforme sono:
\begin{itemize}
	\item \textbf{SPICE}: Riferito alla metrica per misurare il miglioramento continuo (MQPC01);
	\item \textbf{Costo pianificato di progetto}: Riferito alla metrica per misurare l'efficienza dell'utilizzo delle risorse (MQPC02);
	\item \textbf{Costo pianificato di progetto svolto}: Riferito alla metrica per misurare l'efficienza dell'utilizzo delle risorse (MQPC03);
	\item \textbf{Costo reale di progetto svolto}: Riferito alla metrica per misurare l'efficienza dell'utilizzo delle risorse (MQPC04);
	\item \textbf{Variazioni nella pianificazione}: Riferito alla metrica per misurare le variazioni dalla pianificazione (MQPC05);
	\item \textbf{Variazioni nei costi}: Riferito alla metrica per misurare le variazioni dalla pianificazione (MQPC06).
\end{itemize}
\paragraph{SPICE}\mbox{}\\
Questa metrica è di riferimento ai processi ed è stata scelta dal gruppo per verificare il grado di \textit{capability}\textsubscript{G} che ogni processo deve raggiungere. Lo standard definisce vari livelli di \textit{capability}\textsubscript{G}:
\begin{itemize}
	\item \textbf{Livello 0 - Incomplete process}: Il processo non è implementato oppure è incapace di raggiungere i suoi obiettivi;
	\item \textbf{Livello 1 - Performed process}: Il processo è attivo e può essere completato ma non è sottoposto a controlli;
	\item \textbf{Livello 2 - Managed process}: Il processo processo è attivo e pianificato, e può completare i suoi obbiettivi attraverso vari controlli;
	\item \textbf{Livello 3 - Established process}: Il processo è definito da degli standard;
	\item \textbf{Livello 4 - Predictable process}: Il processo è attivo secondo standard e viene controllato in modo dettagliato per renderlo in futuro prevedibile e ripetibile;
	\item \textbf{Livello 5 - Optimizing process}: Il processo è completamente definito e tracciato, e viene analizzato e migliorato in maniera continua.
\end{itemize}
Per misurare la \textit{capability}\textsubscript{G} si utilizzano i vari attributi di un processo:
\begin{itemize}
	\item Process performance;
	\item Performance management;
	\item Work product management;
	\item Process definition;
	\item Process deployment;
	\item Process measurement;
	\item Process control;
	\item Process innovation;
	\item Process optimization.
\end{itemize}
Ogni attributo di processo viene valutato su una scala di valutazione NPLF\textsubscript{G}.\\ Il gruppo si impegna a raggiungere un grado di \textit{capability}\textsubscript{G} minimo di 2 per ogni processo.
\paragraph{Costo pianificato di progetto}\mbox{}\\
Questa metrica è di riferimento ai processi ed è stata scelta dal gruppo per indicare il costo totale di progetto pianificato alla data corrente. Il valore si può osservare nella sezione \textit{Preventivo} del \textbf{Piano di progetto}. Questo valore deve essere $\geq$ 0 e minore del budget totale disponibile.
\paragraph{Costo pianificato di progetto svolto}\mbox{}\\
Questa metrica è di riferimento ai processi ed è stata scelta dal gruppo per indicare il valore effettivo del prodotto ottenuto fino alla data corrente.
\paragraph{Costo reale di progetto svolto}\mbox{}\\
Questa metrica è di riferimento ai processi ed è stata scelta dal gruppo per indicare il costo reale impiegato per svolgere il progetto fino alla data corrente. Il valore si può osservare nella sezione \textit{Consuntivo} del \textbf{Piano di progetto}. Questo valore deve essere un intorno\textsubscript{G} del BCWS con un errore non superiore al 20\%.
\paragraph{Variazioni nella pianificazione}\mbox{}\\
Questa metrica è di riferimento ai processi ed è stata scelta dal gruppo per misurare in che percentuale ci sono state variazioni rispetto alla pianificazione preventivata. Questa metrica si calcola come segue:
\begin{align*}
	\centering
	\textbf{VP} = \frac{100 * (BCWP - BCWS)}{BCWS}
\end{align*}
Dove:
\begin{itemize}
	\item \textbf{VP} sta per \textit{Variazione pianificazione};
	\item \textbf{BCWP} sta per \textit{Budgeted Cost of Work Performed};
	\item \textbf{BCWS} sta per \textit{Budgeted Cost of Work Scheduled}.
\end{itemize}
\paragraph{Variazioni nei costi}\mbox{}\\
Questa metrica è di riferimento ai processi ed è stata scelta dal gruppo per misurare in che percentuale ci sono state variazioni tra i costi di sviluppo pianificati e quelli reali. Questa metrica si calcola come segue:
\begin{align*}
	\centering
	\textbf{VC} = \frac{100 * (BCWS - ACWP)}{BCWS}
\end{align*}
Dove:
\begin{itemize}
	\item \textbf{VC} sta per \textit{Variazione costi};
	\item \textbf{ACWP} sta per \textit{Actual Cost of Work Performed};
	\item \textbf{BCWS} sta per \textit{Budgeted Cost of Work Scheduled}.
\end{itemize}
\newpage
\subsection{Obiettivi e metriche di qualità di prodotto}
Riferendoci alla serie di standard ISO/IEC 25000 SQuaRE possiamo osservare un insieme di caratteristiche che il prodotto deve avere per essere considerato di qualità. Queste caratteristiche saranno misurabili tramite metriche apposite, le quali forniranno i valori accettabili per il raggiungimento dell'obiettivo.
\subsubsection{Obiettivi di qualità di prodotto}
\subsubsubsection{Documentazione}
\begin{table}[H]
	\centering
	\begin{tabularx}{\textwidth}{|c|X|X|X|}
		\hline
		\textbf{ID} & \textbf{Nome} & \textbf{Descrizione} & \textbf{Metriche associate}\\
		\hline
		OQPD01 & Leggibilità dei documenti & I documenti devono essere comprensibile all'utente medio & MQPD01 - Indice di Gulpease\\
		\hline
		OQPD02 & Correttezza ortografica & I documenti devono essere scritti senza errori ortografici & MQPD02 - Correttezza documento \\
		\hline
		\caption{Obiettivi di qualità di processo specifici.}
	\end{tabularx}
\end{table}
\subsubsubsection{Software}
\begin{table}[H]
	\centering
	\begin{tabularx}{\textwidth}{|c|c|X|X|}
		\hline
		\textbf{ID} & \textbf{Nome} & \textbf{Descrizione} & \textbf{Metriche associate}\\
		\hline
		OQPD03 & Appropriatezza funzionale & Si vogliono soddisfare in modo completo i requisiti presenti nel documento \textit{Analisi dei requisiti} & MQPD03 - Copertura funzionale \\
		\hline
		OQPD04 & Efficienza & Si vuole realizzare un prodotto che soddisfi gli obiettivi prefissati dando all'utente un'esperienza che utilizzi al meglio le capacità del sistema. & MQPD04 - Tempo di risposta dei servizi all'utente\\
		\hline
		OQPD05 & Affidabilità & Si vuole che il prodotto fornito sia sempre disponibile e con meno errori possibili. Nel caso se ne verifichino il prodotto deve poter rispondere adeguatamente. & MQPD05 - Copertura dei test, MQPD06 - Robustezza agli errori\\
		\hline
		OQPD06 & Usabilità & Si vuole realizzare un prodotto facilmente usabile dagli utenti e che non richieda sforzi nel capire il suo funzionamento. & MQPD07 - Completezza di descrizione, MQPD08 - Completezza della guida utente\\
		\hline
		QQPD07 & Sicurezza & Si vuole realizzare un prodotto che garantisca la sicurezza dei sistemi e degli utenti che interagiscono con quest'ultimo. & MQPD10 - Procedure di autenticazione \\
		\hline
		OQPD08 & Manutenibilità & Si vuole ottenere un prodotto riutilizzabile e facilmente migliorabile in futuro. & MQPD11 - Accoppiamento\textsubscript{G} di componenti, MQPD12 - Adeguatezza della complessità ciclomatica\textsubscript{G}, MQPD13 - Completezza della funzione di test\\
		\hline
		OQPD09 & Compatibilità & Il prodotto dovrà essere accessibile al numero più elevato di utenti possibile, garantendo quindi la compatibilità con tutti i browser più diffusi. & MQPD14 - Browser supportati \\
		\hline
		\caption{Obiettivi di qualità di prodotto.}
	\end{tabularx}
	
\end{table}

\subsubsection{Metriche di qualità di prodotto}
Alcuni valori accettabili e ottimi per le metriche di qualità di prodotto verranno fissati in futuro.
\begin{table}[H]
	\centering
	\begin{tabularx}{\textwidth}{|c|X|X|X|X|}
		\hline
		\textbf{ID} & \textbf{Descrizione} & \textbf{Obiettivo} & \textbf{Valore accettabile} & \textbf{Valore ottimo}\\
		\hline
		MQPD01 & Indice di Gulpease & OQPD01 - Leggibilità dei documenti & $\geq$ 40 & $\geq$ 80 \\
		\hline
		MQPD02 & Numero errori ortografici & OQPD02 - Correttezza ortografica & 0 & 0 \\
		\hline
		MQPD03 & Copertura funzionale & OQPD03 - Appropriatezza funzionale & 100\% dei requisiti obbligatori & 100\% di tutti i requisiti\\
		\hline
		MQPD04 & Tempo di risposta dei servizi all'utente & OQPD04 - Efficienza & - & - \\
		\hline
		MQPD05 & Copertura dei test & OQPD05 - Affidabilità & 100\% & 100\% \\
		\hline
		MQPD06 & Robustezza agli errori & OQPD05 - Affidabilità & 80\% & 100\% \\
		\hline
		MQPD07 & Completezza di descrizione & OQPD06 - Usabilità & 100\% &  100\% \\
		\hline
		MQPD08 & Completezza della guida utente & OQPD06 - Usabilità & 80\% &  100\% \\
		\hline
		MQPD09 & Interfaccia utente auto-esplicativa & OQPD06 - Usabilità & 70\% &  100\% \\
		\hline
		MQPD10 & Procedure di autenticazione & OQPD07 - Sicurezza & 25\% &  0\% \\
		\hline
		MQPD11 & Accoppiamento\textsubscript{G} di componenti & OQPD08 - Manutenibilità & - & - \\
		\hline
		MQPD12 & Adeguatezza della complessità ciclomatica\textsubscript{G} & OQPD08 - Manutenibilità & - & - \\
		\hline
		MQPD13 & Completezza della funzione di test & OQPD08 - Manutenibilità & 90\% & 100\% \\
		\hline
		MQPD14 & Browser supportati & OQPD09 - Compatibilità & 75\% & 100\% \\
		\hline
		\caption{Metriche di qualità di prodotto.}
	\end{tabularx}
\end{table}
\newpage
\subsubsubsection {Dettagli delle metriche di qualità di prodotto utilizzate}
Le metriche di qualità a cui ogni prodotto deve essere conforme sono divisi in due categorie:
\begin{itemize}
	\item \textbf{Metriche per la qualità della documentazione};
	\item \textbf{Metriche per la qualità del software}.
\end{itemize}
\subsubsubsection{Metriche di qualità della documentazione}
Le metriche di qualità a cui solo la documentazione deve essere conforme sono:
\begin{itemize}
	\item \textbf{Indice di Gulpease}: Che fa riferimento alla metrica \textit{MQPD01};
	\item \textbf{Correttezza ortografica}: Che fa riferimento alla metrica \textit{MQPD02}.
\end{itemize}
\paragraph{Indice di Gulpease}\mbox{}\\
L'indice di Gulpease è una metrica di riferimento ai prodotti di documentazione che il gruppo ha scelto di utilizzare per verificare la leggibilità della documentazione prodotta. L'indice è tarato sulla lingua italiana e si calcola in questo modo:
\begin{align*}
	\centering
	\textbf{IG} = 89 + \frac{300 * Nfrasi - 10 * Nlettere}{Nparole}
\end{align*}
Il gruppo ha scelto come valore minimo di accettabilità 40. Questo viene indicato come limite dato che un valore minore implica una difficoltà di lettura anche per chi ha conferito un diploma di scuola superiore.
\paragraph{Correttezza ortografica}\mbox{}\\
Questa metrica è di riferimento ai prodotti di documentazione ed è utilizzata dal gruppo per assicurare la correttezza ortografica di ogni parola presente nei documenti. Non devono esserci errori grammaticali per far sì che un documento sia accettato.
\subsubsubsection{Metriche di qualità del software}
Le metriche di qualità a cui solo il software deve essere conforme sono:
\begin{itemize}
	\item \textbf{Copertura funzionale}: Che fa riferimento alla metrica \textit{MQPD03};
	\item \textbf{Tempo di risposta dei servizi all'utente}: Che fa riferimento alla metrica \textit{MQPD04};
	\item \textbf{Copertura dei test}: Che fa riferimento alla metrica \textit{MQPD05};
	\item \textbf{Robustezza agli errori}: Che fa riferimento alla metrica \textit{MQPD06};
	\item \textbf{Completezza di descrizione}: Che fa riferimento alla metrica \textit{MQPD07};
	\item \textbf{Completezza della guida utente}: Che fa riferimento alla metrica \textit{MQPD08};
	\item \textbf{Interfaccia utente auto-esplicativa}: Che fa riferimento alla metrica \textit{MQPD09};
	\item \textbf{Procedure di autenticazione}: Che fa riferimento alla metrica \textit{MQPD10};
	\item \textbf{Accoppiamento\textsubscript{G} di componenti}: Che fa riferimento alla metrica \textit{MQPD11};
	\item \textbf{Adeguatezza della complessità ciclomatica}\textsubscript{G}: Che fa riferimento alla metrica \textit{MQPD12};
	\item \textbf{Completezza della funzione di test}: Che fa riferimento alla metrica \textit{MQPD13};
	\item \textbf{Browser supportati}: Che fa riferimento alla metrica \textit{MQPD14}.
\end{itemize}
\paragraph{Copertura funzionale}\mbox{}\\
Questa metrica è di riferimento ai prodotti software ed è utilizzata dal gruppo per verificare che tutti i requisiti obbligatori del progetto siano stati integrati nel prodotto finale. Questa metrica è calcolata attraverso il rapporto tra il numero di requisiti soddisfatti e quello di requisiti obbligatori totali:
\begin{align*}
	\centering
	\textbf{CF} = \frac{RqSoddisfatti}{RqTotali}
\end{align*}
Dove \textbf{CF} sta per \textit{Copertura funzionale}.
\paragraph{Tempo di risposta dei servizi all'utente}\mbox{}\\
Questa metrica è di riferimento ai prodotti software ed è utilizzata dal gruppo per assicurare che i tempi di risposta del prodotto siano accettabili. Un tempo di risposta adeguato in un sistema CAPTCHA\textsubscript{G} è molto importante e per questo è un obiettivo fondamentale. Il valore accettabile verrà analizzato in una fase più avanzata di progetto.
\paragraph{Copertura dei test}\mbox{}\\
Questa metrica è di riferimento ai prodotti software ed è utilizzata dal gruppo per verificare che i test svolti sul prodotto finale coprano tutti i requisiti e casi d’uso identificati. Questa metrica è calcolata attraverso il rapporto tra il numero di requisiti e casi d'uso testati e quello di requisiti e casi d'uso totali da testare:
\begin{align*}
	\centering
	\textbf{CdT} = \frac{RqUCTestati}{RqUCTotali}
\end{align*}
Dove \textbf{CdT} sta per \textit{Copertura dei test}.
\paragraph{Robustezza agli errori}\mbox{}\\
Questa metrica è di riferimento ai prodotti software ed è utilizzata dal gruppo per verificare quale parte di tutti gli errori critici, ovvero quelli che possono determinare blocchi del sistema, è stata messa sotto controllo. Questa metrica è calcolata attraverso il rapporto tra il numero di errori critici gestiti e il numero totale di errori critici da gestire in totale:
\begin{align*}
	\centering
	\textbf{RaE} = \frac{ErrCritGestiti}{ErrCritTotali}
\end{align*}
Dove \textbf{RaE} sta per \textit{Robustezza agli errori}.
\newpage
\paragraph{Completezza di descrizione}\mbox\\
Questa metrica è di riferimento ai prodotti software ed è utilizzata dal gruppo per verificare la percentuale degli scenari d’uso che è descritta nella documentazione rispetto al totale. Questo per poter garantire informazioni complete agli utilizzatori del prodotto. Questa metrica è calcolata attraverso il rapporto tra il numero di scenari descritti e il numero di scenari effettivamente presenti nel dominio\textsubscript{G}:
\begin{align*}
	\centering
	\textbf{CdD} = \frac{ScenariDescritti}{ScenariPresenti}
\end{align*}
Dove \textbf{CdD} sta per \textit{Completezza di descrizione}.
\paragraph{Completezza della guida utente}\mbox{}\\
Questa metrica è di riferimento ai prodotti software ed è utilizzata dal gruppo per verificare la percentuale delle funzioni utilizzabili dall'utente che hanno una descrizione completa nei vari manuali. Questa metrica è calcolata attraverso il rapporto tra il numero di funzionalità descritte e il numero di funzionalità totali:
\begin{align*}
	\centering
	\textbf{CdGU} = \frac{FunzDescritte}{FunzTotali}
\end{align*}
Dove \textbf{CdGU} sta per \textit{Completezza della guida utente}.
\paragraph{Interfaccia utente auto-esplicativa}\mbox{}\\
Questa metrica è di riferimento ai prodotti software ed è utilizzata dal gruppo per verificare la percentuale degli elementi di informazione che sono presentati all’utente inesperto in modo che possa completare un’attività senza un addestramento preliminare o assistenza esterna. Questa metrica è calcolata attraverso il rapporto tra il numero di informazioni fornite all'utente rispetto a quelle di cui avrebbe bisogno per completare ogni piccolo passo:
\begin{align*}
	\centering
	\textbf{IUAE} = \frac{InfoFornite}{InfoRichieste}
\end{align*}
Dove \textbf{IUAE} sta per \textit{Interfaccia utente auto-esplicativa}.
\paragraph{Procedure di autenticazione}\mbox{}\\
Questa metrica è di riferimento ai prodotti software ed è utilizzata dal gruppo per verificare il grado di efficacia del sistema CAPTCHA\textsubscript{G} implementato per l'autenticazione di un utente. Il gruppo definisce un grado di accettabilità per la percentuale di accessi indesiderati non bloccati:
\begin{center}
	\textbf{AINB} $\le$ 25\%
\end{center}
Dove \textbf{AINB} sta per \textit{Accessi indesiderati non bloccati}.
\paragraph{Accoppiamento\textsubscript{G} di componenti}\mbox{}\\
Questa metrica è di riferimento ai prodotti software ed è utilizzata dal gruppo per controllare quanti componenti del sistema sono strettamente  indipendenti e quanti sono esenti da impatti conseguenti a cambiamenti negli altri componenti.
In futuro verrà definito un valore per misurarla al meglio. 
\newpage
\paragraph{Adeguatezza della complessità ciclomatica}\textsubscript{G}\mbox{}\\
Questa metrica è di riferimento ai prodotti software ed è utilizzata dal gruppo per verificare quanti moduli software hanno una complessità ciclomatica\textsubscript{G} accettabile. Per verificarla il gruppo deciderà una soglia di accettabilità per i vari linguaggi di programmazione e per il tipo di modulo o di funzione utilizzati durante il progetto.\\
Nelle metriche software la complessità ciclomatica\textsubscript{G} è usata per valutare la complessità di un algoritmo ed è basata sulla struttura del grafo che rappresenta l’algoritmo da misurare. Per calcolarla si fa uso di questa formula:
\begin{align*}
	\centering
	\textbf{v(G)} = L - N + 2 * P
\end{align*}
Dove:
\begin{itemize}
	\item \textbf{v(G)}: Numero ciclomatico relativo al grafo G;
	\item \textbf{L}: Numero di archi nel grafo;
	\item \textbf{N}: Numero di nodi del grafo;
	\item \textbf{P}: Numero dei componenti del grafo disconnessi.
\end{itemize}
\paragraph{Completezza della funzione di test}\mbox{}\\
Questa metrica è di riferimento ai prodotti software ed è utilizzata dal gruppo per verificare la percentuale di completezza delle funzioni di test implementate. Questa metrica è calcolata attraverso il rapporto tra il numero di test implementati e il numero di test totali da fare:
\begin{align*}
	\centering
	\textbf{CdFT} = \frac{TestImpl}{TestTot}
\end{align*}
Dove \textbf{CdFT} sta per \textit{Completezza della funzione di test}.
\paragraph{Browser supportati}\mbox{}\\
Questa metrica è di riferimento ai prodotti software ed è utilizzata dal gruppo per verificare il numero di browser che supportano il prodotto sviluppato. Il gruppo definirà un grado di accettabilità per la percentuale di browser che deve supportare il prodotto.
\begin{center}
	\textbf{BRS} $\geq$ 75\%
\end{center}
Dove \textbf{BRS} sta per \textit{Browser supportati}.

