\section{Obiettivi e metriche di qualità}
\subsection{Obiettivi e metriche di qualità di processo}
In questa sezione vengono illustrati come il gruppo vuole verificare e misurare i progressi dei processi primari e secondari nel corso del progetto.
\subsubsection{Obiettivi di qualità di processo}
\begin{table}[H]
	\centering
	\begin{tabularx}{\textwidth}{|c|X|X|X|}
	\hline
	\textbf{ID} & \textbf{Nome} & \textbf{Descrizione} & \textbf{Metriche associate}\\
	\hline
	OQPC01 & Miglioramento continuo & Il processo si deve poter valutare e migliorare continuamente & MQPC01 - SPICE\\
	\hline
	OQPC02 & Efficienza nell'utilizzo delle risorse & Le risorse disponibili durante la durata del progetto devono essere distribuite ed utilizzate al meglio & MQPC02 - Costo previsto di un'attività programmata; \hspace{25pt} MQPC03 - Costo previsto di un'attività svolta; MQPC04 - Costo reale di un'attività svolta \\
	\hline
	OQPC03 & Rispetto della pianificazione & Assicurare che le scadenze e i limiti di costi illustrati nel documento \textit{piano\_di\_progetto} siano rispettati &  MPC05: Variazioni nella programmazione;\hspace{65pt} MPC06: Variazioni nei costi. \\
	\hline
	OQPC04 & Leggibilità dei documenti & I documenti devono essere comprensibile all'utente medio & MQPC07 - Indice di Gulpease\\
	\hline
	OQPC05 & Correttezza ortografica & I documenti devono essere scritti senza errori ortografici & MQPC08 - Correttezza documento \\
	\hline
	\end{tabularx}
	\caption{Tabella 3: Obiettivi di qualità di processo.}
\end{table}
\subsubsection{Metriche di qualità di processo}
\begin{table}[H]
	\centering
	\begin{tabularx}{\textwidth}{|c|X|X|X|X|}
		\hline
		\textbf{ID} & \textbf{Nome} & \textbf{Obiettivo} & \textbf{Valore accettabile} & \textbf{Valore ottimo}\\
		\hline
		MQPC01 & SPICE & OQPC01 - Miglioramento continuo & Level of Capability $\geq$ 2 (Managed process) & Level of Capability $\geq$ 4 (Predictable process) \\
		\hline
		MQPC02 & Costo previsto di un'attività programmata & OQPC02 - Efficienza nell'utilizzo delle risorse & $\geq$ 0 & $\geq$ 0 \\
		\hline
		MQPC03 & Costo previsto di un'attività svolta & OQPC02 - Efficienza nell'utilizzo delle risorse & $\geq$ 0 & $\geq$ BCWS \\
		\hline
		MQPC04 & Costo reale di un'attività svolta & OQPC02 - Efficienza nell'utilizzo delle risorse & $\geq$ BCWS & $\geq$ BCWS \\
		\hline
		MQPC05 & Variazioni nella programmazione & OQPC03 - Rispetto della pianificazione & -20\% & 0\% \\
		\hline
		MQPC06 & Variazioni nei costi & OQPC03 - Rispetto della pianificazione & -15\% & 0\% \\
		\hline
	\end{tabularx}
	\caption{Tabella 4: Metriche di qualità di processo.}
\end{table}
\subsection{Obiettivi e metriche di qualità di prodotto}
Riferendoci allo standard ISO/IEC 25000 SQuaRE è possibile osservare un insieme di caratteristiche che il prodotto deve avere per essere considerato di qualità. Queste caratteristiche saranno misurabili tramite metriche apposite, le quali forniranno i valori accettabili per il raggiungimento dell'obiettivo.
\subsubsection{Obiettivi di qualità di prodotto}
\begin{table}[H]
	\centering
	\begin{tabularx}{\textwidth}{|c|c|X|c|}
		\hline
		\textbf{ID} & \textbf{Nome} & \textbf{Descrizione} & \textbf{Metriche associate}\\
		\hline
		MQPD01 & Funzionalità & Si vogliono soddisfare in modo completo i requisiti presenti nel documento \textit{Analisi\_dei\_requisiti} & MQPD01 - \\
		\hline
		MQPD02 & Efficienza & Si vuole realizzare un prodotto che soddisfi gli obiettivi prefissati utilizzando il minor numero di risorse possibili. & MQPD02 - \\
		\hline
		MQPD03 & Affidabilità & Si vuole che il prodotto fornito sia sempre disponibile e con meno errori possibili. Nel caso se ne verifichino il prodotto deve poter rispondere adeguatamente. & MQPD03 -\\
		\hline
		MQPD04 & Usabilità & Si vuole realizzare un prodotto facilmente usabile dagli utenti e che non richieda sforzi nel capire il suo funzionamento. & MQPD04 -\\
		\hline
		MQPD05 & Sicurezza & Si vuole realizzare un prodotto che garantisca la sicurezza dei sistemi e degli utenti che interagiscono con quest'ultimo. & MQPD05 -\\
		\hline
		MQPD06 & Manutenibilità & Si vuole ottenere un prodotto riutilizzabile e facilmente migliorabile in futuro. & MQPD06 -\\
		\hline
		MQPD07 & Compatibilità & Il prodotto dovrà essere accessibile al numero più elevato di utenti possibile, garantendo quindi la compatibilità con tutti i browser più diffusi. & MQPD07 -\\
		\hline
	\end{tabularx}
	\caption{Tabella 5: Obiettivi di qualità di prodotto.}
\end{table}
\subsubsection{Metriche di qualità di prodotto}
MQPD01