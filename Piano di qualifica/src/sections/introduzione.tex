\section{Introduzione}

\subsection{Scopo del documento}
Questo documento ha come obiettivo quello di fissare gli standard che permetteranno al gruppo \textit{Catch Em All} di garantire qualità al prodotto e ai processi durante l'intera durata del progetto. Verranno quindi definiti metodi di verifica\textsubscript{G} e validazione\textsubscript{G} continui che permetteranno al gruppo di agire in modo rapido e incisivo nel momento in cui si dovranno fare delle correzioni su eventuali errori o andamenti indesiderati. Questo allo scopo di sprecare meno risorse possibili e produrre un prodotto che sia facilmente mantenibile. 

\subsection{Scopo del prodotto}
Gli attuali sistemi di rilevazione dei bot\textsubscript{G} rispetto agli esseri umani prevedono l'utilizzo di un test CAPTCHA\textsubscript{G}, progettato per cercare di bloccare azioni con fini malevoli nel web da parte di sistemi automatizzati. Nel capitolato “CAPTCHA: Umano o Sovrumano?” viene evidenziata una criticità presente in tali sistemi: grazie ai notevoli progressi nel campo dell’intelligenza artificiale si è nel tempo giunti al punto che i task\textsubscript{G} i quali si ritenevano impossibili (o quantomeno, molto difficili) da svolgere per una macchina ora vengono effettuate dai bot\textsubscript{G} talvolta persino meglio delle persone.
Dal proponente “Zucchetti S.p.A” viene richiesto lo sviluppo di un'applicazione web contenente una pagina di login con un sistema in grado di rilevare i bot\textsubscript{G} rispetto agli esseri umani in maniera più efficace.

\subsection{Glossario}
Per risolvere ambiguità relative al linguaggio utilizzato nei documenti prodotti, è stato creato un documento denominato \textbf{Glossario v.1.0.0}. Questo documento fornisce le definizioni relative a tutti i termini tecnici utilizzati nei vari documenti, segnalando questi termini con pedice G accanto alla parola.

\subsection{Standard di progetto}
Per lo svolgimento del progetto il gruppo \textit{Catch Em All} ha scelto di utilizzare come riferimenti formativi la serie standard \textbf{ISO/IEC 25000 SQuaRE} per i requisiti\textsubscript{G} e valutazione della qualità di un prodotto e lo standard \textbf{ISO/IEC 15504 SPICE} per definire al meglio la qualità di un processo.

\subsection{Riferimenti}
\subsubsection{Riferimenti normativi}\:
\begin{itemize}
	\item Norme di Progetto v0.0.4;
	\item Capitolato d'appalto C1 \textit{CAPTCHA: Umano o Sovrumano?} : \\
		\url{https://www.math.unipd.it/~tullio/IS-1/2022/Progetto/C1.pdf}.
\end{itemize}
	
\subsubsection{Riferimenti informativi}\:
\begin{itemize}
	\item Processi di ciclo di vita - Materiale didattico del corso di Ingegneria del Software: \\
		\url{https://www.math.unipd.it/~tullio/IS-1/2022/Dispense/T03.pdf};
	\item Qualità di prodotto\textsubscript{G} - Materiale didattico del corso di Ingegneria del Software: \\
		\url{https://www.math.unipd.it/~tullio/IS-1/2022/Dispense/T08.pdf};
	\item Qualità di processo\textsubscript{G} - Materiale didattico del corso di Ingegneria del Software: \\
		\url{https://www.math.unipd.it/~tullio/IS-1/2022/Dispense/T09.pdf};
	\item Standard SQuaRE: \\
		\url{http://www.iso25000.it/styled/};
	\item Standard SPICE: \\
		\url{https://en.wikipedia.org/wiki/ISO/IEC_15504};
	\item Matriche di prodotto: \\
		\url{https://metriche-per-il-software-pa.readthedocs.io/it/latest/documento-in-consultazione/metriche-e-strumenti.html#misurazioni-di-manutenibilita}
	\item Metriche di progetto: \\
		\url{https://it.wikipedia.org/wiki/Metriche_di_progetto}.
\end{itemize}

