\section{Processi primari}

\subsection{Acquisizione}
\textit{Zucchetti S.p.A.} richiede la realizzazione di un progetto creativo riguardante lo sviluppo di un sistema Captcha attraverso l'esposizione della lettera di presentazione \textit{"CAPTCHA: Umano o Sovrumano?"} in data 18 ottobre 2022.

\subsection{Fornitura}
\textit{CatchEmAll} in seguito alla presentazione dei capitolati si riunisce per valutare le proposte e le opinioni dei componenti del team: in seguito ad un processo di valutazione (inizialmente generico poi specifico, riassunto nella sezione \textit{Motivazione scelta capitolato} del documento \textit{Lettera di presentazione}) emerge la preferenza per il progetto proposto dal referente Dr. Gregorio Piccoli.  
Successivamente viene schedulata una riunione con il proponente per approfondire le richieste e rispondere a domande. 
In data 28 ottobre 2022 viene inviata al committente la \href{https://github.com/catchEmAll-SWE/catchEmAll-Docs/blob/main/Assegnazione%20appalti/LetteraCandidatura.pdf}{lettera di presentazione} il quale, in seguito ad una richiesta di modifica delle tempistiche di consegna, ci aggiudica l'appalto.

\subsection{Sviluppo}
    \subsubsection{Versionamento}
    GitHub è lo strumento utilizzato dal gruppo per il versionamento del codice.
    Il team è identificato in tale piattaforma come organizzazione (\href{https://github.com/catchEmAll-SWE}{vedi}).
    Inoltre, al fine di documentare il più possibile, ogni commit che porta valore al progetto risolvendo (totalmente o anche solo parzialmente) un ticket fa riferimento a quest'ultimo. 
    
    \subsubsection{Issues tracking}
    Si è deciso di di adottare il framework \textbf{Srum} per la gestione del ciclo di sviluppo del progetto con le seguenti caratteristiche:
    \begin{itemize}
        \item sprint della durata di una settimana
        \item utilizzo di una board avente 4 stati:
            \begin{itemize}
                \item to do
                \item in progress
                \item in review (ogni ticket deve essere validato da uno o più componenti del gruppo per essere considerato chiuso)
                \item done
            \end{itemize}
    \end{itemize}
    JIRA piattaforma che offre un servizio di \textit{Issue Tracking} è il supporto scelto per implementare tale metodo agile.
    La definizione dei ticket è regolata dalla seguente convenzione:
    \begin{itemize}
        \item titolo e descrizione devono, oltre ad essere sempre presenti, eplicitare in maniera chiara il problema
        \item utilizzo di label
        \item stima del lavoro necessario al completamento
        \item utilizzo di ereditarietà (rapporti di parentela)
    \end{itemize}

