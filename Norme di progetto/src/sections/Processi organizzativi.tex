\section{Processi organizzativi}
    \subsection{Gestione delle comunicazioni}
        \subsubsection{Comunicazioni interne}
        Le comunicazioni interne:
        \begin{itemize}
            \item riguardano solamente i componenti del team 
            \item avvengono su \textit{WhatsApp}.
            \item utilizzate per:
                \begin{itemize}
                    \item comunicazioni istantanee tra tutti i componenti
                    \item discussioni
                    \item pianificazione degli incontri
                    \item \textit{daily scrum meeting} 
                \end{itemize}
        \end{itemize}
            
        \subsubsection{Comunicazioni esterne}
        Le comunicazioni esterne:
        \begin{itemize}
            \item riguardano il gruppo e le altre figure (proponente e committente)
            \item utilizzo del dominio di gruppo (\textit{catchemallswe3@gmail.com}) di posta elettronica
            \item utilizzate per comunicazioni ufficiali tra il team e le altre figure
        \end{itemize}

        
        
    \subsection{Gestione degli incontri}
        \subsubsection{Incontri interni}
        Gli incontri interni sono necessari sia per una corretta adozione del framework Scrum (incontro organizzativo settimanale)
        sia per permettere al team di interagire direttamente, discutendo, proponendo e valutando idee, problematiche e possibili 
        soluzioni: per questo si tratta di uno strumento largamente utilizzato
        \newline
        Si predilige la modalità virtuale per comodità cercando di schedulare riunioni in cui tutti riescano a partecipare.
        \newline
        La piattaforma utilizzata è \textit{discord}, la quale permette la creazione e l'utilizzo di:
        \begin{itemize}
            \item canali testuali
            \item canali video (con possibilità di condivisione schermo)
        \end{itemize}
        Al termine degli incontri il responsabile di progetto inserisce nello sprint corrente il compito di redigere i verbali. 

        \subsubsection{Incontri esterni}
        Gli incontri esterni sono schedulati in seguito alla presenza di dubbi (implementativi, riguardanti requisiti o richieste di altro tipo) all'interno del 
        team: questi incontri sono preceduti dallo svolgimento di una o più riunioni interne nelle quali si affrontano e si definiscono tali problematiche.
        \newline
        Per quanto riguarda l'organizzazione viene contattato tramite email il referente di progetto proponendogli diverse date e orari affinchè si trovi quella 
        più comoda per entrambe le parti. 
        \newline
        Come per quelli interni gli incontri esterni sono tenuti in modalità virtuale ma a loro differenza si utilizza una riunione \textit{Zoom} definita dal gruppo. 
        \newline
        I verbali hanno lo scopo di documentare in maniera dettagliata tutti gli argomenti trattati affinchè si possa costruire uno storico identificando
        e motivando le decisioni prese.
        \newline
        Come per quelli interni il responsabile di progetto inserisce nello sprint corrente il compito di redigere tali documenti.

