\section{Introduzione}

\subsection{Scopo del documento}
Questo documento ha come obiettivo il fissare gli standard che permetteranno al gruppo \textit{Catch Em All} di garantire qualità al prodotto\textsubscript{G} e ai processi durante l'intera durata del progetto. Verranno quindi definiti metodi di verifica\textsubscript{G} e validazione\textsubscript{G} continui che permetteranno al gruppo di agire in modo rapido e incisivo nel momento in cui si dovranno fare delle correzioni su eventuali errori o andamenti indesiderati. Questo allo scopo di sprecare meno risorse possibili e produrre un prodotto che sia facilmente manutenibile. 

\subsection{Scopo del prodotto}
Dal proponente \textit{Zucchetti S.p.A.} viene evidenziata, nel capitolato da loro proposto, una criticità negli attuali sistemi di sicurezza sulla rilevazione dei bot\textsubscript{G} rispetto agli esseri umani. Oggigiorno il meccanismo più utilizzato per risolvere questo problema è il test CAPTCHA\textsubscript{G}.\\
Un bot\textsubscript{G} non è altro che una procedura automatizzata che, in questo caso, ha fini malevoli, come per esempio:
\begin{itemize}
	\item Registrazione presso siti web;
	\item Creazione di spam\textsubscript{G};
	\item Violare sistemi di sicurezza.
\end{itemize}
I bot\textsubscript{G}, grazie alle nuove tecnologie sviluppate con sistemi che utilizzano principalmente l'intelligenza artificiale, riescono a svolgere compiti che fino a poco tempo fa venivano considerati impossibili da svolgere per una macchina.\\
Ciò evidenzia che i CAPTCHA\textsubscript{G} attuali risultano sempre più obsoleti, non andando a individuare correttamente tutti i bot\textsubscript{G}, se non quasi nessuno.\\
Un'altra criticità individuata dal proponente è il sistema di classificazione delle immagini che sta effettuando Google grazie al proprio reCAPTCHA\textsubscript{G}, che attualmente è il sistema più diffuso.\\
Questa criticità nasce dal beneficio che questa big tech\textsubscript{G} ottiene dall'interazione degli utenti nel risolvere i task\textsubscript{G} proposte, che portano alla creazione di enormi dataset\textsubscript{G} di immagini classificate che possono essere utilizzate per l'apprendimento dei propri sistemi di machine learning, oppure possono essere vendute a terzi.\\
Il capitolato C1 richiede di sviluppare un' applicazione web costituita da una pagina di login provvista di questo sistema di rilevazione in grado di distinguere un utente umano da un bot\textsubscript{G}.\\
L'utente quindi, dopo aver compilato il form in cui inserirà il nome utente e la password, dovrà svolgere una task\textsubscript{G} che sarà il cosiddetto test CAPTCHA\textsubscript{G}.



\subsection{Glossario}
Per evitare ambiguità relative al linguaggio utilizzato nei documenti prodotti, viene fornito il \textbf{Glossario v 1.0.0}. In questo documento sono contenuti tutti i termini tecnici, i quali avranno una definizione specifica per comprenderne al meglio il loro significato.\\
Tutti i termini inclusi nel Glossario, vengono segnalati all'interno del documento Norme di Progetto con una G a pedice.

\subsection{Riferimenti}

\subsubsection{Riferimenti normativi:}\:
\begin{itemize}
	\item Capitolato C1 “CAPTCHA: umano o sovrumano?”
	\url{https://www.math.unipd.it/~tullio/IS-1/2022/Progetto/C1.pdf}
\end{itemize}

\subsubsection{Riferimenti informativi:}\:
\begin{itemize}
	\item Processi di ciclo di vita - Materiale didattico del corso di Ingegneria del Software: \url{https://www.math.unipd.it/~tullio/IS-1/2022/Dispense/T02.pdf};
	\item Il ciclo di vita del Software - Materiale didattico del corso di Ingegneria del Software: \url{https://www.math.unipd.it/~tullio/IS-1/2022/Dispense/T03.pdf};
	\item Gestione di progetto - Materiale didattico del corso di Ingegneria del Software: \url{https://www.math.unipd.it/~tullio/IS-1/2022/Dispense/T04.pdf};
	\item \url{https://it.wikipedia.org/wiki/ISO/IEC_12207};
	\item Approfondimento standard ISO/IEC\textsubscript{G} 12207:  \url{https://www.math.unipd.it/~tullio/IS-1/2009/Approfondimenti/ISO_12207-1995.pdf};
	\item Qualità di prodotto\textsubscript{G} - Materiale didattico del corso di Ingegneria del Software: \url{https://www.math.unipd.it/~tullio/IS-1/2022/Dispense/T08.pdf};
	\item Qualità di processo\textsubscript{G} - Materiale didattico del corso di Ingegneria del Software: \url{https://www.math.unipd.it/~tullio/IS-1/2022/Dispense/T09.pdf};
	\item Standard SQuaRE: \url{http://www.iso25000.it/styled/};
	\item Standard SPICE: \url{https://it.frwiki.wiki/wiki/ISO/CEI_15504};
	\item Regolamento del progetto didattico – Materiale didattico del corso di Ingegneria del Software::
	\url{https://www.math.unipd.it/~tullio/IS-1/2022/Dispense/PD02.pdf};
\end{itemize}
