\section{Introduzione}

\subsection{Scopo del documento}
Questo documento ha come obiettivo il fissare gli standard che permetteranno al gruppo \textit{Catch Em All} di garantire qualità al prodotto\textsubscript{G} e ai processi durante l'intera durata del progetto. Verranno quindi definiti metodi di verifica e validazione\textsubscript{G} continui che permetteranno al gruppo di agire in modo rapido e incisivo nel momento in cui si dovranno fare delle correzioni su eventuali errori o andamenti indesiderati. Questo allo scopo di sprecare meno risorse possibili e produrre un prodotto che sia facilmente mantenibile. 

\subsection{Scopo del prodotto}
Dal proponente Zucchetti S.p.A. viene evidenziato, nel capitolato da loro proposto, una criticità negli attuali sistemi di sicurezza sulla rilevazione dei bot rispetto agli esseri umani. Oggi giorno il meccanismo più utilizzato per risolvere questo problema è il test CAPTCHA.\\
Un bot non è altro che una procedura automatizzata che, in questo caso, ha fini malevoli, come per esempio:
\begin{itemize}
	\item Registrazione presso siti web;
	\item Creazione di spam;
	\item Violare sistemi di sicurezza.
\end{itemize}
I bot, grazie alle nuove tecnologie sviluppate con sistemi che utilizzano principalmente l'intelligenza artificiale, riescono a svolgere compiti che fino a poco tempo fa venivano considerati impossibili da svolgere per una macchina.\\
Ciò evidenzia che i CAPTCHA attuali risultano sempre più obsoleti, non andando a individuare correttamente tutti i bot, se non quasi nessuno.\\
Un'altra criticità individuata dal proponente è il sistema di classificazione delle immagini che sta effettuando Google grazie al proprio reCAPTCHA, che attualmente è il sistema più diffuso.\\
Questa criticità nasce dal beneficio che questa big tech ottiene dall'interazione degli utenti nel risolvere le task proposte, che portano alla creazione di enormi dataset di immagini classificate che possono essere utilizzate per l'apprendimento dei propri sistemi di machine learning o vendibili a terzi.\\
Il capitolato C1 richiede di sviluppare una applicazione web costituita da una pagina di login provvista di un sistema CAPTCHA che sia in grado di distinguere un utente umano da un bot.\\

\subsection{Glossario}
Per evitare ambiguità relative al linguaggio utilizzato nei documenti prodotti, viene fornito il documento \textbf{Glossario v1.0.0}. Qui vi sono contenuti tutti i termini specifici al dominio del problema, i quali avranno una definizione che servirà per comprenderne al meglio il loro significato. Ogni termine che avrà un riferimento al glossario dovrà avere una \textit{G} come pedice. 

\subsection{Riferimenti}

\subsubsection{Riferimenti normativi}:
\begin{itemize}
	\item Capitolato C1 “CAPTCHA: umano o sovrumano?”
	\url{https://www.math.unipd.it/~tullio/IS-1/2022/Progetto/C1.pdf}
\end{itemize}

\subsubsection{Riferimenti informativi}:
\begin{itemize}
	\item Processi di ciclo di vita - Materiale didattico del corso di Ingegneria del Software: \url{https://www.math.unipd.it/~tullio/IS-1/2022/Dispense/T02.pdf};
	\item Il ciclo di vita del Software - Materiale didattico del corso di Ingegneria del Software: \url{https://www.math.unipd.it/~tullio/IS-1/2022/Dispense/T03.pdf};
	\item Gestione di progetto - Materiale didattico del corso di Ingegneria del Software: \url{https://www.math.unipd.it/~tullio/IS-1/2022/Dispense/T04.pdf};
	\item \url{https://it.wikipedia.org/wiki/ISO/IEC_12207};
	\item Approfondimento standard ISO/IEC 12207:  \url{https://www.math.unipd.it/~tullio/IS-1/2009/Approfondimenti/ISO_12207-1995.pdf};
	\item Qualità di prodotto - Materiale didattico del corso di Ingegneria del Software: \url{https://www.math.unipd.it/~tullio/IS-1/2022/Dispense/T08.pdf};
	\item Qualità di processo - Materiale didattico del corso di Ingegneria del Software: \url{https://www.math.unipd.it/~tullio/IS-1/2022/Dispense/T09.pdf};
	\item Standard SQuaRE: \url{http://www.iso25000.it/styled/};
	\item Standard SPICE: \url{https://it.frwiki.wiki/wiki/ISO/CEI_15504};
	\item Regolamento del progetto didattico – Materiale didattico del corso di Ingegneria del Software::
	\url{https://www.math.unipd.it/~tullio/IS-1/2022/Dispense/PD02.pdf};
\end{itemize}
