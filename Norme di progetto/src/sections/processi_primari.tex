\section{Processi primari}
\subsection{Acquisizione}
\textit{Zucchetti S.p.A.} richiede la realizzazione di un progetto creativo riguardante lo sviluppo di un sistema CAPTCHA\textsubscript{G} attraverso l'esposizione della lettera di presentazione \textit{"CAPTCHA: Umano o Sovrumano?"} in data 18 ottobre 2022.\\
Successivamente alla presentazione dei capitolati il gruppo \textit{CatchEmAll} si riunisce per valutare le proposte e le opinioni dei componenti del team attraverso un processo di valutazione, inizialmente generico poi specifico, riassunto nella sezione \textit{Motivazione scelta capitolato} del documento \href{https://github.com/catchEmAll-SWE/catchEmAll-Docs/blob/main/Assegnazione appalti/LetteraCandidatura.pdf}{Lettera di candidatura}. Da queste discussioni emerge una preferenza per il progetto proposto dal referente Dr. Gregorio Piccoli. 
A seguito di questo viene organizzata una riunione con il proponente con l'obiettivo di approfondire e consolidare le richieste del capitolato. \\ 
In data 28 ottobre 2022 il gruppo \textit{Catch em All} si candida a prendere in carico il progetto attraverso la \href{https://github.com/catchEmAll-SWE/Docs/blob/main/Assegnazione%20appalti/lettera_candidatura_v.1.0.0.pdf}{Lettera di candidatura}.\\
Viene infine confermata l'assegnazione dell'appalto da parte del committente in data 04 novembre 2022.


\subsection{Fornitura}
\subsubsection{Scopo}
Il processo di fornitura richiederà al gruppo di definire le norme che dovranno essere rispettate per poter diventare un adeguato fornitore dell'azienda proponente \textit{Zucchetti S.p.A.} e dei committenti Prof. Tullio Vardanega e Prof. Riccardo Cardin. Di conseguenza verranno illustrati i prodotti e documenti che dovranno essere forniti per rispettare i requisiti\textsubscript{G} concordati.
\subsubsection{Rapporto con il proponente}
Durante il corso del progetto il gruppo ha intenzione di instaurare un rapporto di collaborazione con il proponente Dr. Gregorio Piccoli in modo da:
\begin{itemize}
	\item Approfondire gli aspetti chiave del progetto per far fronte ai suoi bisogni;
	\item Chiarire i vari dubbi che emergeranno durante il progetto;
	\item Definire i requisiti\textsubscript{G} e vincoli da rispettare;
	\item Definire una stima dei costi, di tempo e denaro per la durata del progetto;
	\item Garantire che il prodotto soddisfi le richieste, accordandosi sulla qualifica di questo.
\end{itemize} 

\subsubsection{Prodotti e documenti forniti}
\subsubsubsection{Analisi\_dei\_requisiti\_v.2.0.0}
Questo documento, stilato dagli analisti del gruppo, contiene tutti i requisiti\textsubscript{G} e casi d'uso\textsubscript{G} individuati per il progetto. I requisiti sono ottenuti dal documento di presentazione del capitolato e in seguito integrati sia attraverso discussioni tra i membri del gruppo, sia organizzando incontri con il proponente. 
\subsubsubsection{Piano\_di\_progetto\_v.2.0.0}
\paragraph {Scopo}\mbox{}\\
Questo documento stilato dal responsabile di progetto servirà ad individuare ed organizzare i vari periodi del progetto. Esso conterrà preventivi temporali e costi per ognuna di tali peridi, compiendo un'analisi dei rischi che si possono incontrare durante il corso del progetto.
\paragraph {Analisi dei rischi}\mbox{}\\
In questa sezione si analizzano i diversi rischi in cui il team può incombere durante la durata del progetto.
Ogni rischio appartiene ad una specifica categoria, ovvero:
\begin{itemize}
	\item Rischi personali;
	\item Rischi tecnologici;
	\item Rischi organizzativi.
\end{itemize}
Ogni rischio è inoltre composto da:
\begin{itemize}
	\item Nome;
	\item Descrizione;
	\item Identificazione;
	\item Precauzioni;
	\item Pericolosità;
	\item Stima di manifestazione;
	\item Conseguenze;
	\item Piano di contingenza.
\end{itemize}
\paragraph {Modello di sviluppo}\mbox{}\\
In questa sezione viene specificato il modello di sviluppo che il team ha deciso di adottare, in questo caso il \textit{modello AGILE\textsubscript{G}}.
\paragraph {Pianificazione}\mbox{}\\
In questa sezione sono contenute le pianificazioni temporali dei periodi in cui il responsabile di progetto ha deciso di suddividere taluni.
Ogni periodo è contraddistinto da:
\begin{itemize}
	\item Nome identificativo;
	\item Descrizione;
	\item Periodo;
	\item Precondizioni;
	\item Postcondizioni;
	\item Attività;
	\item Ruoli attivi.
\end{itemize}
Ogni periodo è inoltre suddiviso in altri vari periodi temporali per raggruppare al meglio le diverse attività che lo compongono.\\
Infine ogni periodo possiede un proprio diagramma di Gantt\textsubscript{G}.
\begin{figure}[h!]
	\centering
	\includegraphics[width=15cm]{img/4_analisi.png}
	\caption{Esempio di diagramma di Gantt\textsubscript{G}}
\end{figure}
\paragraph {Preventivo}\mbox{}\\
In questa sezione sono contenuti i preventivi sulla distribuzione oraria del lavoro assegnato ad ogni periodo. Inoltre viene preventivato il costo di ognuna ed il costo totale del progetto. Ogni preventivo sarà composto da:
\begin{itemize}
	\item Due tabelle che indicano le ore e i costi necessari per lo svolgimento del periodo;
	\item Un istogramma che illustra come sono state distribuite le ore fra i vari membri del gruppo;
	\item Un grafico a torta che mostra quanto ogni ruolo abbia inciso nel determinato periodo.
\end{itemize}
\paragraph {Consuntivo}\mbox{}\\
In questa sezione vengono indicati il numero di ore di lavoro impiegate e i relativi costi effettivi di ogni periodo. Questi vengono poi relazionati con i preventivi fatti nella sezione \textit{preventivo}.

\subsubsubsection{Piano\_di\_qualifica\_v.2.0.0}
\paragraph {Descrizione}\mbox{}\\
Questo documento, stilato dai membri con il ruolo di analista e di verificatore, contiene i vari obiettivi e metriche che permettono di garantire la qualità della verifica\textsubscript{G} e della validazione\textsubscript{G} dei processi e dei prodotti del progetto.
\paragraph {Struttura documento}\mbox{}\\
Questo documento è suddiviso in:
\begin{itemize}
	\item \textbf{Obiettivi e metriche di qualità di}:
	\begin{itemize}
		\item \textbf{Processo}: contiene i vari obiettivi generici e specifici e le metriche correlate ad essi che permettono di valutare la qualità di un processo;
		\item \textbf{Prodotto}: contiene i vari obiettivi e le metriche correlate ad essi che permettono di valutare la qualità di un prodotto.
	\end{itemize}
	\item \textbf{Specifiche dei test}: vengono definiti i vari test che dovranno essere eseguiti, ovvero:
	\begin{itemize}
		\item Test di unità;
		\item Test di integrazione;
		\item Test di sistema;
		\item Test di regressione;
		\item Test di collaudo.
	\end{itemize}
	\item \textbf{Resoconto delle attività di verifica}: vengono illustrati i risultati ottenuti rispetto alle metriche scelte;
	\item \textbf{Risultati dei test}: vengono inseriti i vari risultati ottenuti dai test definiti nella sezione \textit{Specifiche dei test};
	\item \textbf{Valutazione per il miglioramento}: sono inserite le varie osservazioni e valutazioni fatte dal gruppo sia per poter migliorare l'efficacia e l'efficienza delle varie attività di verifica\textsubscript{G} future, che per migliorare i risultati ottenuti nei resoconti delle verifiche.
\end{itemize} 
\paragraph {Struttura obiettivi}\mbox{}\\
Ogni obiettivo sarà contrassegnato da un codice univoco così composto:
\begin{center}
	\verb|OQ<<Tipo di obiettivo>><<ID>>|
\end{center}
Dove:
\begin{itemize}
	\item \verb|OQ| sta per obiettivo di qualità;
	\item \verb|<<Tipo di obiettivo>>| identifica se è di processo o prodotto (PC-PD);
	\item \verb|<<ID>>| è un contatore correlato al tipo di obiettivo.
\end{itemize}
\paragraph {Struttura metriche}\mbox{}\\
Ogni metrica sarà contrassegnata da un codice univoco così composto:
\begin{center}
	\verb|MQ<<Tipo di metrica>><<ID>>|
\end{center}
Dove:
\begin{itemize}
	\item \verb|MQ| sta per metrica di qualità;
	\item \verb|<<Tipo di metrica>>| identifica se è di processo o prodotto (PC-PD);
	\item \verb|<<ID>>| è un contatore correlato al tipo di metrica.
\end{itemize}

\subsubsubsection{Specifica\_architetturale\_v.1.0.0}
\paragraph {Descrizione}\mbox{}\\
Questo documento, stilato dai membri con il ruolo di progettista, conterrà informazioni sulle tecnologie scelte per la realizzazione del prodotto software, assieme all'architettura del sistema e le scelte progettuali effettuate.
\paragraph {Struttura documento}\mbox{}\\
Questo documento è suddiviso in:
\begin{itemize}
	\item \textbf{Tecnologie coinvolte}:
	\begin{itemize}
		\item \textbf{Tecnologie per la codifica}: contiene informazioni sui linguaggi, strumenti, framework adottati;
        \item \textbf{Strumenti per l'analisi del codice}.
    \end{itemize}
    \item \textbf{Architettura}:
    \begin{itemize}
		\item \textbf{Diagrammi delle classi}: diagrammi che descrivono le classi progettate, con le proprietà di ciascuna classe e le relazioni tra le classi;
        \item \textbf{Architettura di dettaglio}.
    \end{itemize}
\end{itemize}

\subsubsubsection{Manuale\_utente\_v.1.0.0}
\paragraph {Descrizione}\mbox{}\\
Questo documento, stilato dai membri con il ruolo di amministratore, conterrà le istruzioni utili per l'utilizzo del prodotto software, in modo che l'utente abbia una fonte di informazioni sempre consultabile in caso di necessità.
\paragraph {Struttura documento}\mbox{}\\
Questo documento è suddiviso in:
\begin{itemize}
	\item \textbf{Requisiti}: specifica i requisiti minimi per poter usufruire del prodotto software;
    \item \textbf{Installazione}: contiene brevi indicazioni per installare il prodotto software;
    \item \textbf{Istruzioni per l'uso}: contiene indicazioni chiare e schematiche per l'utilizzo del prodotto software.
\end{itemize}

\subsubsubsection{Proof of Concept}
Un esempio di software che va ad analizzare alcune sezioni critiche per lo sviluppo del progetto, individuate in seguito ad un'analisi del gruppo. Questo software permetterà di determinare la fattibilità pratica e l'applicabilità di alcuni concetti necessari per la progettazione e codifica del prodotto finale.

\subsubsubsection{Prodotto software}
Sviluppato successivamente al \textit{Poc\textsubscript{G}}, il prodotto software conterrà le funzionalità descritte nel documento \textit{Analisi dei requisiti v 2.0.0}, e sarà consegato come prodotto software finale.

\subsection{Sviluppo}
    \subsubsection{Scopo}
    L'obiettivo del processo di sviluppo è definire le attività che il gruppo deve eseguire per realizzare il prodotto finale richiesto dal proponente.
    \subsubsection{Analisi dei requisiti}
    \subsubsubsection{Scopo}
    Lo scopo di questa attività è quello di stilare tutti i requisiti\textsubscript{G} che saranno necessari per il successivo periodo di progettazione e quindi per lo sviluppo di un prodotto che risponda in maniera completa ai bisogni del proponente. \\
    I requisiti individuati dovranno quindi:
    \begin{itemize}
    	\item Fornire dei riferimenti affidabili e precisi per il periodo di progettazione;
    	\item Fissare le funzionalità che il prodotto finale dovrà avere;
    	\item Aiutare a definire dei test precisi e dettagliati per agevolare le verifiche future.
    \end{itemize}
	\subsubsubsection{Contenuti}
    Il documento stilato dovrà contenere:
    \begin{itemize}
    	\item Una descrizione generale del prodotto;
    	\item La scelta delle tecnologie da utilizzare;
    	\item L'analisi dettagliata dei casi d'uso\textsubscript{G};
    	\item I requisiti\textsubscript{G} individuati tramite:
        \begin{itemize}
            \item Documento di presentazione del capitolato;
            \item Confronti tra i membri del gruppo;
            \item Incontri con il proponente.
        \end{itemize}
    \end{itemize} 
    \subsubsubsection{Struttura dei casi d'uso\textsubscript{G}}
    Ogni caso d'uso\textsubscript{G} è identificato utilizzando la seguente convenzione di nomenclatura:
    \begin{center}
		\verb|UC<<ID_CasoBase>>.<<ID_SottoCaso>>|
    \end{center}
    Dove:
    \begin{itemize}
    	\item \verb|<<ID>>| identifica l'use case;
    	\item \verb|<<ID_SottoCaso>>| identifica eventuali sottocasi.
    \end{itemize}
	Ogni caso d'uso\textsubscript{G} è composto inoltre da:
	\begin{itemize}
		\item Descrizione: una breve descrizione dell'attività rappresentata dal caso d'uso\textsubscript{G};
		\item Attori\textsubscript{G}: entità esterne al sistema che interagiscono con esso. Ne esistono di due tipologie:
		\begin{itemize}
			\item Primario: interagisce con il sistema per raggiungere un obiettivo;
			\item Secondario: aiuta l'attore primario a raggiungere l'obiettivo.
		\end{itemize}
		\item Precondizione: descrive lo stato del sistema prima dell'attività svolta nel caso d'uso\textsubscript{G};
		\item Postcondizione: descrive lo stato del sistema dopo l'attività svolta nel caso d'uso\textsubscript{G};
		\item Scenario principale: elenco che descrive il flusso degli eventi dell'attività rappresentata dal caso d'uso\textsubscript{G};
		\item Scenari alternativi (se presenti): elenco che descrive gli eventi del caso d'uso\textsubscript{G} dopo un imprevisto che lo ha deviato dallo scenario principale;
		\item Scenari inclusi (se presenti): elenco di casi d'uso\textsubscript{G} che svolgono attività necessarie allo svolgimento dello scenario principale;
		\item Generalizzazioni (se presenti): elenco di casi d'uso\textsubscript{G} che generalizzano il caso d'uso\textsubscript{G} principale.
	\end{itemize}
    
    \subsubsubsection{Struttura dei requisiti\textsubscript{G}}
    Ogni requisito è identificato da un codice univoco così composto:
    \begin{center}
		\verb|R<<TIPOLOGIA DI REQUISITO>>-<<ID>>|
    \end{center}
    Dove:
    \begin{itemize}
        \item \verb|<<TIPOLOGIA DI REQUISITO>>| identifica una classe tra le seguenti:
        \begin{itemize}
            \item Funzionale \{F\};
            \item Qualità \{Q\};
            \item Vincolo \{V\};
            \item Prestazionale \{P\}.
        \end{itemize}
        \item \verb|<<ID>>| identifica numericamente il requisito nella classe di appartenenza.
    \end{itemize}
    Nel documento i requisiti vengono raggruppati per categoria specificandone:
    \begin{itemize}
        \item Grado di obbligatorietà;
        \item Descrizione;
        \item Fonti, le quali possono essere:
        \begin{itemize}
        	\item Il capitolato d'appalto;
        	\item Verbali interni;
        	\item Verbali esterni;
        	\item I casi d'uso\textsubscript{G} identificati.
        \end{itemize}
    \end{itemize}
	
	
	\subsubsection{Progettazione}
	\subsubsubsection{Scopo e descrizione}
	Lo scopo di questa attività è quello di individuare le varie caratteristiche che comporranno il prodotto richiesto dal proponente, avendo come riferimento i requisiti\textsubscript{G} e casi d'uso\textsubscript{G} definiti nel documento \textit{Analisi\_dei\_requisiti v 2.0.0}. Le varie caratteristiche verranno poi messe insieme per costruire una singola soluzione che rispetti i vari obiettivi di qualità del prodotto.
	I vari periodi in cui sarà svolta l'attività di progettazione sono:
	\begin{itemize}
		\item \textbf{Proof of concept};
		\item \textbf{Progettazione architetturale};
		\item \textbf{Progettazione di dettaglio}.
	\end{itemize}
	\subsubsubsection{Proof of concept}
	\paragraph {Scopo}\mbox{}\\
	In questo periodo viene prodotto un software esempio, il quale sarà anche la technology baseline\textsubscript{G} del prodotto finale. Questo andrà ad analizzare alcune sezioni critiche per lo sviluppo del progetto e servirà ad agevolare le successive scelte di progettazione del gruppo, aiutando a determinare la fattibilità e l'applicabilità di alcune scelte analizzate.
	\paragraph {Suddivisione periodi}\mbox{}\\
	Questo periodo è diviso in due:
	\begin{itemize}
		\item Periodo nel quale vengono identificati i requisiti\textsubscript{G} del \textit{Poc\textsubscript{G}} e le tecnologie necessarie a svilupparlo, oltre che lo studio di queste ultime;
		\item Periodo di produzione del \textit{Poc\textsubscript{G}}.
	\end{itemize}
	\subsubsubsection{Progettazione architetturale}
	\paragraph {Scopo}\mbox{}\\
	Lo scopo di questo periodo è il raffinamento della technology baseline\textsubscript{G} definita nel periodo di \textit{Proof of Concept}, e discute ad alto livello l'architettura del prodotto e delle sue componenti. Le scelte che il gruppo effettua in questo periodo riguarderanno la struttura complessiva del sistema. Queste ne influenzeranno varie caratteristiche qualitative come per esempio l'efficienza, l'estensibilità e la manutenibilità.
	\subsubsubsection{Progettazione di dettaglio}
	\paragraph {Scopo}\mbox{}\\
	Lo scopo di questo periodo è definire le specifiche di dettaglio dell’architettura del prodotto e di tutte le sue componenti, scomposte in unità. Queste saranno correlate a diagrammi UML\textsubscript{G} che le descriveranno e ai test di verifica\textsubscript{G} per la qualità, i quali saranno indicati nel documento \textit{Piano\_di\_qualifica v2.0.0}. Tali informazioni costituiranno la Product Baseline, la quale conterrà:
	\begin{itemize}
		\item Desing patterns utilizzati;
		\item Tecnologie utilizzate;
		\item Definizione delle classi;
		\item Tracciamento dei componenti;
		\item Diagrammi UML\textsubscript{G}:
		\begin{itemize}
			\item Diagrammi delle attività;
			\item Diagrammi delle classi.
		\end{itemize}
		\item Test di integrazione.
	\end{itemize}
	\subsubsection{Codifica}
	\subsubsubsection{Scopo e descrizione}
	Il periodo di \textit{Codifica} è assegnato ai membri con il ruolo di programmatore, i quali dovranno realizzare il prodotto software richiesto dal proponente utilizzando ciò che i progettisti hanno definito nel periodo di \textit{Progettazione}.\\
	Per garantire un prodotto adeguato, il codice dovrà essere verificato in modo che rispetti le metriche che garantiscono gli obiettivi di qualità definiti nel documento \textit{Piano\_di\_qualifica v 2.0.0}.
	\subsubsubsection{Sintassi}
	No:
    \begin{itemize}
        \item Chiarezza dei nomi: i nomi di classi, metodi e variabili devono avere un nome che identifica il loro compito in maniera intuitiva;
        \item Abbreviazioni nei nomi: sono ammesse parole abbreviate solo se comunemente note, per esempio \textit{Img} è una valida abbreviazione di \textit{Image};
		\item Classi: il nome delle classi deve essere scritto in stile \textit{PascalCase}, ovvero iniziare con una lettera maiuscola e, se il nome contiene più parole, tutte le parole inizieranno con una lettera maiuscola;
		\item Metodi: il nome dei metodi deve essere scritto in stile \textit{camelCase}, ovvero iniziare con una lettera minuscola e, se il nome contiene più parole, tutte le parole inizieranno con una lettera maiuscola;
		\item Variabili: il nome delle variabili deve essere scritto in stile \textit{snake\_case}, ovvero iniziare con una lettera minuscola e, se il nome contiene più parole, tutte le parole inizieranno con una lettera minuscola e saranno separate dal simbolo \textit{\_}.
    \end{itemize}
    
