\section{Glossario Analisi dei Requisiti}

\paragraph{Algoritmo}~\smallskip \\
Un algoritmo è una successione di istruzioni o passi che effettuano operazioni su dei dati in input per produrre dei dati elaborati in output.

\paragraph{API}~\smallskip \\
API è la sigla di "Application Programming Interface". Nel contesto delle API, la parola "applicazione" si riferisce a qualsiasi software con una funzione distinta. Si può pensare all'interfaccia come a un contratto di servizio tra due applicazioni. Questo contratto definisce come queste due parti comunicano tra loro usando richieste e risposte. La rispettiva documentazione dell'API contiene informazioni su come gli sviluppatori devono strutturare tali richieste e risposte. (internet)

\paragraph{Attori}~\smallskip \\
Soggetti che interagiscono con il sistema.

\paragraph{Booleano}~\smallskip \\
Valore matematico che può assumere solo due valori: vero o falso spesso rappresentati come 1 e 0.

\paragraph{Bot}~\smallskip \\
Macchine automatizzate in grado di interagire con il sistema informatico per azioni a scopo malevolo.

\paragraph{Brute force}~\smallskip \\
Un atttacco brute force è un attacco informatico col fine di individuare password di un utente generico con la caratteristica principale di effettuare un numero molto elevato di tentativi al fine di provare tutte le combinazioni di lettere, caratteri speciali e numeri fino all'individuazione della password cercata.

\paragraph{Casi d'uso}~\smallskip \\
Insieme di scenari che si possono incontrare durante l'utilizzo del prodotto.

\paragraph{CSS3}~\smallskip \\
Linguaggio di programmazione utilizzato per modellare l'aspetto grafico di oggetti creati con altri linguaggi di programmazione come HTML.

\paragraph{Copyright}~\smallskip \\
Titolarità del diritto di riproduzione di un'opera letteraria, discografica, cinematografica ecc.

\paragraph{Dataset}~\smallskip \\
Insieme di immagini.

\paragraph{Database}~\smallskip \\
Archivio di dati utilizzato per memorizzare e gestire le informazioni creato generalmente con il linguaggio Sql.

\paragraph{HTML5}~\smallskip \\
Linguaggio di programmazione utilizzato per la creazione di siti web.

\paragraph{Id}~\smallskip \\
Codice identificativo usato molto nei database per contraddistinguere un elemento in maniera univoca.

\paragraph{Honeypot}~\smallskip \\
La traduzione è "barattolo del miele", nel nostro caso si intende una trappola costruita ad hoc ai fini di individuare tutti i bot che iteragiscono con essa rivelando al sistema la loro vera identità automaticamente, poichè un utente umano non potrà mai interagire con questo sistema.

\paragraph{Open source}~\smallskip \\
Software non protetto da copyright e liberamente modificabile dagli utenti.

\paragraph{Python}~\smallskip \\
Python è un linguaggio di programmazione ampiamente utilizzato nelle applicazioni Web, nello sviluppo di software, nella data science e nel machine learning. (internet)

\paragraph{Proof of work}~\smallskip \\
Viene imposto all'utente intenzionato ad autenticarsi di sottoporre il proprio hardware ad effettuare un lavoro, inteso come tempo di elaborazione. Questo lavoro deve essere moderatamente complesso da svolgere e veloce da controllare da parte del fornitore. Tutto questo per scoraggiare attacchi brute-force.

\paragraph{Query}~\smallskip \\
istruzione che permette l'accesso ai dati contenuti in un database attraverso una opportuna ricerca.

\paragraph{Requisiti}~\smallskip \\
Qualità necessarie e richieste per il raggiungimento dello scopo determinato.

\paragraph{Sql}~\smallskip \\
Linguaggio di programmazione utilizzato per la creazione, la modifica e la gestione dei dati dei database.

\paragraph{Task}~\smallskip \\
Compito al fine di raggiungere un obiettivo.