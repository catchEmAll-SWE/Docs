\section{Glossario Norme di Progetto}

\paragraph{GitHub Workflow}~\smallskip \\
Lavoro che permette di eseguire automaticamente delle azioni su un progetto.

\paragraph{JIRA}~\smallskip \\
E' un tool che monitora e gestisce tutti i progetti in sviluppo, personalizzando il flusso di lavoro di chi collabora nella realizzazione di un software.

\paragraph{Issue Tracking System}~\smallskip \\
Sistema di tracciamento delle problematiche che si verificano durante lo sviluppo di un progetto.

\paragraph{Daily scrum meeting}~\smallskip \\
Incontro giornaliero in cui si parla dei problemi riscontrati e delle soluzioni trovate.

\paragraph{Framework Scrum}~\smallskip \\
Framework che permette di gestire il progetto in modo agile.

\paragraph{Repository}~\smallskip \\
Piattaforma software che permette di conservare una significativa quantità di informazioni. Sui dati memorizzati in questi archivi possono essere svolte numerose operazioni di protezione, classificazione, elaborazione dei documenti. In questi sistemi la gestione delle risorse informative è centralizzata e viene realizzata in un ambiente accessibile da più macchine hardware.

\paragraph{Issue}~\smallskip \\
Lista di punti che possono essere usati per tenere traccia di bug, miglioramenti o altre richieste di progetto.

\paragraph{Branch}~\smallskip \\
Letteralmente significa ramo. Utilizzati in Git per l'implementazione di funzionalità tra loro isolate, cioè sviluppate in modo indipendente l'una dall'altra ma a partire dalla medesima radice.

\paragraph{Merge}~\smallskip \\
Funzione avanzata di fusione tra branch in uno nuovo.

\paragraph{script}~\smallskip \\
Programma che viene eseguito da un altro programma, che lo invoca, per eseguire una determinata operazione.

\paragraph{qualità al prodotto}~\smallskip \\
La qualità al prodotto è la capacità di un prodotto di soddisfare le esigenze del cliente.

\paragraph{qualità di processo}~\smallskip \\
La qualità di processo è la capacità di un processo di produrre un prodotto conforme alle specifiche richieste.

\paragraph{verifica}~\smallskip \\
Verifica è l'attività di esaminare un prodotto, un processo o un sistema per determinare se soddisfa i requisiti specificati.

\paragraph{validazione}~\smallskip \\
Validazione è l'attività di esaminare un prodotto, un processo o un sistema per determinare se soddisfa i requisiti specificati.

\paragraph{ISO/IEC}~\smallskip \\
ISO/IEC è un'organizzazione internazionale che si occupa di standardizzare le tecnologie informatiche.

\paragraph{Scrum}~\smallskip \\
Scrum è un framework agile che permette di gestire il progetto in modo agile.

\paragraph{GitHub}~\smallskip \\
GitHub è un servizio di hosting di repository software.

\paragraph{commit}~\smallskip \\
Commit è un comando di Git che permette di salvare le modifiche effettuate in un file.

\paragraph{feature}~\smallskip \\
Feature è un termine che indica una funzionalità che deve essere implementata.

\paragraph{sprint}~\smallskip \\
Sprint è un termine che indica un periodo di tempo in cui si deve implementare una o più feature.

\paragraph{framework}~\smallskip \\
Framework è un'architettura logica di supporto sulla quale un software può essere progettato e realizzato, spesso facilitandone lo sviluppo da parte del programmatore.
