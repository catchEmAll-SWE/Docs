\section{Glossario Norme di Progetto}

\paragraph{reCAPTCHA}~\smallskip \\
E' un servizio gratuito di Google che consente di proteggere i siti web da spam e abusi permettendo di effettuare una distinzione tra gli utenti umani e i bot automatizzati.

\paragraph{GitHub Workflow}~\smallskip \\
Lavoro che permette di eseguire automaticamente delle azioni su un progetto.

\paragraph{JIRA}~\smallskip \\
E' un tool che monitora e gestisce tutti i progetti in sviluppo, personalizzando il flusso di lavoro di chi collabora nella realizzazione di un software.

\paragraph{Issue Tracking System}~\smallskip \\
Sistema di tracciamento delle problematiche che si verificano durante lo sviluppo di un progetto.

\paragraph{Daily scrum meeting}~\smallskip \\
Incontro giornaliero in cui si parla dei problemi riscontrati e delle soluzioni trovate.

\paragraph{Framework Scrum}~\smallskip \\

\paragraph{Repository}~\smallskip \\
Piattaforma software che permette di conservare una significativa quantità di informazioni. Sui dati memorizzati in questi archivi possono essere svolte numerose operazioni di protezione, classificazione, elaborazione dei documenti. In questi sistemi la gestione delle risorse informative è centralizzata e viene realizzata in un ambiente accessibile da più macchine hardware.

\paragraph{Issue}~\smallskip \\
Lista di punti che possono essere usati per tenere traccia di bug, miglioramenti o altre richieste di progetto.

\paragraph{Branch}~\smallskip \\
Letteralmente significa ramo. Utilizzati in Git per l'implementazione di funzionalità tra loro isolate, cioè sviluppate in modo indipendente l'una dall'altra ma a partire dalla medesima radice.

\paragraph{Merge}~\smallskip \\
Funzione avanzata di fusione tra branch in uno nuovo.
