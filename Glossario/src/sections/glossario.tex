\section{Glossario}

\subsection{A...E}

\paragraph{Accoppiamento}~\smallskip \\
Grado di interdipendenza tra componenti del software. Un basso livello di accoppiamento è desiderabile, in quanto indice del fatto che i componenti sono perlopiù indipendenti l'uno dall'altro; il livello di accoppiamento non può tuttavia mai essere nullo in quanto un sistema è, per definizione, un insieme di parti.

\paragraph{Algoritmo}~\smallskip \\
Un algoritmo è una successione di istruzioni o passi che effettuano operazioni su dei dati in input per produrre dei dati elaborati in output.

\paragraph{API}~\smallskip \\
API è la sigla di "Application Programming Interface". Nel contesto delle API, la parola "applicazione" si riferisce a qualsiasi software con una funzione distinta. Si può pensare all'interfaccia come a un contratto di servizio tra due applicazioni. Questo contratto definisce come queste due parti comunicano tra loro usando richieste e risposte. La rispettiva documentazione dell'API contiene informazioni su come gli sviluppatori devono strutturare tali richieste e risposte.

\paragraph{Attori}~\smallskip \\
Soggetti che interagiscono con il sistema.

\paragraph{Booleano}~\smallskip \\
Valore matematico che può assumere solo due valori: vero o falso spesso rappresentati come 1 e 0.

\paragraph{Bot}~\smallskip \\
Macchine automatizzate in grado di interagire con il sistema informatico per azioni a scopo malevolo.

\paragraph{Branch}~\smallskip \\
Letteralmente significa ramo. Utilizzati in Git per l'implementazione di funzionalità tra loro isolate, cioè sviluppate in modo indipendente l'una dall'altra ma a partire dalla medesima radice.

\paragraph{Brute force}~\smallskip \\
Un atttacco brute force è un attacco informatico col fine di individuare password di un utente generico con la caratteristica principale di effettuare un numero molto elevato di tentativi al fine di provare tutte le combinazioni di lettere, caratteri speciali e numeri fino all'individuazione della password cercata.

\paragraph{Capability}~\smallskip \\
Misura l'efficacia e l'efficienza di un determinato processo. Un basso livello di capability significa che il processo viene seguito in maniera opportunistica e frettolosa, il che rende difficile prevederne l'esito e la qualità; al contrario, un alto livello di capability significa il processo viene seguito da tutti in maniera sistematica e disciplinata.

\paragraph{CAPTCHA}~\smallskip \\
È un test fatto per determinare se l'utente sia un umano e non un computer o, più precisamente, un bot.
Esso comprende l'insieme di immagini, il Proof or Work e l'Honeypot.

\paragraph{Casi d'uso}~\smallskip \\
Insieme di scenari che si possono incontrare durante l'utilizzo del prodotto.

\paragraph{Commit}~\smallskip \\
Commit è un comando di Git che permette di salvare le modifiche effettuate in un file.

\paragraph{Complessità ciclomatica}~\smallskip \\
Metrica software utilizzata per valutare la complessità di un programma, misurando il numero di cammini linearmente indipendenti tramite la costruzione di un grafo di controllo del flusso.

\paragraph{Copyright}~\smallskip \\
Titolarità del diritto di riproduzione di un'opera letteraria, discografica, cinematografica ecc.

\paragraph{CSS3}~\smallskip \\
Linguaggio di programmazione utilizzato per modellare l'aspetto grafico di oggetti creati con altri linguaggi di programmazione come HTML.

\paragraph{Daily scrum meeting}~\smallskip \\
Incontro giornaliero in cui si parla dei problemi riscontrati e delle soluzioni trovate.

\paragraph{Database}~\smallskip \\
Archivio di dati utilizzato per memorizzare e gestire le informazioni creato generalmente con il linguaggio Sql.

\paragraph{Dataset}~\smallskip \\
Insieme di immagini.

\paragraph{Design pattern}~\smallskip \\
Può essere definito come "una soluzione progettuale generale ad un problema ricorrente".
Si tratta di una descrizione o modello logico da applicare per la risoluzione di un problema che può 
presentarsi in diverse situazioni durante le fasi di progettazione e sviluppo del software, ancor prima della definizione 
dell'algoritmo risolutivo della parte computazionale. È un approccio spesso efficace nel contenere o ridurre il debito tecnico.

\paragraph{Diagramma di Gantt}~\smallskip \\
Strumento molto utile per rappresentare e visualizzare graficamente le tempistiche e l'avanzamento di un progetto.

\paragraph{Dominio}~\smallskip \\
Contesto operativo.

\subsection{F...J}

\paragraph{Feature}~\smallskip \\
Feature è un termine che indica una funzionalità che deve essere implementata.

\paragraph{Framework}~\smallskip \\
Framework è un'architettura logica di supporto sulla quale un software può essere progettato e realizzato, spesso facilitandone lo sviluppo da parte del programmatore.

\paragraph{Framework Scrum}~\smallskip \\
Framework che permette di gestire il progetto in modo agile.

\paragraph{GitHub}~\smallskip \\
GitHub è un servizio di hosting di repository software.

\paragraph{GitHub Workflow}~\smallskip \\
Lavoro che permette di eseguire automaticamente delle azioni su un progetto.

\paragraph{Honeypot}~\smallskip \\
La traduzione è "barattolo del miele", nel nostro caso si intende una trappola costruita ad hoc ai fini di individuare tutti i bot che iteragiscono con essa rivelando al sistema la loro vera identità automaticamente, poichè un utente umano non potrà mai interagire con questo sistema.

\paragraph{HTML5}~\smallskip \\
Linguaggio di programmazione utilizzato per la creazione di siti web.

\paragraph{Id}~\smallskip \\
Codice identificativo usato molto nei database per contraddistinguere un elemento in maniera univoca.

\paragraph{Intorno}~\smallskip \\
Nel linguaggio matematico, l'intorno di un valore è un insieme di valori "vicini" ad esso.

\paragraph{ISO/IEC}~\smallskip \\
ISO/IEC è un'organizzazione internazionale che si occupa di standardizzare le tecnologie informatiche.

\paragraph{Issue}~\smallskip \\
Lista di punti che possono essere usati per tenere traccia di bug, miglioramenti o altre richieste di progetto.

\paragraph{Issue Tracking System}~\smallskip \\
Sistema di tracciamento delle problematiche che si verificano durante lo sviluppo di un progetto.

\paragraph{JIRA}~\smallskip \\
E' un tool che monitora e gestisce tutti i progetti in sviluppo, personalizzando il flusso di lavoro di chi collabora nella realizzazione di un software.

\subsection{K...O}

\paragraph{Merge}~\smallskip \\
Funzione avanzata di fusione tra branch in uno nuovo.

\paragraph{Milestone}~\smallskip \\
Data del calendario di progetto che indica traguardi importanti durante lo sviluppo di un prodotto.

\paragraph{Modello AGILE}~\smallskip \\
La "metodologia agile" indica un insieme di metodi di sviluppo del software direttamente o indirettamente derivati dai principi del "Manifesto per lo sviluppo agile del software".
Tali metodi di sviluppo del software si basano sulla distribuzione continua di software efficienti creati in modo rapido e iterativo, quindi
consente di adottare un approccio più leggero alla stesura della documentazione software e di integrare le modifiche in qualsiasi fase del ciclo di vita, anziché ostacolarle.

\paragraph{MVP}~\smallskip \\
Il termine si riferisce al “minimum viable product” ossia alla versione iniziale di un prodotto con 
caratteristiche sufficienti da poter essere utilizzato dai primi clienti, e permette, attraverso i feedback raccolti, di raggiungere lo sviluppo del prodotto finale.

\paragraph{NPLF}~\smallskip \\
Metodologia utilizzata per determinare la capability di un processo. Ciascun attributo del processo in esame è valutato secondo una scala a quattro valori, N-P-L-F:
\begin{itemize}
	\item Not achieved (0\% - 15\%);
	\item Partially achieved (16\% - 50\%);
	\item Largely achieved (51\% - 85\%);
    \item Fully achieved (86\% - 100\%).
\end{itemize}

\paragraph{Open source}~\smallskip \\
Software non protetto da copyright e liberamente modificabile dagli utenti.

\subsection{P...T}

\paragraph{PoC}~\smallskip \\
Realizzazione incompleta o abbozzata di un determinato prodotto, allo scopo di provarne la fattibilità o dimostrare la fondatezza di alcuni principi o concetti costituenti.

\paragraph{Product Backlog Refinement}~\smallskip \\
Indica il processo di stima del tempo necessario per i task nel backlog esistente, utilizzando gli story points, 
raffinando i criteri di accettazione per le storie, e dividendo storie più grandi in storie di minore grandezza e complessità.

\paragraph{Proof of work}~\smallskip \\
Viene imposto all'utente intenzionato ad autenticarsi di sottoporre il proprio hardware ad effettuare un lavoro, inteso come tempo di elaborazione. Questo lavoro deve essere moderatamente complesso da svolgere e veloce da controllare da parte del fornitore. Tutto questo per scoraggiare attacchi brute-force.

\paragraph{Python}~\smallskip \\
Python è un linguaggio di programmazione ampiamente utilizzato nelle applicazioni Web, nello sviluppo di software, nella data science e nel machine learning. (internet)

\paragraph{Qualità di processo}~\smallskip \\
La qualità di processo è la capacità di un processo di produrre un prodotto conforme alle specifiche richieste.

\paragraph{Qualità di prodotto}~\smallskip \\
La qualità di prodotto è la capacità di un prodotto di soddisfare le esigenze del cliente.

\paragraph{reCAPTCHA}~\smallskip \\
E' un servizio gratuito di Google che consente di proteggere i siti web da spam e abusi permettendo di effettuare una distinzione tra gli utenti umani e i bot automatizzati.

\paragraph{Repository}~\smallskip \\
Piattaforma software che permette di conservare una significativa quantità di informazioni. Sui dati memorizzati in questi archivi possono essere svolte numerose operazioni di protezione, classificazione, elaborazione dei documenti. In questi sistemi la gestione delle risorse informative è centralizzata e viene realizzata in un ambiente accessibile da più macchine hardware.

\paragraph{Requisiti}~\smallskip \\
Qualità necessarie e richieste per il raggiungimento dello scopo determinato.

\paragraph{Script}~\smallskip \\
Programma che viene eseguito da un altro programma, che lo invoca, per eseguire una determinata operazione.

\paragraph{Scrum}~\smallskip \\
Scrum è un framework agile che permette di gestire il progetto in modo agile.

\paragraph{Spam}~\smallskip \\
Spam indica l'invio, attraverso indirizzi generici non verificati o sconosciuti, nel nostro caso da bot, di dati indesiderati, non richiesti o voluti.

\paragraph{Sprint}~\smallskip \\
Intervallo di tempo fisso ripetibile durante il quale viene creato un prodotto "vendibile" del valore più alto possibile.

\paragraph{Sprint review}~\smallskip \\
Si tratta della revisione delle attività svolte nello sprint trascorso. Durante la revisione viene controllata la correttezza
dei lavoro scovando le incompletezze e gli errori se presenti.

\paragraph{Sql}~\smallskip \\
Linguaggio di programmazione utilizzato per la creazione, la modifica e la gestione dei dati dei database.

\paragraph{Task}~\smallskip \\
Compito al fine di raggiungere un obiettivo.

\paragraph{Technology Baseline}~\smallskip \\
Dimostrare al committente di disporre di librerie, tecnologie e framework utili e neccessari per lo sviluppo di un prodotto.

\subsection{U...Z}

\paragraph{UML}~\smallskip \\
Acronimo di Unified Modeling Language, è stato creato per realizzare un linguaggio di modellazione visivo comune, ricco 
sia nella semantica che nella sintassi, per l'architettura, la progettazione e l'implementazione di sistemi 
software complessi sia dal punto di vista strutturale che comportamentale. (internet)

\paragraph{Unità software}~\smallskip \\
Minimo componente di un programma che ha un funzionamento autonomo.

\paragraph{Validazione}~\smallskip \\
Validazione è l'attività di esaminare un prodotto, un processo o un sistema per determinare se soddisfa i requisiti specificati.

\paragraph{Verifica}~\smallskip \\
Verifica è l'attività di esaminare un prodotto, un processo o un sistema per determinare se soddisfa i requisiti specificati.

\paragraph{Query}~\smallskip \\
istruzione che permette l'accesso ai dati contenuti in un database attraverso una opportuna ricerca.
