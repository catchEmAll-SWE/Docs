\section{Glossario Piano di Qualifica}

\paragraph{Accoppiamento}~\smallskip \\
Grado di interdipendenza tra componenti del software. Un basso livello di accoppiamento è desiderabile, in quanto indice del fatto che i componenti sono perlopiù indipendenti l'uno dall'altro; il livello di accoppiamento non può tuttavia mai essere nullo in quanto un sistema è, per definizione, un insieme di parti.

\paragraph{Capability}~\smallskip \\
Misura l'efficacia e l'efficienza di un determinato processo. Un basso livello di capability significa che il processo viene seguito in maniera opportunistica e frettolosa, il che rende difficile prevederne l'esito e la qualità; al contrario, un alto livello di capability significa il processo viene seguito da tutti in maniera sistematica e disciplinata.

\paragraph{Complessità ciclomatica}~\smallskip \\
Metrica software utilizzata per valutare la complessità di un programma, misurando il numero di cammini linearmente indipendenti tramite la costruzione di un grafo di controllo del flusso.

\paragraph{Dominio}~\smallskip \\
Contesto operativo.

\paragraph{Intorno}~\smallskip \\
Nel linguaggio matematico, l'intorno di un valore è un insieme di valori "vicini" ad esso.

\paragraph{NPLF}~\smallskip \\
Metodologia utilizzata per determinare la capability di un processo. Ciascun attributo del processo in esame è valutato secondo una scala a quattro valori, N-P-L-F:
\begin{itemize}
	\item Not achieved (0\% - 15\%);
	\item Partially achieved (16\% - 50\%);
	\item Largely achieved (51\% - 85\%);
    \item Fully achieved (86\% - 100\%).
\end{itemize}

\paragraph{Unità software}~\smallskip \\
Minimo componente di un programma che ha un funzionamento autonomo.
