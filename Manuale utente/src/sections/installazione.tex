\section{Installazione}
Il prodotto software fornito può essere utilizzato nelle seguenti modalità:
\begin{itemize}
	\item \textbf{Senza installazione}: per l'utente che preferisce utilizzare l'applicazione disponibile online digitando l'indirizzo nella barra di ricerca del proprio browser\textsubscript{G};
	\item \textbf{Reinstallazione}: per l'utente che desidera installare il prodotto software in locale.
\end{itemize}    
L'utente, dopo aver scelto la modalità di utilizzo desiderata, potrà consultare la relativa sezione per ulteriori indicazioni.

\subsection{Utilizzo dell'applicazione senza installazione}
Di seguito vengono elencati i passaggi per avviare l'applicazione senza doverla installare:
\begin{itemize}
	\item Avviare uno dei browser\textsubscript{G} compatibili con l'applicazione;
	\item Digitare il seguente indirizzo: \textbf{\href{https://swe.gdr00.it}}{https://swe.gdr00.it} .
\end{itemize}    

\subsection{Reinstallazione del prodotto software}
Di seguito vengono elencati i passaggi per l'installazione dell'applicazione in un server\textsubscript{G} locale. Si consiglia di seguire i passaggi di questa sezione della guida nell'ordine in cui vengono presentati.

\subsubsection{Installare il sistema operativo\textsubscript{G}}
Sul computer scelto come server\textsubscript{G} dell'applicazione installare il seguente sistema operativo\textsubscript{G}:
\begin{verbatim}
Ubuntu 20.04.6 LTS (Focal Fossa)
\end{verbatim} \\

\subsubsection{Scaricare i pacchetti\textsubscript{G}}
Eseguire il seguente comando per scaricare i pacchetti\textsubscript{G} per l'utilizzo di PHP\textsubscript{G}, Nginx\textsubscript{G}, sqlite3\textsubscript{G}, FPM\textsubscript{G}, cURL\textsubscript{G}, mbstring\textsubscript{G}:
\begin{verbatim}
sudo apt update && sudo apt full-upgrade -y && 
apt install nginx php php-fpm php-sqlite3 php-curl php-mbstring -y && 
systemctl enable nginx.service && systemctl enable php-fpm php-dom
\end{verbatim}

\subsubsection{Verifica delle estensioni\textsubscript{G}}
Verificare se le seguenti estensioni\textsubscript{G} sono abilitate, e in caso contrario abilitarle:
\begin{itemize}
	\item OpenSSL PHP Extension;
	\item PDO PHP Extension;
 \item Mbstring PHP Extension;
 \item Tokenizer PHP Extension;
 \item XML PHP Extension;
 \item Ctype PHP Extension;
 \item JSON PHP Extension;
 \item BCMath PHP Extension.
\end{itemize}   

\subsubsection{Clonare il repository\textsubscript{G}}
Spostarsi nella cartella \textit{/var/www} del web server\textsubscript{G}, in modo che poi esso abbia tutti i permessi necessari, e poi clonare il repository\textsubscript{G} da GitHub\textsubscript{G}:
\begin{verbatim}
cd /var/www
git clone https://github.com/catchEmAll-SWE/dev.git
\end{verbatim}

\subsubsection{Installare il gestore di pacchetti\textsubscript{G} PHP\textsubscript{G}}
\begin{verbatim}
php -r "copy('https://getcomposer.org/installer', 'composer-setup.php');"
sudo php composer-setup.php --install-dir=/bin --filename=composer
composer global require laravel/installer
echo 'export PATH="$PATH:$HOME/.config/composer/vendor/bin"' >> ~/.bashrc
source ~/.bashrc
cd /dev/captcha
cp .env.example .env
composer install
composer require --dev knuckleswtf/scribe
php artisan key:generate
php artisan scribe:generate
\end{verbatim}

\subsubsection{Impostazione dei permessi}
Spostarsi nella cartella \textit{cd/dev/captcha} ed eseguire i seguenti comandi, in modo che Nginx\textsubscript{G} abbia i permessi per modificare i file nelle cartelle dedicate. Attenzione: la parte \textbf{username} dei comandi \textbf{chown} va sostituita con l'effective userID\textsubscript{G}. Se non lo si conosce, lo si può ottenere con il comando \textbf{whoami}. 
\begin{verbatim}
cd/dev/captcha

sudo chmod -R 775 storage
sudo chmod -R 775 bootstrap/cache

sudo chown -R username:www-data /full/path/to/dev/captcha/bootstrap/cache 
sudo chown -R username:www-data /full/path/to/dev/captcha/storage
\end{verbatim}

\subsubsection{Modificare il file di configurazione di Nginx\textsubscript{G}}
Aprire in modifica il file \textbf{nginx.conf} che si trova al percorso \textit{/etc/nginx/nginx.conf} e impostarlo come segue:
\begin{verbatim}
#user www-data;
worker_processes  1;

#error_log  logs/error.log;
#error_log  logs/error.log  notice;
#error_log  logs/error.log  info;

#pid        logs/nginx.pid;


events {
    worker_connections  1024;
}


http {
    charset utf-8;
    sendfile on;
    tcp_nopush on;
    tcp_nodelay on;
    server_tokens off;
    log_not_found off;
    types_hash_max_size 4096;
    client_max_body_size 16M;

    # MIME
    include mime.types;
    default_type application/octet-stream;

    # logging
    access_log /var/log/nginx/access.log;
    error_log /var/log/nginx/error.log warn;

    # load configs
    include /etc/nginx/conf.d/*.conf;
    include /etc/nginx/sites-enabled/*;

    #log_format  main  '$remote_addr - $remote_user [$time_local] "$request" '
    #                  '$status $body_bytes_sent "$http_referer" '
    #                  '"$http_user_agent" "$http_x_forwarded_for"';

    #access_log  logs/access.log  main;

    #keepalive_timeout  0;
    keepalive_timeout  65;

    #gzip  on;

    server_names_hash_bucket_size 64;


    # another virtual host using mix of IP-, name-, and port-based configuration
    #
    #server {
    #    listen       8000;
    #    listen       somename:8080;
    #    server_name  somename  alias  another.alias;

    #    location / {
    #        root   html;
    #        index  index.html index.htm;
    #    }
    #}


    # HTTPS server
    #
    #server {
    #    listen       443 ssl;
    #    server_name  localhost;

    #    ssl_certificate      cert.pem;
    #    ssl_certificate_key  cert.key;

    #    ssl_session_cache    shared:SSL:1m;
    #    ssl_session_timeout  5m;

    #    ssl_ciphers  HIGH:!aNULL:!MD5;
    #    ssl_prefer_server_ciphers  on;

    #    location / {
    #        root   html;
    #        index  index.html index.htm;
    #    }
    #}

}
\end{verbatim}

\subsubsection{Creare il file captcha.test}
Eseguire il seguente comando:
\begin{verbatim}
sudo touch /etc/nginx/sites-available/captcha.test 
\end{verbatim}

\subsubsection{Ottenere il percorso assoluto\textsubscript{G}}
Eseguire i seguenti comandi, e salvare l'output del comando \textbf{pwd} che sarà utile nel paragrafo successivo \textbf{3.2.10}:
\begin{verbatim}
cd/dev/captcha/public
pwd
\end{verbatim}

\subsubsection{Modificare il file captcha.test}
Aprire in modifica il file \textbf{captcha.test} che si trova al percorso \textit{/etc/nginx/sites-available/captcha.test } e impostarlo come segue. Attenzione: il percorso indicato alla voce \textbf{root} deve essere l'output del comando \textbf{pwd} ottenuto al precedente paragrafo \textbf{3.2.9}.
\begin{verbatim}
server {
    listen 80;
    listen [::]:80;
    server_name captcha.test;
    root /path/to/web/root/dev/captcha/public;
 
    add_header X-Frame-Options "SAMEORIGIN";
    add_header X-Content-Type-Options "nosniff";
 
    index index.php;
 
    charset utf-8;
 
    location / {
        try_files $uri $uri/ /index.php?$query_string;
    }
 
    location = /favicon.ico { access_log off; log_not_found off; }
    location = /robots.txt  { access_log off; log_not_found off; }
 
    error_page 404 /index.php;
 
    location ~ \.php$ {
        fastcgi_pass unix:/var/run/php-fpm/php-fpm.sock;
        fastcgi_param SCRIPT_FILENAME $realpath_root$fastcgi_script_name;
        include fastcgi_params;
    }
 
    location ~ /\.(?!well-known).* {
        deny all;
    }
}
\end{verbatim}

\subsubsection{Creare il link simbolico\textsubscript{G}}
Eseguire il seguente comando per creare il link simbolico\textsubscript{G} del file \textbf{captcha.test} nella cartella \textbf{sites-enable} usata da Nginx\textsubscript{G}:
\begin{verbatim}
sudo ln -s /etc/nginx/sites-available/captcha.test /etc/nginx/sites-enabled
\end{verbatim}

\subsubsection{Riavviare Nginx\textsubscript{G}}
Ora che la configurazione di Nginx\textsubscript{G} è completata, sarà necessario riavviare il servizio\textsubscript{G}:
\begin{verbatim}
sudo systemctl restart nginx.service
\end{verbatim}

\subsubsection{Configurare il file hosts}
Aprire in modifica il file \textbf{hosts} che si trova al percorso \textit{/etc/hosts} e aggiungere la riga seguente, in modo da poter avviare l'applicazione più intuitivamente:
\begin{verbatim}
127.0.0.1 captcha.test
\end{verbatim}

\subsubsection{Utilizzo dell'applicazione dopo l'installazione}
Se tutti i passaggi sono stati eseguiti correttamente, sarà possibile aprire l'applicazione seguendo questi passaggi:
\begin{itemize}
	\item Avviare uno dei browser\textsubscript{G} compatibili con l'applicazione;
	\item Digitare il seguente indirizzo: \textbf{\href{captcha.test}}{captcha.test} .
\end{itemize}    
