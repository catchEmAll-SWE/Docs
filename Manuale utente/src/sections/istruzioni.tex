\section{Istruzioni per l'uso}
L'applicazione è composta da un'unica pagina web che prevede l'esecuzione di una serie operazioni ordinate da parte dell'utente:
\begin{itemize}
	\item Inserimento delle credenziali e generazione del captcha\textsubscript{G};
	\item Superamento del test captcha\textsubscript{G};
    \item Verifica delle credenziali e del superamento del test.
\end{itemize}    

\subsection{Inserimento delle credenziali e generazione del captcha\textsubscript{G}}
Una volta aperta la pagina web dell'applicazione viene visualizzata la pagina di login, dove l'utente deve:
\begin{itemize}
	\item Inserire l'username nell'apposito campo: \textbf{CatchEmAll};
	\item Inserire la password nell'apposito campo: \textbf{captcha};
    \item Premere il tasto \textbf{Genera captcha}.
\end{itemize}    

\begin{figure}[H]
    \centering
    \includegraphics[scale=0.8]{img/login.png}
    \caption{Form di inserimento credenziali}
\end{figure}

\subsection{Superamento del test captcha\textsubscript{G}}
Il test captcha\textsubscript{G} è composto da due parti visibili all'utente:
\begin{itemize}
	\item Test immagini;
	\item Attesa operazioni in background\textsubscript{G}.
\end{itemize} 

\subsubsection{Test immagini}
Per poter essere riconosciuto come utente umano e non come un bot\textsubscript{G}, l'utente deve superare il test immagini. Questo test è composto da:
\begin{itemize}
    \item Una tipologia di immagini da selezionare;
	\item Una griglia 3x3 su cui sono disposte 9 immagini cliccabili, di cui solo alcune appartengono alla tipologia di immagine da selezionare.
\end{itemize} 

\begin{figure}[H]
    \centering
    \includegraphics[scale=0.6]{src/img/computerlaptop.png}
    \caption{Test immagini}
\end{figure}

\newpage
 
L'utente dovrà selezionare solo le immagini che ritiene facciano parte della tipologia richiesta. 
\begin{itemize}
	\item \textit{Per esempio, nell'immagine sottostante è richiesto di selezionare le immagini appartententi alla classe \textbf{laptop}, e le immagini corrette da selezionare vengono evidenziate in rosso:}
\end{itemize} 

\begin{figure}[H]
    \centering
    \includegraphics[scale=0.6]{src/img/computerlaptopevidenziati.png}
    \caption{Test immagini con soluzione evidenziata}
\end{figure}

\subsubsection{Attesa operazioni in background\textsubscript{G}}
L'utente potrà procedere con il paragrafo seguente solo dopo che l'indicatore di progresso ha raggiunto il 100\%, come indicato nella seguente immagine:
\begin{figure}[H]
    \centering
    \includegraphics[scale=0.8]{img/barra.png}
    \caption{Indicatore di progresso}
\end{figure}    

\begin{itemize}
	\item \textit{Questa è un'operazione che richiede pochi secondi, e normalmente quando l'utente ha terminato la compilazione del test immagini l'avanzamento ha già raggiunto il 100\%.}
\end{itemize} 

\subsection{Verifica delle credenziali e del superamento del test}
Una volta effettuati i passaggi illustrati nei paragrafi precedenti, l'utente potrà procedere con l'autenticazione cliccando sul tasto \textbf{Login}. Se l'autenticazione è andata a buon fine, l'utente visualizzerà un messaggio di successo; altrimenti visualizzerà un messaggio di errore e potrà ritentare il login ripetendo i passaggi.

\begin{figure}[H]
    \centering
    \includegraphics[scale=0.8]{img/tastologin.png}
    \caption{Pulsante di login}
\end{figure}   

\subsection{Problemi comuni}
Di seguito vengono elencati alcuni suggerimenti in merito alle operazioni alle quali prestare particolare attenzione nel caso in cui l'utente non riesca ad autenticarsi sul sistema.
\begin{itemize}
	\item L'utente non ha inserito correttamente le credenziali indicate al paragrafo \textbf{4.1};
    \item L'utente ha compilato erroneamente il test captcha\textsubscript{G} illustrato al paragrafo \textbf{4.2.2}
    \item L'utente non ha aspettato il completamento delle attività in background\textsubscript{G} come indicato al paragrafo \textbf{4.2.2}.
\end{itemize} 