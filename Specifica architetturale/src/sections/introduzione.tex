\section{Introduzione}

\subsection{Scopo del documento}
Questo documento ha lo scopo di servire da linea guida per gli sviluppatori che andranno ad estendere o manutenere il prodotto. 
Di seguito lo sviluppatore troverà nel documento tutte le informazioni riguardanti i linguaggi e le tecnologie utilizzate, l’architettura del sistema e le scelte progettuali effettuate
per il prodotto.

\subsection{Scopo del prodotto}
Dal proponente Zucchetti S.p.A. viene evidenziato, nel capitolato da loro proposto, una criticità negli attuali sistemi di sicurezza sulla rilevazione dei bot\textsubscript{G} rispetto agli esseri umani. Oggi giorno il meccanismo più utilizzato per risolvere questo problema è il test CAPTCHA\textsubscript{G}.\\
Un bot\textsubscript{G} non è altro che una procedura automatizzata che, in questo caso, ha fini malevoli, come per esempio:
\begin{itemize}
	\item Registrazione presso siti web;
	\item Creazione di spam\textsubscript{G};
	\item Violare sistemi di sicurezza.
\end{itemize}
I bot\textsubscript{G}, grazie alle nuove tecnologie sviluppate con sistemi che utilizzano principalmente l'intelligenza artificiale, riescono a svolgere compiti che fino a poco tempo fa venivano considerati impossibili da svolgere per una macchina.\\
Ciò evidenzia che i CAPTCHA\textsubscript{G} attuali risultano sempre più obsoleti, non andando a individuare correttamente tutti i bot\textsubscript{G}, se non quasi nessuno.\\
Un'altra criticità individuata dal proponente è il sistema di classificazione delle immagini che sta effettuando Google grazie al proprio reCAPTCHA\textsubscript{G}, che attualmente è il sistema più diffuso.\\
Questa criticità nasce dal beneficio che questa big tech\textsubscript{G} ottiene dall'interazione degli utenti nel risolvere le task\textsubscript{G} proposte, che portano alla creazione di enormi dataset\textsubscript{G} di immagini classificate che possono essere utilizzate per l'apprendimento dei propri sistemi di machine learning o vendibili a terzi.\\
Il capitolato C1 richiede di sviluppare una applicazione web costituita da una pagina di login provvista di questo sistema di rilevazione in grado di distinguere un utente umano da un bot\textsubscript{G}.\\
L'utente quindi, dopo aver compilato il form in cui inserirà il nome utente e la password, dovrà svolgere una task\textsubscript{G} che sarà il cosiddetto test CAPTCHA\textsubscript{G}.



\subsection{Glossario}
Per evitare ambiguità relative al linguaggio utilizzato nei documenti prodotti, viene fornito il \textbf{Glossario v 1.0.0}. In questo documento sono contenuti tutti i termini tecnici, i quali avranno una definizione specifica per comprenderne al meglio il loro significato.\\
Tutti i termini inclusi nel Glossario, vengono segnalati all'interno del documento Piano di qualifica con una G a pedice.


\subsection{Standard di progetto}
Per lo svolgimento del progetto il gruppo \textit{CatchEmAll} ha scelto di utilizzare come norme di riferimento informativo la serie di standard \textbf{ISO/IEC 25000 SQuaRE} per definire i requisiti\textsubscript{G} e le metriche per valutazione della qualità di un prodotto e lo standard \textbf{ISO/IEC 15504 SPICE} per definire al meglio la qualità e le metriche di un processo.

\subsection{Riferimenti}
\subsubsection{Riferimenti normativi}
Riferimenti normativi utilizzati:
\begin{itemize}
	\item Norme di Progetto v1.0.0;
	\item Capitolato d'appalto C1 \textit{CAPTCHA: Umano o Sovrumano?} : \\
		\url{https://www.math.unipd.it/~tullio/IS-1/2022/Progetto/C1.pdf}.
\end{itemize}
\subsubsection{Riferimenti informativi}
Riferimenti informativi utilizzati:
\begin{itemize}
	\item Diagrammi delle Classi - Materiale didattico del corso di Ingegneria del Software: \\
		\url{https://www.math.unipd.it/~rcardin/swea/2023/Diagrammi\%20delle\%20Classi.pdf};
	\item Software Architecture Patterns - Materiale didattico del corso di Ingegneria del Software: \\
		\url{https://www.math.unipd.it/~rcardin/swea/2022/Software\%20Architecture\%20Patterns.pdf};
	\item Design Pattern\textsubscript{G} Architetturali - Materiale didattico del corso di Ingegneria del Software:
	\begin{itemize}
		\item Dependency Injection: 
		\url{https://www.math.unipd.it/~rcardin/swea/2022/Design\%20Pattern\%20Architetturali\%20-\%20Dependency\%20Injection.pdf};
		\item Model View Controller:
		\url{https://www.math.unipd.it/~rcardin/sweb/2022/L02.pdf}.
	\end{itemize}
	\item Design Pattern\textsubscript{G} Creazionali:\\
		\url{https://www.math.unipd.it/~rcardin/swea/2022/Design\%20Pattern\%20Creazionali.pdf};
	\item Design Pattern\textsubscript{G} Strutturali:\\
		\url{https://www.math.unipd.it/~rcardin/swea/2022/Design\%20Pattern\%20Strutturali.pdf};
	\item Design Pattern\textsubscript{G} Comportamentali:\\
		\url{https://www.math.unipd.it/~rcardin/swea/2021/Design\%20Pattern\%20Comportamentali_4x4.pdf};
	\item SOLID Principles of Object-Oriented Design:\\
		\url{https://www.math.unipd.it/~rcardin/swea/2021/SOLID\%20Principles\%20of\%20Object-Oriented\%20Design_4x4.pdf}.
\end{itemize}

