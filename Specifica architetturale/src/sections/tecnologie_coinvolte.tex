\section{Tecnologie coinvolte}

\subsection{Tecnologie per la codifica}

\subsubsection{Linguaggi}

\begin{center}
\setlength\extrarowheight{5pt}
\rowcolors{2}{gray!10}{gray!40}
\renewcommand\tabularxcolumn[1]{>{\Centering}m{#1}}
\begin{tabularx}{\textwidth}{| c | c | X | X | X |} 
	\hline
	\rowcolor{white}
	\textbf{Tecnologia} & \textbf{Versione} & \textbf{Descrizione}\\
	\hline
	PHP & 8.1 & Linguaggio principale per lo sviluppo del progetto, usato per gestire tutti le componenti dell'API. E' il linguaggio utilizzato dal framework Laravel\\
	\hline
	HTML5 & - & Linguaggio di markup utilizzato per impostare la struttura delle pagine web.\\
	\hline
	CSS & - & Utilizzato per la formattazione e la definizione dello stile delle pagine HTML.\\
	\hline
	JavaScript & - & Utilizzato per implementare il proof of work una volta richiesta la generazione del catpcha e per la gestione dinamica del front-end.\\
	\hline
	Python & 3.10.9 & Utilizzato per creazioni degli script di download e elaborazione delle immagini da Unsplash.\\
	\hline
	\rowcolor{white}
	\caption{Linguaggi utilizzati}
\end{tabularx}
\end{center}

\subsubsection{Strumenti}

\begin{center}
\setlength\extrarowheight{5pt}
\rowcolors{2}{gray!10}{gray!40}
\renewcommand\tabularxcolumn[1]{>{\Centering}m{#1}}
\begin{tabularx}{\textwidth}{| c | c | X | X | X |} 
	\hline
	\rowcolor{white}
	\textbf{Tecnologia} & \textbf{Versione} & \textbf{Descrizione}\\
	\hline
	Composer & 2.5.5 &  Gestore di dipendenze per il linguaggio di programmazione PHP.\\
	\hline
	NodeJS(npm) & 8.1.0 & Consente di installare, gestire e condividere pacchetti e moduli di terze parti per lo sviluppo di applicazioni Node.js.\\
	\hline
	Sqlite & 3 & Utilizzato per salvare le varie informazioni necessarie per la gestione dei captcha in un DB;\\
	\hline
	Scribe & 4.19 & Utilizzato per la generazione della documentazione dell'API.\\
	\rowcolor{white}
	\caption{Strumenti utilizzati}
\end{tabularx}
\end{center}
\newpage
\subsubsection{Framework e librerie}

\begin{center}
\setlength\extrarowheight{5pt}
\rowcolors{2}{gray!10}{gray!40}
\renewcommand\tabularxcolumn[1]{>{\Centering}m{#1}}
\begin{tabularx}{\textwidth}{| c | c | X | X | X |} 
	\hline
	\rowcolor{white}
	\textbf{Tecnologia} & \textbf{Versione} & \textbf{Descrizione}\\
	\hline
	Laravel & 10.10 & Framework open-source scritto in PHP, semplifica lo sviluppo delle applicazioni web offrendo una vasta gamma di funzionalità.\\
	\hline
	Sanctum & 3.2 & Libreria offerta da Laravel che permette ai vari utenti di generare più API token per il loro account. Ogni token può dare accesso a certe funzionalità specifiche.\\
	\hline
	\rowcolor{white}
	\caption{Framework e librerie utilizzati}
\end{tabularx}
\end{center}

\subsection{Strumenti per l’analisi del codice}

\begin{center}
\setlength\extrarowheight{5pt}
\rowcolors{2}{gray!10}{gray!40}
\renewcommand\tabularxcolumn[1]{>{\Centering}m{#1}}
\begin{tabularx}{\textwidth}{| c | c | X | X | X |} 
	\hline
	\rowcolor{white}
	\textbf{Strumento} & \textbf{Versione} & \textbf{Descrizione}\\
	\hline
	GitHub Action & -  & Strumento fornito da GitHub che permette di definire workflows personalizzati all’interno di repository.\\
	\hline
	Phpunit & 10.1 & Strumento per l'analisi statica del codice;\\
	\hline
	Xdebug & - & Strumento utilizzato per il calcolo del code coverage.\\
	\hline
	\rowcolor{white}
	\caption{Strumenti per analisi utilizzati}
\end{tabularx}
\end{center}