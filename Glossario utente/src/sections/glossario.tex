\section{Glossario}
\addcontentsline{toc}{subsection}{B}
\subsection*{B}

\paragraph{Background}~\smallskip \\
Il termine viene utilizzato per descrivere applicazioni che possono continuare ad eseguire operazioni senza che l'utente stia lavorando attivamente in una finestra dell'applicazione.

\paragraph{Bot}~\smallskip \\
Macchine automatizzate in grado di interagire con il sistema informatico per azioni a scopo malevolo.

\paragraph{Browser}~\smallskip \\
Applicazione che permette ad un utente di navigare in internet e visualizzare pagine web.

\newpage
\addcontentsline{toc}{subsection}{C}
\subsection*{C}

\paragraph{CAPTCHA}~\smallskip \\
È un test fatto per determinare se l'utente sia un umano e non un computer o, più precisamente, un bot\textsubscript{G}.

\paragraph{cURL}~\smallskip \\
Strumento che consente di trasferire i dati da o verso un web server\textsubscript{G}.

\newpage
\addcontentsline{toc}{subsection}{D}
\subsection*{D}
\paragraph{Database}~\smallskip \\
Archivio di dati utilizzato per memorizzare e gestire le informazioni, creato generalmente con il linguaggio SQL\textsubscript{G}.

\newpage
\addcontentsline{toc}{subsection}{E}
\subsection*{E}
\paragraph{Effective userID}~\smallskip \\
Viene utilizzato dal sistema operativo\textsubscript{G} per stabilire se l'utente è autorizzato ad eseguire una determinata operazione o meno.

\paragraph{Estensione}~\smallskip \\
Applicazione aggiuntiva che amplia le funzionalità disponibili all'utente.

\newpage
\addcontentsline{toc}{subsection}{F}
\subsection*{F}
\paragraph{FPM}~\smallskip \\
Strumento che gestisce i processi PHP\textsubscript{G}, ovvero riceve comandi dal web server\textsubscript{G} ed esegue script\textsubscript{G} PHP\textsubscript{G}. 

\newpage
\addcontentsline{toc}{subsection}{G}
\subsection*{G}
\paragraph{GitHub}~\smallskip \\
GitHub\textsubscript{G} è un servizio di hosting di repository\textsubscript{G} software.

\newpage
\addcontentsline{toc}{subsection}{L}
\subsection*{L}
\paragraph{Link simbolico}~\smallskip \\
E' un file che viene creato per puntare ad un altro file o cartella.

\newpage
\addcontentsline{toc}{subsection}{M}
\subsection*{M}
\paragraph{mbstring}~\smallskip \\
Strumento che offre funzionalità per gestire determinate codifiche in PHP\textsubscript{G}. 

\newpage
\addcontentsline{toc}{subsection}{N}
\subsection*{N}
\paragraph{Nginx}~\smallskip \\
E' un web server\textsubscript{G}.

\newpage
\addcontentsline{toc}{subsection}{P}
\subsection*{P}
\paragraph{Pacchetto}~\smallskip \\
I pacchetti sono insiemi di librerie, script, file che abilitano l'utente ad installare un'applicazione.

\paragraph{Percorso assoluto}~\smallskip \\
E' il percorso di un file o cartella, definito in maniera indipendente dalla cartella di lavoro corrente dell'utente.

\paragraph{PHP}~\smallskip \\
Linguaggio di scripting per la realizzazione di pagine web dinamiche, dove il contenuto della pagina web può cambiare e aggiornarsi in risposta alle azioni dell'utente.

\newpage
\addcontentsline{toc}{subsection}{R}
\subsection*{R}
\paragraph{reCAPTCHA}~\smallskip \\
E' un servizio gratuito di Google che consente di proteggere i siti web da spam e abusi permettendo di effettuare una distinzione tra gli utenti umani e i bot\textsubscript{G} automatizzati.

\paragraph{Repository}~\smallskip \\
Piattaforma software che permette di conservare una significativa quantità di informazioni. Sui dati memorizzati in questi archivi possono essere svolte numerose operazioni di protezione, classificazione, elaborazione dei documenti. In questi sistemi la gestione delle risorse informative è centralizzata e viene realizzata in un ambiente accessibile da più macchine hardware.

\newpage
\addcontentsline{toc}{subsection}{S}
\subsection*{S}
\paragraph{Script}~\smallskip \\
Programma che viene eseguito da un altro programma, che lo invoca, per eseguire una determinata operazione.

\paragraph{Server}~\smallskip \\
E' un computer in grado di comunicare e fornire funzionalità ad altri computer.

\paragraph{Servizio}~\smallskip \\
E' un processo che viene eseguito in background\textsubscript{G}, e non interagisce con l'utente.

\paragraph{Sistema operativo}~\smallskip \\
E' un programma che gestisce le risorse hardware e software del computer sul quale è installato, e permette all'utente di utilizzarne le funzionalità.

\paragraph{Spam}~\smallskip \\
Spam\textsubscript{G} indica l'invio, attraverso indirizzi generici non verificati o sconosciuti, nel nostro caso da bot\textsubscript{G}, di dati indesiderati, non richiesti o voluti.

\paragraph{SQL}~\smallskip \\
Linguaggio di programmazione utilizzato per la creazione, la modifica e la gestione dei dati dei database\textsubscript{G}.

\paragraph{sqlite3}~\smallskip \\
Libreria che consente di implementare un database\textsubscript{G} SQL\textsubscript{G}.


\newpage
\addcontentsline{toc}{subsection}{T}
\subsection*{T}
\paragraph{Task}~\smallskip \\
Compito al fine di raggiungere un obiettivo.

\newpage
\addcontentsline{toc}{subsection}{W}
\subsection*{W}
\paragraph{Web server}~\smallskip \\
E' un'applicazione che, in esecuzione su un server\textsubscript{G}, gestisce le richieste provenienti dai browser\textsubscript{G}.

